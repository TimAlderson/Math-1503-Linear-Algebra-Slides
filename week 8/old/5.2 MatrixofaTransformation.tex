%%%%%%%%%%%%%%%%%%%%%%
%%Options for presentations (in-class) and handouts (e.g. print). 
\documentclass[pdf
,handout
]{beamer}
\usepackage{pgfpages}
\pgfpagesuselayout{2 on 1}[letterpaper,border shrink=5mm]

\graphicspath{{../}}
%%%%%%%%%%%%%%%%%%%%%%
%% Change this for different slides so it appears in bar
\usepackage{authoraftertitle}
\date{Linear Transformations: Matrix of a Linear Transformation}

%%%%%%%%%%%%%%%%%%%%%%
%% Upload common style file
\usepackage{../LyryxLinearAlgebraSlidesStyle}

\begin{document}
	
	%%%%%%%%%%%%%%%%%%%%%%%
	%% Title Page and Copyright Common to All Slides	
	%Title Page
	\input ../frontmatter/titlepage.tex	
	%LOTS Page
	%\input frontmatter/lyryxopentexts.tex	
	%Copyright Page
	\input ../frontmatter/copyright.tex	
	%%%%%%%%%%%%%%%%%%%%%%%%%


\section{Linear Transformations}

%-------------- start slide -------------------------------%
\frame{\frametitle{Recall: Linear Transformations}
\begin{definition}
A transformation $T:\RR^n\rightarrow \RR^m$ is a
\alert{linear transformation} if it satisfies the following
two properties for all $\vect{x},\vect{y}\in\RR^n$ and all (scalars) $a\in\RR$.
\pause
\begin{enumerate}
\item $T(\vect{x}+\vect{y})=T(\vect{x})+T(\vect{y})$ \hfill\alert{(preservation of addition)}
\pause
\item $T(a\vect{x})=aT(\vect{x})$
\hfill\alert{(preservation of scalar multiplication)}
\end{enumerate}
\end{definition}
}
%-------------- end slide -------------------------------%

%-------------- start slide -------------------------------%
\frame{\frametitle{Matrix Transformations}

\begin{theorem}\em
Let $T:\RR^n\rightarrow\RR^m$ be a linear transformation. Then we can find an $n \times m$ matrix $A$ such that 


\[ 
T(\vect{x})=A\vect{x} 
\]

\pause 

In this case, we say that $T$ is induced, or determined, by $A$ and we write
\[
T_A(\vect{x}) = A\vect{x}
\]
\end{theorem}

}
%-------------- end slide -------------------------------%


%-------------- start slide -------------------------------%
\frame{
\begin{problem}\em
The transformation
$T:\RR^3\rightarrow\RR^4$ defined by
$T\left[\begin{array}{c} a \\ b\\ c\end{array}\right]
=\left[ \begin{array}{c} 
a+b \\ b+c \\ a-c\\ c-b \end{array}\right]$
for each $\vect{x}\in\RR^3$
is another matrix transformation, that is,
$T(\vect{x}) = A\vect{x}$ for some matrix $A$. 
\alert{Can you find a matrix $A$ that works?}
\end{problem}
\pause
\begin{solution}\em
\pause
First, since $T:\RR^3\to\RR^4$, we know that $A$ must have
size
\pause
\alert{$4\times 3$}.
\pause
Now consider the product
\[ \left[\begin{array}{rrr}
? & ? & ? \\ ? & ? & ? \\ ? & ? & ? \\
? & ? & ? \end{array}\right]
\left[\begin{array}{c} a \\ b\\ c\end{array}\right]
=
\left[ \begin{array}{c}
a+b \\ b+c \\ a-c\\ c-b \end{array}\right],
\]
and try to fill in the values of the matrix.
\end{solution}
}
%-------------- end slide -------------------------------%

%-------------- start slide -------------------------------%
\frame{
\begin{solution}[continued]\em
We can deduce from the product that $T$ is induced by
the matrix
\[ A=\left[\begin{array}{rrr}
1 & 1 & 0 \\ 0 & 1 & 1 \\ 1 & 0 & -1 \\
0 & -1 & 1 \end{array}\right]. \]
\end{solution}
}
%-------------- end slide -------------------------------%

%--------------------start slide--------------%
\frame{
\begin{alertblock}
{Is there an easier way to find the matrix of $T$?}
\pause
For some transformations guess and check will work, but this is not an efficient method. The next theorem gives a method for finding the matrix of $T$. 
\end{alertblock}

\pause

\begin{definition}
The set of columns $\{ \vect{e}_1, \vect{e}_2, \ldots, \vect{e}_n\}$ of $I_n$ is
called the \alert{standard basis of $\RR^n$.}
\end{definition}

}
%-------------- end slide -------------------------------%


%------------------start slide--------------------------%
\frame{\frametitle{Matrix and Linear Transformations}

\begin{theorem}\em
Let $T:\RR^n\rightarrow \RR^m$ be a linear transformation.
\pause
Then $T$ is a matrix transformation.
\pause
Furthermore, 
$T$ is induced by the
\alert{unique} matrix
\[ A =
\left[\begin{array}{cccc}
T(\vect{e}_1) & T(\vect{e}_2) & \cdots & T(\vect{e}_n) \end{array}\right],
\]
where $\vect{e}_j$ is the $j^{\mbox{th}}$ column of $I_n$,
and $T(\vect{e}_j)$ is the $j^{\mbox{th}}$ column of $A$.
\end{theorem}
\pause

\begin{alertblock}{Corollary}
A transformation $T:\RR^n\rightarrow \RR^m$ is a linear transformation if and only if it is a matrix transformation.
\end{alertblock}


}
%------------------------end slide-------------------------%

\section{Finding the Matrix}

%-------------- start slide -------------------------------%
{\small
\frame{
\begin{problem}\em
Let $T:\RR^2\to\RR^2$ be a linear transformation defined by
\[ T \left[ \begin{array}{c} x \\ y \end{array} \right]
=
\left[ \begin{array}{c} x+ 2y \\ x - y \end{array} \right]
\]
for each $\vect{x}\in\RR^2$.
Find the matrix, $A$, of $T$.
\end{problem}
\pause
\begin{solution}\em
\pause
To find $A$, we must find $T(\vect{e}_1)$ and $T(\vect{e}_2)$, where
$\vect{e}_1$ and $\vect{e}_2$ are the standard basis vectors of $\RR^2$.
\pause
\[
T\left[ \begin{array}{r} 1 \\ 0 \end{array} \right]
\pause
= \left[ \begin{array}{c} 1 + 2(0) \\ 1-0 \end{array} \right]
\pause
= \left[ \begin{array}{r} 1 \\ 1 \end{array} \right]
\pause
\mbox{ and }
T\left[ \begin{array}{r} 0 \\ 1 \end{array} \right] 
\pause
= \left[ \begin{array}{c} 0 + 2(1) \\ 0-1 \end{array} \right] 
\pause
= \left[ \begin{array}{r} 2 \\ -1 \end{array} \right]
\]
\pause
The columns $T(\vect{e}_1)$ and $T(\vect{e}_2)$ become the columns of $A$, 
\pause
so
\[ A = \left[ \begin{array}{rr}
1 & 2 \\ 1 & -1
\end{array} \right], \]
\vspace*{-.1in}

and $T(\vect{x}) = A\vect{x}$ for every $\vect{x}\in\RR^2$.
\pause
Therefore $A$ is the matrix for $T$.
\end{solution}
}}
%------------------------end slide-------------------------%

%------------------------start slide---------------%
\frame{\frametitle{Find the Matrix of $T$}
\begin{problem}\em
Sometimes $T$ is not defined so nicely for us. 
Suppose $T$ is given as 
\[ T\left[
\begin{array}{r}
1 \\
1
\end{array}
\right] =\left[
\begin{array}{r}
1 \\
2
\end{array}
\right] ,\ T\left[
\begin{array}{r}
0 \\
-1 
\end{array}
\right] =\left[
\begin{array}{r}
3 \\
2
\end{array}
\right]
\]

\uncover<2->{
Find the matrix $A$ of $T$.}
\end{problem}
}
%--------------------end slide-------------------%

%----------------- start slide------------------%
\frame{
\begin{solution}[continued]\em
We need to write $\vect{e}_1$ and $\vect{e}_2$ as a linear combination of the vectors provided. 
First, find $x$ and $y$ such that 
\[
\left[
\begin{array}{r}
1 \\
0
\end{array}
\right] = x\left[
\begin{array}{r}
1\\
1
\end{array}
\right] +y\left[
\begin{array}{r}
0 \\
-1 
\end{array}
\right]
\]

\uncover<2->{
Once we find $x$ and $y$ we can compute
\[
T\left[
\begin{array}{r}
1 \\
0 
\end{array}
\right]  = x T\left[
\begin{array}{r}
1 \\
1
\end{array}
\right] +y T\left[
\begin{array}{r}
0 \\
-1 
\end{array}
\right]
\]
\[
=
 x\left[
\begin{array}{r}
1 \\
2
\end{array}
\right] +y\left[
\begin{array}{r}
3 \\
2
\end{array}
\right]
\]

}
\end{solution}
}
%-----------------end slide--------------%

%-------------------start slide------------%
\frame{
\begin{solution}[continued]\em
Finding $x$ and $y$ involves solving the following system of equations.
\[
\begin{array}{c}
x = 1 \\
x - y = 0
\end{array}
\]

\uncover<2->{
The solution is $x=1, y=1$.}

\uncover<3->{
Hence, we can find $T(\vect{e}_1)$ as follows.
\[
T\left[
\begin{array}{r}
1 \\
0 
\end{array}
\right] = 
 1 \left[
\begin{array}{r}
1 \\
2
\end{array}
\right] + 1 \left[
\begin{array}{r}
3 \\
2
\end{array}
\right] 
= 
 \left[
\begin{array}{r}
1 \\
2
\end{array}
\right] + \left[
\begin{array}{r}
3 \\
2
\end{array}
\right]
=
\left[
\begin{array}{r}
4 \\
4
\end{array}
\right]
\]
}

\uncover<4->{
This is the first column of the matrix $A$. Similarly, we can find $T(\vect{e}_2)$ which will be the second column of $A$. The resulting matrix is 
\[
A
=
\left[
\begin{array}{rr}
4 & -3 \\
4 & -2
\end{array}
\right]
\]
}
\end{solution}
}
%---------------------end slide----------------------%

%-------------- start slide -------------------------------%
{\small
\frame{\frametitle{Determining if a Transformation is Linear}
\begin{example}
Let $T:\RR^2\rightarrow\RR^3$ be a transformation defined by
$T\left[\begin{array}{c} x \\ y \end{array}\right] =
\left[\begin{array}{c} 2x \\ y \\ -x +2y \end{array}\right]$.
\pause

One way to show that $T$ is a linear transformation is to show
that it \textcolor{blue}{preserves addition and scalar
multiplication}.
\pause
However, now that we know that linear transformations are
matrix transformations, we can use this to our advantage.
\pause
\medskip

\alert{If} $T$ were a linear transformation, then
\alert{$T$ would be induced by the matrix}
\pause
\vspace*{-.1in}

\[ A= \left[ \begin{array}{cc} T(\vect{e}_1) & T(\vect{e}_2)
\end{array}\right]
\pause
=
\left[\begin{array}{cc}
{T \left[\begin{array}{c} 1 \\ 0 \end{array}\right]} &
{T \left[\begin{array}{c} 0 \\ 1 \end{array}\right]}
\end{array}\right]
\pause
=
\left[\begin{array}{rr}
2 & 0 \\ 0 & 1 \\ -1 & 2
\end{array}\right].
\]
\pause
Since
\[
A\left[\begin{array}{c} x\\y \end{array}\right]
=
\left[\begin{array}{rr}
2 & 0 \\ 0 & 1 \\ -1 & 2
\end{array}\right]
\left[\begin{array}{c} x\\ y \end{array}\right]
=
\left[\begin{array}{c} 2x\\ y \\ -x+2y  \end{array}\right]
=T\left[\begin{array}{c} x\\y \end{array}\right],
\]
$T$ is a matrix transformation, and therefore a
linear transformation.
\end{example}
}
}
%------------------------end slide-------------------------%

%-------------- start slide -------------------------------%
{\small
\frame{
\begin{example}
Let $T:\RR^2\rightarrow\RR^2$ be a transformation defined by
$T\left[\begin{array}{c} x \\ y \end{array}\right] =
\left[\begin{array}{c} xy \\ x+y \end{array}\right]$.
\pause

\alert{If} $T$ were a linear transformation, then 
\alert{$T$ would be
induced by the matrix}
\pause
\[ A= \left[ \begin{array}{cc} T(\vect{e}_1) & T(\vect{e}_2)
\end{array}\right]
\pause
=
\left[\begin{array}{cc}
{T \left[\begin{array}{c} 1 \\ 0 \end{array}\right]} &
{T \left[\begin{array}{c} 0 \\ 1 \end{array}\right]}
\end{array}\right]
\pause
=
\left[\begin{array}{rr}
0 & 0 \\ 1 & 1
\end{array}\right].
\]
\pause
However,
\[ A\left[\begin{array}{c} x \\ y \end{array}\right]
=
\left[\begin{array}{rr}
0 & 0 \\ 1 & 1
\end{array}\right]
\left[\begin{array}{c} x \\ y \end{array}\right]
= \left[\begin{array}{c} 0 \\ x+y \end{array}\right].
\]
\pause
We see from this that
if $x=0$ or $y=0$, then $xy=0$, so
$A\left[\begin{array}{c} x \\ y \end{array}\right]
=T\left[\begin{array}{c} x \\ y \end{array}\right]$.
\pause

But if we take $x=y=1$, 
\pause then 
\[ A\left[\begin{array}{c} x \\ y \end{array}\right]
\pause
=A\left[\begin{array}{c} 1 \\ 1 \end{array}\right]
\pause
=\left[\begin{array}{c} 0 \\ 2 \end{array}\right]
\pause
\mbox{ while }
T\left[\begin{array}{c} x \\ y \end{array}\right]
\pause
=T\left[\begin{array}{c} 1 \\ 1 \end{array}\right]
\pause
=\left[\begin{array}{c} 1 \\ 2 \end{array}\right], \]
\pause
i.e., $A\left[\begin{array}{c} 1 \\ 1 \end{array}\right]
\neq T\left[\begin{array}{c} 1 \\ 1 \end{array}\right]$.
\pause
Therefore, $T$ in \alert{not} a linear transformation.
\end{example}
}
}
%------------------------end slide-------------------------%


\end{document}
