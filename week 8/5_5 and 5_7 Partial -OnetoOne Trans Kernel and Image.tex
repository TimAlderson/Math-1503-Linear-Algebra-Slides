%%Options for presentations (in-class) and handouts (e.g. print). 
\documentclass[pdf
%,handout
]{beamer}
\usepackage{pgfpages}
%\pgfpagesuselayout{2 on 1}[letterpaper,border shrink=5mm]

\graphicspath{{../}}

%%%%%%%%%%%%%%%%%%%%%%
%% Change this for different slides so it appears in bar
\usepackage{authoraftertitle}
\date{5.5 (partial) and 5.7:\\
	Linear Transformations: One to One and Onto, Kernel, and Image}

%%%%%%%%%%%%%%%%%%%%%%
%% Upload common style file
\usepackage{../LyryxLinearAlgebraSlidesStyle}

\begin{document}
	
	%%%%%%%%%%%%%%%%%%%%%%%
	%% Title Page and Copyright Common to All Slides
	
	%Title Page
	\input ../frontmatter/titlepage.tex
	
	%LOTS Page
	%\input frontmatter/lyryxopentexts.tex
	
	%Copyright Page
	\input ../frontmatter/copyright.tex
	
	%%%%%%%%%%%%%%%%%%%%%%%%%

\section{}
%%------------------start slide----------------------%
%\frame{\frametitle{Range of a Transformation}
%
%\begin{alertblock}{}
%Let $T: \mathbb{R}^n \mapsto \mathbb{R}^m$ be a linear transformation given by $T(\vect{x}) = A\vect{x}$. Consider all the vectors of the form $A\vect{x}$ for some $\vect{x} \in \mathbb{R}^n$.  This set of vectors is called the \alert{range} or \alert{image} of $T$. 
%
%\pause
%
%We denote this set as $T\mathbb{R}^n$, $T(\mathbb{R}^n)$ or Im$(T)$. Notice that these vectors $A(\vect{x})$ are in $\mathbb{R}^m$. 
%\end{alertblock}
%}
%%----------------end slide--------------------%
%
%%---------------start slide-------------------------%
%\frame{\frametitle{The Form $A\vect{x}$}
%
%\begin{theorem}\em
%Let $A$ be an $m \times n$ matrix where $A_1, ..., A_n$ denote the columns of $A$. Then for a vector $\vect{x} = \leftB
%\begin{array}{c}
%x_1 \\
%\vdots \\
%x_n
%\end{array}
%\rightB$ in $\mathbb{R}^n$,
%\[
%A\vect{x} = \sum_{k=1}^{n}x_{k}A_{k}
%\]
%\pause
%Therefore $A(\mathbb{R}^n)$ is the collection of all linear combinations of these products. 
%\end{theorem}
%}
%%------------------end slide-----------------------%

\section{One to One}
%--------------- start slide------------------------------%
\frame{\frametitle{Injections}
\begin{definition}
Let $T: \RR^n\to\RR^m$ be a linear transformation, and let
$\vect{x}_1$ and $\vect{x}_2$ be in $\mathbb{R}^n$.
\pause
We say that $T$ is an \alert{injection} or is
\alert{one-to-one} (sometimes written as 1-1)
if $\vect{x}_1 \neq \vect{x}_2$ implies that
\[ T\left( \vect{x}_1 \right) \neq T \left(\vect{x}_2\right).\]
\pause
Equivalently, if $T\left( \vect{x}_1 \right) =T\left( \vect{x}_2\right)$,
then $\vect{x}_1 = \vect{x}_2$.
Thus,  $T$ is one-to-one if two distinct vectors are 
never transformed into the same vector.
\end{definition}
\pause
\begin{theorem}\em
Let $A$ be an $m\times n$ matrix and let $\vect{x}$ be a vector of 
length $n$.
Then the transformation induced by $A$, $T_A$,
is one-to-one if and only if $A\vect{x}=0$ implies $\vect{x}=0$ (equivalently, iff $ A $ has full column rank).
\end{theorem}
\pause
Since every linear transformation is induced by a
matrix $A$, in order to show that $T$ is one to one,
it suffices to show that $A\vect{x}=0$ has a unique solution.
}
%--------------------------end slide -----------------------------%

%------------------------start slide--------------------------------%
{\small
\frame{
\begin{problem}\em
Show that the transformation defined by 
\[
T \left[ \begin{array}{c} x \\ y \end{array} \right]
= \left[ \begin{array}{rr} 1 & 1\\ 0 & 1 \end{array} \right]
\left[ \begin{array}{c} x \\ y \end{array} \right] \]
is one-to-one.
\end{problem}
\pause
\begin{solution}\em
Since $T$ is a matrix transformation induced by
$A= \left[\begin{array}{rr}
1 & 1 \\ 0 & 1 \end{array}\right]$, and $ A $ has full column rank (=2),
it follows that
$A\vect{x}=0$ has the unique solution $\vect{x}=0$.
\pause
From the previous theorem we conclude that $ T $ is one-to-one.\\
\pause
Alternatively, observe that
\[ \left[\begin{array}{rr|r}
1 & 1 & 0 \\ 0 & 1 & 0 \end{array}\right]
\pause
\rightarrow
\left[\begin{array}{rr|r}
1 & 0 & 0 \\ 0 & 1 & 0 \end{array}\right].
\]
 has unique solution
$\vect{x} = \left[ \begin{array}{c} 0 \\ 0 \end{array} \right]$,
so $T$ is a one-to-one. \pause  (Any other ways?) 



%it follows from the previous theorem that all
%we need to show is that
%$A\vect{x}=0$ has the unique solution $\vect{x}=0$.
%\pause
%We may do this in the standard way, by taking the augmented
%matrix of the system $A\vect{x}=0$ and putting it in 
%reduced row-echelon form.
%\pause
%\[ \left[\begin{array}{rr|r}
%1 & 1 & 0 \\ 0 & 1 & 0 \end{array}\right]
%\pause
%\rightarrow
%\left[\begin{array}{rr|r}
%1 & 0 & 0 \\ 0 & 1 & 0 \end{array}\right].
%\]
%\pause
%
%From this we see that the system has unique solution
%$\vect{x} = \left[ \begin{array}{c} 0 \\ 0 \end{array} \right]$,
%and therefore $T$ is a one-to-one. \pause  (Any other ways?) 
\end{solution}
}
}
%--------------------------end slide -----------------------------%

\section{Onto}
%-------------------- start slide ------------------------%
\frame{\frametitle{Surjections}
\begin{definition}
Let $T: \RR^n\to\RR^m$ be a linear transformation.
\pause
We say that $T$ is a \alert{surjection} or \alert{onto} if, for every $\vect{b} \in \RR^m$ there exists
an $\vect{x}\in\RR^n$ so that $T(\vect{x}) =\vect{b}$.
\end{definition}
\pause
\begin{example}
Let $T:\RR^2\to\RR^2$ be the linear transformation 
defined by
\[ T\left[\begin{array}{c} a\\ b\end{array}\right] 
=
\left[\begin{array}{c} a+b\\ 0\end{array}\right]
\mbox{ for all }\left[\begin{array}{c} a\\ b\end{array}\right]\in\RR^2.\]
\pause
Then $T$ is \alert{not onto}.  
\pause 
To see why, choose
$\vect{b}=\left[\begin{array}{c} 0\\ 1\end{array}\right] \in\RR^2$.
\pause
Then there is no vector $\vect{x}\in\RR^2$ so that $T(\vect{x})=\vect{b}$; applying
$T$ to any vector results in a vector whose second entry is \alert{$0$},
and the second entry of $\vect{b}$ is $1$.
\end{example}
}
%--------------------------end slide -----------------------------%

%-------------------- start slide ------------------------%
\frame{
\begin{example}[continued]
Consider the system $A\vect{x}=\vect{b}$, where $A$ is the matrix induced by $T$,
\pause
\vspace*{-.1in}

\[ A=\left[\begin{array}{rr} 1 & 1 \\ 0 & 0\end{array}\right].\]

\pause

Then the augmented matrix is 
\[ \left[\begin{array}{rr|r} 1 & 1 & 0 \\ 0 & 0 & 1\end{array}\right], \]
\pause
which is already in reduced row-echelon form.
\pause
The fact that this system is inconsistent implies that $T$ is 
not onto.
\end{example}
\pause
\begin{theorem}\em
Let $A$ be an $m\times n$ matrix.
\pause
Then the transformation $T_A$, induced by $A$, is onto if and only
if $A\vect{x}=\vect{b}$ is consistent for every vector $\vect{b}$ in $\RR^m$.
\end{theorem}
}
%--------------------------end slide -----------------------------%

%-------------------- start slide ------------------------%
\frame{
\begin{problem}\em
Show that the transformation defined by 
\[ T \left[ \begin{array}{c} x \\ y \end{array} \right]
= \left[ \begin{array}{rr}
1 & 2 \\ 3 & 5 
\end{array} \right]
\left[ \begin{array}{c} x \\ y \end{array} \right]
\]
is onto.
\end{problem}
\pause
\begin{solution}\em

$ T $ is a matrix transformation, $ T(\vect{x})= A\vect{x} $, where $ A =  \left[ \begin{array}{rr}
1 & 2 \\ 3 & 5 
\end{array} \right]$.\\

 \pause Since $ \det(A) = -1 \ne 0 $, $ A $ is invertible.\\

 \pause  Therefore, for every choice of $ \vect{b} $, the system  $A\vect{x}=\vect{b}$ has an unique solution (namely $\vect{x}=A^{-1}\vect{b}$) . So $ T $ is onto.

%Since $T:\RR^2\to\RR^2$, we must show that for every
%$\vect{b}=\left[ \begin{array}{c} a \\ b \end{array} \right]\in\RR^2$,
%the system $A\vect{x}=\vect{b}$ is consistent.
%\pause
%\medskip
%
%Putting the augmented matrix of $A\vect{x}=\vect{b}$ into row-echelon form,
%\pause
%\[
%\left[ \begin{array}{rr|c}
%1 & 2 & a \\ 3 & 5 & b
%\end{array} \right]
%\pause
%\rightarrow
%\left[ \begin{array}{rr|c}
%1 & 2 & a \\ 0 & -1 & b-3a
%\end{array} \right]
%\pause
%\rightarrow
%\left[ \begin{array}{rr|c}
%1 & 2 & a \\ 0 & 1 & 3a-b
%\end{array} \right].
%\]
%\pause
%We see that the system is consistent for all values of $a$ and $b$,
%and therefore $T$ is onto.
\end{solution}
}
%--------------- end slide----------------%

\section{Examples}

%----------------start slide---------------%
\frame{\frametitle{Not one-to-one}
\begin{problem}\em
Let $T$ be the linear transformation induced by 
$A=\left[ \begin{array}{rrr}
1 & 2 & -1 \\ 3 & 5 & 0
\end{array} \right]$.
Show that $T_A$ is not one-to-one.
\end{problem}
\pause
\begin{solution}\em
Let $R$ be a row-echelon form of $A$.
\[ 
A=\left[ \begin{array}{rrr}
1 & 2 & -1 \\ 3 & 5 & 0
\end{array} \right]
\rightarrow
\left[ \begin{array}{rrr}
1 & 2 & -1 \\ 0 & 1 & -3
\end{array} \right]=R
\]
%For every $\vect{b}$ in $\RR^2$, the rank of the augmented matrix 
%$[A|\vect{b}]$ is equal to two, which is the rank of $A$.
%Therefore, the system $A\vect{x}=\vect{b}$ is consistent for every $\vect{b}$,
%so $T_A$ is onto.
\bigskip

Since $A$ has rank two, $A\vect{x}=0$ has infinitely many solutions,
so $\vect{x}=0$ is not the only solution. 
Therefore, $T_A$ is not one-to-one.
\end{solution}
}
%---------------------end slide-------------------------------%

%------------------------start slide--------------------------------%
{\small
\frame{\frametitle{One-to-one}
\begin{problem}\em
Let $T$ be the linear transformation induced by 
$A=\left[ \begin{array}{rr}
1 & -1 \\ 2 & 2 \\ -1 & 2
\end{array} \right]$.
Show that $T_A$ is one-to-one.
\end{problem}
\pause
\begin{solution}\em
Let $R$ be a row-echelon form of $A$.
\[
A=\left[ \begin{array}{rr}
1 & -1 \\ 2 & 2 \\ -1 & 2
\end{array} \right]
\rightarrow
\left[ \begin{array}{rr}
1 & -1 \\ 0 & 1 \\ 0 & 0
\end{array} \right] =R\]
\vspace*{-.1in}

%There exist vectors $\vect{b}\in\RR^3$ for which the rank of
%$[A|\vect{b}]$ will be equal to three, while the rank of $A$ is
%only two.  
%Therefore, the system $A\vect{x}=\vect{b}$ is not consistent for every $\vect{b}$,
%so $T_A$ is not onto.
%\bigskip

Since $A$ has rank two, every variable in $A\vect{x}=0$ 
is a leading variable,
so $\vect{x}=0$ is the unique solution.
Therefore, $T_A$ is one-to-one.
\end{solution}
}
}
%-----------------------end slide----------------------%

%------------------------start slide--------------------------------%
\frame{\frametitle{One-to-one and onto}
\begin{problem}\em
Let $T$ be the linear transformation induced by 
$A=\left[ \begin{array}{rr}
1 & -1 \\ 2 & -1 
\end{array} \right]$.
Show that $T_A$ is one-to-one and onto.
\end{problem}
\pause
\begin{solution}\em
Let $R$ be a row-echelon form of $A$.
\[
A=\left[ \begin{array}{rr}
1 & -1 \\ 2 & -1
\end{array} \right]
\rightarrow
\left[ \begin{array}{rr}
1 & -1 \\ 0 & 1
\end{array} \right] = R\]
In this case, $A$ is invertible, so $A\vect{x}=\vect{b}$ has a \alert{unique} solution
$\vect{x}$ for every $\vect{b}$ in $\RR^2$.
Therefore $T_A$ is both one-to-one and onto.
\end{solution}
}
%-----------------------end slide----------------------%

%------------------------start slide--------------------------------%
{\footnotesize
\frame{\frametitle{Neither one-to-one nor onto}


\begin{problem}\em
Let $T$ be the linear transformation induced by 
$A=\left[ \begin{array}{rrr}
1 & -1 & 1 \\ -1 & 2 & 1 \\ 1 & 0 & 3 
\end{array} \right]$.
Show that $T_A$ is neither one-to-one nor onto.
\end{problem}
\pause
\begin{solution}\em
Let $R$ be a row-echelon form of $A$.
\[ 
A=\left[ \begin{array}{rrr}
1 & -1 & 1 \\ -1 & 2 & 1 \\ 1 & 0 & 3
\end{array} \right]
\rightarrow
\left[ \begin{array}{rrr}
1 & -1 & 1 \\ 0 & 1 & 2 \\ 0 & 0 & 0
\end{array} \right]=R\]
\vspace*{-.1in}

Since $A$ has rank two, the augmented matrix $[A|\vect{b}]$ will have rank
three for some choice of $\vect{b}\in\RR^3$, resulting in $A\vect{x}=\vect{b}$ being
inconsistent.
Therefore, $T_A$ is not onto. \pause
\medskip

The augmented matrix $[A|0]$ has rank two, so the system $A\vect{x}=0$ has
a non-leading variable, and hence does not have unique solution $\vect{x}=0$.
Therefore, $T_A$ is not one-to-one.
\end{solution}
}
%-----------------------end slide----------------------%
%-----------------start slide-------------%
\frame{\frametitle{Matrix of a One to One or Onto Transformation}
	
	\begin{Theorem}[5.34, Matrix of a One to One or Onto Transformation]
	Let $ T: \mathbb{R}^n \to \mathbb{R}^m$ be a linear transformation induced by the $ m\times n $ matrix $ A $. Then $ T $ is one to one if and only if the rank of $ A $ is $ n $. $ T $ is onto if and only if the rank of $ A $ is $ m $.
	\end{Theorem}
	
}
%----------------end slide---------------%

%-----------------start slide-------------%
\frame{\frametitle{Kernel and Image}
	
	\begin{definition}[Kernel]
		Let $V$ be a subspace of $\mathbb{R}^n$ and $W$ a subspace of $\mathbb{R}^m$, and
		let $T: V \mapsto W$ be a linear transformation. 
		\pause
		
		Then the \alert{kernel} of $T$, $\ker(T)$, consists of all $\vect{v} \in V$ such that $T(\vect{v}) = \vect{0}$. 
		
		\pause
		
		\[
		\ker(T) = \left\{ \vect{v} \in V : T(\vect{v}) = \vect{0} \right\}
		\]
		If $ A $ is the matrix corresponding to $ T $, then $ \ker(T) $ is the null-space of $ A $.
	\end{definition}
	
	\medskip
	\pause
	
	\begin{definition}[Image]
		Let $V$ be a subspace of $\mathbb{R}^n$ and $W$ a subspace of $\mathbb{R}^m$, and
		let $T: V \mapsto W$ be a linear transformation. 
		\pause
		
		Then the \alert{image} of $T$, $\im(T)$, consists of all $\vect{w} \in W$ such that $\vect{w} = T(\vect{v})$ for some $\vect{v} \in V$. 
				\pause
		\[
		\im(T) = \left\{ T(\vect{v}) : \vect{v} \in V \right\}
		\]
	If $ A $ is the matrix corresponding to $ T $, then $ \im(T) $ is the column space of $ A $.	
	\end{definition}
	
}}
%----------------end slide---------------%

%%--------------start slide-----------------%
%\frame{\frametitle{Problem to Try}
%	
%	\begin{problem}
%		Let $V$ be a subspace of $\mathbb{R}^n$ and $W$ a subspace of $\mathbb{R}^m$, and
%		let $T: V \mapsto W$ be a linear transformation.  
%		
%		\medskip
%		\pause
%		
%		Show that $\ker(T)$ is a subspace of $V$ and $\im(T)$ is a subspace of $W$. 
%	\end{problem}
%	
%}
%%------------end slide-----------------%

\section{Finding the Kernel and Image}

%--------start slide----------------%
\frame{
	\begin{example}
		Let $T: \mathbb{R}^3 \mapsto \mathbb{R}^2$ be defined by 
		\[
		T \left( \leftB \begin{array}{r}
		a \\
		b \\
		c
		\end{array} \rightB \right) = \leftB \begin{array}{c}
		a + b + c \\
		c - a
		\end{array}\rightB
		\]
		Then $T$ is a linear transformation. Find a basis for $\ker(T)$ and $\im(T)$. 
	\end{example}
	
	\medskip
	\pause
	
	\begin{solution}\em
		You can (and should!) verify that $T$ is a linear transformation.
	\end{solution}
}
%-------------end slide--------------%

%--------------start slide-----------%
\frame{
	
	\begin{solution}[continued]\em
		\textbf{Kernel of $T$:}
		We look for all vectors $\vect{x} \in \mathbb{R}^3$ such that $T(\vect{x}) = \vect{0}$. % for $\vect{0} \in \mathbb{R}^2$. 
		\vspace{-.1in}
		{\small
			\[
			T \left( \leftB \begin{array}{r}
			a \\ 
			b \\
			c
			\end{array}\rightB \right) =  \leftB \begin{array}{c}
			a + b + c \\
			c - a
			\end{array}\rightB =  \leftB \begin{array}{c}
			0 \\
			0
			\end{array}\rightB
			\]
		}
		\pause
		%\medskip
		%\vspace{-.1in}
		This gives a system of equations:
		%\vspace{-.1in}
		\begin{eqnarray*}
			a + b + c &=& 0 \\
			c - a &=& 0
		\end{eqnarray*}
		
		\pause
		%\medskip
		%\vspace{-.1in}
		The general solution is
		%\vspace{-.1in}
		{\small
			\[
			\left( \leftB \begin{array}{r} a \\ b \\ c \end{array}\rightB \right) =
			\left\{ \leftB \begin{array}{r} t \\ -2t \\ t \end{array}\rightB : t \in \mathbb{R} \right\} =
			\left\{ t \leftB \begin{array}{r} 1 \\ -2 \\ 1 \end{array}\rightB : t \in \mathbb{R} \right\}
			\]
		}
		%\vspace{-.2in}
		And therefore a basis for the kernel is {\small $
			\left\{ \leftB \begin{array}{r}
			1 \\
			-2 \\
			1
			\end{array} \rightB \right\}.
			$
		}
	\end{solution}
}
%------------end slide----------------%

%--------------start slide-----------%
\frame{
	
	\begin{solution}[continued]\em
		\textbf{Image of $T$:}
		We can write the image as %{\small
		\[ \begin{array}{cl}
		\im(T) & = \left\{ \leftB \begin{array}{c}
		a + b + c \\
		c - a
		\end{array}\rightB :  a,b,c  \in \mathbb{R} \right\} \\
		& = \left\{ \leftB \begin{array}{c} a \\ -a  \end{array} \rightB
		+ \leftB \begin{array}{c} b \\ 0  \end{array} \rightB
		+ \leftB \begin{array}{c} c \\ c  \end{array} \rightB
		:  a,b,c  \in \mathbb{R} \right\} \\
		& = \left\{ a \leftB  \begin{array}{c} 1 \\ -1  \end{array} \rightB
		+ b \leftB \begin{array}{c} 1 \\ 0  \end{array} \rightB
		+ c \leftB \begin{array}{c} 1 \\ 1  \end{array} \rightB
		:  a,b,c  \in \mathbb{R} \right\}
		\end{array}
		\]
		%}
		
		\pause
		%\medskip
		
		Thus $\im(T) = \mbox{span}\; \left\{ 
		\leftB \begin{array}{r}
		1 \\
		-1 
		\end{array}\rightB, 
		\leftB \begin{array}{r}
		1 \\
		0 
		\end{array}\rightB, 
		\leftB \begin{array}{r}
		1 \\
		1 
		\end{array}\rightB
		\right\} $.
		
		\pause
		%\medskip
		
		These vectors are not linearly independent, but the first two are so a basis for the image of $T$ is 
		{\small
			\[
			\left\{ \leftB \begin{array}{r}
			1 \\
			-1
			\end{array} \rightB, \leftB \begin{array}{r} 
			1 \\
			0
			\end{array}\rightB
			\right\}.
			\]
		}
		
	\end{solution}
}
%------------end slide----------------%

%-------------start slide------------%
\frame{\frametitle{Kernel and One to One}
	
	\begin{alertblock}{}
		The kernel of a linear transformation gives important information
		about whether the transformation is one to one. Recall that a linear
		transformation $T$ is one to one if and only if $T(\vect{x}) = \vect{0}$ implies $\vect{x} = \vect{0}$.
	\end{alertblock}
	
	\medskip
	\pause
	
	\begin{theorem}[]
		Let $T: V \mapsto W$ be a linear transformation where
		$V$ is a subspace of $\mathbb{R}^n$ and $W$ a subspace of $\mathbb{R}^m$. \\
		Then $T$ is one to one if and only if $\ker(T) = \left\{ \vect{0} \right\}$. 
	\end{theorem}
}
%-----------------end slide------------%

\section{Dimension}

%----------------start slide-----------%
\frame{\frametitle{Dimension of the Kernel and Image}
	
	\begin{theorem}[Dimension Theorem]\em
		Let $T: V \mapsto W$ be a linear transformation where
		$V$ is a subspace of $\mathbb{R}^n$ and $W$ is a subspace of $\mathbb{R}^m$. 
		Suppose further that the dimension of $V$ is $k$. Then
		\[
		k = \dim (\ker(T)) + \dim (\im(T))
		\]
	\end{theorem}
	
	\medskip
%	\pause
%	
%	\begin{corollary}[]\em
%		Let $T, V, W$ be defined as above, with $\dim(V) = k$. Then
%		\begin{eqnarray*}
%			\dim(\ker(T)) & \leq   k & \leq   n\\
%			\dim(\im(T))  & \leq   k & \leq   n \\
%		\end{eqnarray*}
%	\end{corollary}
}
%-------------------end slide----------%

%-----------------start slide---------%
\frame{
	\begin{example}[Revisited]
		Let $T: \mathbb{R}^3 \mapsto \mathbb{R}^2$ be defined by 
		\[
		T \left( \leftB \begin{array}{r}
		a \\
		b \\
		c
		\end{array} \rightB \right) = \leftB \begin{array}{c}
		a + b + c \\
		c - a
		\end{array}\rightB
		\]
		Find the dimension of $\ker(T)$ and $\im(T)$.
	\end{example}
	
	\medskip
	\pause
	
	\begin{solution}\em
		We already know that a basis for the kernel of $T$ is given by 
		\[
		\left\{ \leftB \begin{array}{r}
		1 \\
		-2 \\
		1
		\end{array} \rightB \right\}
		\]
		
		\medskip
		\pause
		
		Therefore $\dim(\ker(T)) = 1$. 
	\end{solution}
}
%-----------------end slide-----------%

%---------------start slide-----------%
\frame{
	\begin{solution}[continued]\em
		We also found a basis for the image of $T$ as 
		\[
		\left\{ \leftB \begin{array}{r}
		1 \\
		-1
		\end{array} \rightB, \leftB \begin{array}{r} 
		1 \\
		0
		\end{array}\rightB
		\right\}
		\]
		and this of course shows $\dim(\im(T)) = 2$. 
		
		\medskip
		\pause
		
		But we could have found the dimension of $\im(T)$ without finding a basis. That's because 
		since the dimension of $\mathbb{R}^3$ is $3$, and the dimension of $\ker(T)$ is 1, we get by the Dimension Theorem that:
		\begin{eqnarray*}
			\dim (\im(T)) &=& \dim (\mathbb{R}^3) - \dim (\ker(T)) \\
			&=& 3-1 =2 \\ 
		\end{eqnarray*}
	\end{solution}
}
%----------------end slide------------%

\end{document}
