%%%%%%%%%%%%%%%%%%%%%%
%%%%%%%%%%%%%%%%%%%%%%
%%Options for presentations (in-class) and handouts (e.g. print). 
%\documentclass[pdf]{beamer} 
\documentclass[pdf
%,handout
]{beamer}
%\usepackage{pgfpages}
%\pgfpagesuselayout{2 on 1}[letterpaper,border shrink=5mm]

%%%%%%%%%%%%%%%%%%%%%%
%% Change this for different slides so it appears in bar
\usepackage{authoraftertitle}
\date{Matrices: Matrix Arithmetic}

%%%%%%%%%%%%%%%%%%%%%%
%% Upload common style file
\usepackage{../LyryxLinearAlgebraSlidesStyle}

\begin{document}
	
	%%%%%%%%%%%%%%%%%%%%%%%
	%% Title Page and Copyright Common to All Slides
	
	%Title Page
	\input ../frontmatter/titlepage.tex
	
	%LOTS Page
	%\input frontmatter/lyryxopentexts.tex
	
	%Copyright Page
	\input ../frontmatter/copyright.tex
	
	%%%%%%%%%%%%%%%%%%%%%%%%%


\section{Matrices}
%-------------- start slide -------------------------------%
{\small
\frame{\frametitle{Matrices - Basic Definitions and Notation}
\begin{definitions}
Let $m$ and $n$ be positive integers.
\begin{itemize}
\item \alert{An $m\times n$ matrix} is a rectangular array
of numbers having $m$ rows and $n$ columns.  
Such a matrix is said to have \alert{size $m\times n$}.
\pause
\item \alert{A row matrix} (or row) is a $1\times n$ matrix, and
\alert{a column matrix} (or column)
is an $m\times 1$ matrix.
\pause
\item \alert{A square matrix} is an $n \times n$ matrix.
\pause
\item \alert{The $(i,j)$-entry of a matrix} is the
entry in row $i$ and column $j$. For a matrix $A$, the 
$(i,j)$-entry of $A$ is often written as $a_{ij}$.
\end{itemize}
\end{definitions}
\pause
General notation for an $m\times n$ matrix, $A$:
\vspace*{-.1in}

\[ A=\left[\begin{array}{ccccc}
a_{11} & a_{12} & a_{13} & \ldots & a_{1n} \\
a_{21} & a_{22} & a_{23} & \ldots & a_{2n} \\
a_{31} & a_{32} & a_{33} & \ldots & a_{3n} \\
\vdots & \vdots & \vdots & & \vdots \\
a_{m1} & a_{m2} & a_{m3} & \ldots & a_{mn} \\
\end{array}\right] = \left[ a_{ij} \right] \]
}}
%-------------- end slide -------------------------------%

%-------------- start slide -------------------------------%
\frame{\frametitle{Matrices -- Properties and Operations}
\pause
\begin{enumerate}
\item {\bf Equality:}
two matrices are equal if and only if they have the same
size and the corresponding entries are equal.
\pause
\item {\bf Zero Matrix:}
an $m\times n$ matrix with all entries equal to zero.
\pause
\item {\bf Addition:}
matrices must have the same size; add corresponding entries.
\pause
\item {\bf Scalar Multiplication:}
multiply each entry of the matrix by the scalar.
\pause
\item {\bf Negative of a Matrix:}
for an $m\times n$ matrix $A$, its negative is denoted $-A$ and
$-A=(-1)A$.
\pause
\item {\bf Subtraction:}
for $m\times n$ matrices $A$ and $B$,
$A-B= A+(-1)B$.
\end{enumerate}
}
%-------------- end slide -------------------------------%

\section{Matrix Addition}
%-------------- start slide -------------------------------%
\frame{\frametitle{Matrix Addition} 

\begin{definition}
Let $A = \leftB a_{ij} \rightB$ and $B = \leftB b_{ij} \rightB$ be two $m \times n$ matrices. Then $A+B = C$ where $C$ is the $m \times n$ matrix $C = \leftB c_{ij} \rightB$ defined by 
\[
c_{ij} = a_{ij} + b_{ij}
\]
\end{definition}

\pause

\begin{example}
Let $A = \leftB
\begin{array}{rr}
1 & 3 \\
2 & 5 
\end{array}\rightB, B = \leftB
\begin{array}{rr}
0 & -2 \\
6 & 1 
\end{array}
\rightB$. Then,
\begin{eqnarray*}
A + B &=& 
%\leftB
%\begin{array}{rr}
%1 + 0 & 3 + -2 \\
%2 + 6 & 5 + 1 
%\end{array}
%\rightB \\
%&=& 
%\leftB
%\begin{array}{rr}
%1 & 1 \\
%8 & 6 
%\end{array}
%\rightB
\end{eqnarray*}
\vspace*{10mm}
\end{example}
}
%-------------- end slide -------------------------------%

%-------------------start slide----------------------%
\frame{
\begin{theorem}[Properties of Matrix Addition]\em
Let $A,B$ and $C$ be $m\times n$ matrices.
Then the following properties hold. 
\pause
\begin{enumerate}
\item 
$A+B=B+A$
\textcolor{blue}{(matrix addition is commutative)}.
\medskip
\pause
\item 
$( A+B) +C=A+( B+C)$
\textcolor{blue}{(matrix addition is associative)}.
\medskip
\pause
\item 
There exists an $m\times n$ zero matrix, \alert{$0$}, such that 
\alert{$A+0=A$}.\\
\textcolor{blue}{(existence of an additive identity)}.
\medskip
\pause
\item 
There exists an $m\times n$ matrix \alert{$-A$} such that
\alert{$A+(-A) =0$}.\\
\textcolor{blue}{(existence of an additive inverse)}.
\end{enumerate}
\end{theorem}
}
%------------------end slide--------------------------%

\section{Scalar Multiplication}
%---------------- start slide --------------------------%
\frame{\frametitle{Scalar Multiplication}
\begin{definition}
Let $A = \leftB a_{ij} \rightB$ be an $m\times n$ matrix
and let $k$ be a scalar. Then $kA = \leftB ka_{ij} \rightB$.
\end{definition}
\pause

\begin{example}
Let $A = \leftB
\begin{array}{rrr}
2 & 0 & -1 \\
3 & 1 & -2 \\
0 & 4 & 5 
\end{array}
\rightB$. 

\pause
Then 
\begin{eqnarray*}
3A &=&
\invisible{\leftB
\begin{array}{rrr}
3(2) & 3(0) & 3(-1) \\
3(3) & 3(1) & 3(-2) \\
3(0) & 3(4) & 3(5) 
\end{array}
\rightB \\
&=& 
\leftB
\begin{array}{rrr}
6 & 0 & -3 \\
9 & 3 & -6 \\
0 & 12 & 15 
\end{array}
\rightB}
\end{eqnarray*}
\end{example}

}
%---------------- end slide ----------------------------%

%-----------------start slide----------------------------%
\frame{
\begin{theorem}[Properties of Scalar Multiplication]\em
Let $A, B$ be $m\times n$ matrices and let $k, p\in\RR$ (scalars).
Then the following properties hold.
\pause
\begin{enumerate}
\item 
$k \left( A+B\right) =k A+ kB$. \\
\textcolor{blue}{(scalar multiplication distributes over matrix 
addition)}.
\medskip
\pause
\item 
$\left( k +p \right) A= k A+p A$. \\
\textcolor{blue}{(addition distributes over scalar multiplication)}.
\medskip
\pause
\item 
$k \left( p A\right) = \left( k p \right) A$.
\textcolor{blue}{(scalar multiplication is associative)}.
\medskip
\pause
\item 
$1A=A$. 
\textcolor{blue}{(existence of a multiplicative identity)}.
\end{enumerate}
\end{theorem}
}
%------------------end slide-------------------------%

%-------------- start slide -------------------------------%
\frame{
\begin{example}
\[
2\left[\begin{array}{rr} -1 & 0 \\ 1 & 1 \end{array}
\right]
+
4\left[\begin{array}{rr} -2 & 1 \\ 3 & 0 \end{array}
\right]
-
\left[\begin{array}{rr} 6 & 8 \\ 1 & -1 \end{array}
\right]
=
%\pause
%\left[\begin{array}{rr} -16 & -4 \\ 13 & 3 \end{array}
%\right]
\]
\end{example}
\pause

\begin{problem}\em
Let $A$ and $B$ be $m\times n$ matrices.
Simplify the expression
\[ 2[9(A-B)+7(2B-A)] - 2[3(2B+A) - 2(A+3B) - 5(A+B)] \]
\end{problem}
\pause
\begin{solution}\em
\begin{eqnarray*}
& & 2[9(A-B)+7(2B-A)] - 2[3(2B+A) - 2(A+3B) - 5(A+B)] \\
& = & 2(9A-9B+14B-7A) -2(6B+3A-2A-6B-5A-5B) \\
& = & 2(2A + 5B) -2(-4A-5B) \\
& = & 12A + 20B
\end{eqnarray*}
\end{solution}
}
%-------------- end slide -------------------------------%

\section{Vectors}
%-------------- start slide -------------------------------%
\frame{\frametitle{Vectors}

\begin{definitions}
A row matrix or column matrix is often called a
\alert{vector}, and such matrices are referred to as 
\alert{row vectors} and \alert{column vectors},
respectively.
If $X$ is a row vector of size $1\times n$, and
$Y$ is a column vector of size $m\times 1$, 
then we write
\[
X= \left[\begin{array}{cccc} x_1 & x_2 & \cdots & x_n\end{array} \right]
 \mbox{ and }
Y= \left[\begin{array}{c} y_1 \\ y_2 \\ \vdots \\ y_m\end{array} \right]
\]
\end{definitions}
}

%-------------- end slide -------------------------------%

%-------------start slide-------------------------------%
\frame{\frametitle{Vector form of a system of linear equations}
\begin{definition}
Consider the system of linear equations
\begin{equation*}
\begin{array}{ccccccccc}
a_{11}x_{1} & + & a_{12}x_2 & + & \cdots & + & a_{1n}x_{n} & = & b_{1} \\
a_{21}x_{1} & + & a_{22}x_2 & + & \cdots & + & a_{2n}x_{n} & = & b_{2} \\
 \vdots & & \vdots & & & & \vdots & & \vdots \\
a_{m1}x_{1} & + & a_{m2}x_2 & + & \cdots & + & a_{mn}x_{n} & = & b_{m} 
\end{array}
\end{equation*}
\pause
Such a system can be expressed in \alert{vector form} 
or as a \alert{vector equation} by using
\textcolor{blue}{linear combinations} of column vectors:
\[
x_1 \left[ \begin{array}{c} a_{11}\\ a_{21}\\ \vdots \\ a_{m1}
\end{array} \right]
+
x_2 \left[ \begin{array}{c} a_{12}\\ a_{22}\\ \vdots \\ a_{m2}
\end{array} \right]
+ \cdots +
 x_n \left[ \begin{array}{c} a_{1n}\\ a_{2n}\\ \vdots \\ a_{mn}
\end{array} \right]
=
\left[ \begin{array}{c} b_1\\ b_2\\ \vdots \\ b_m \end{array} \right]
\]
\end{definition}
}
%---------------------end slide------------------------%

%---------------------start slide----------------------%
\frame{\frametitle{Vector form of a system of linear equations}
\begin{problem}\em
Express the following system of linear equations in vector form.
\[ \begin{array}{ccccccr}
2x_1 & + & 4x_2 & - & 3x_3 & = & -6 \\
  &  - & x_2  & + & 5 x_3 & = & 0 \\
x_1 & + & x_2 & + & 4x_3 & = & 1
\end{array}\]
\end{problem}
\pause
\begin{solution}\em
\[ x_1 \left[ \begin{array}{r} 2\\ 0\\ 1 \end{array} \right]
+ x_2 \left[ \begin{array}{r} 4\\ -1\\ 1 \end{array} \right]
+ x_3 \left[ \begin{array}{r} -3\\ 5\\ 4 \end{array} \right]
= \left[ \begin{array}{r} -6\\ 0\\ 1 \end{array} \right] \]
\end{solution}
}
%-----------------------end slide---------------------%

\section{Multiplication of Matrices}
%-------------- start slide -------------------------------%
\frame{\frametitle{Matrix Vector Multiplication}
\begin{definition}
Let $A=\left[ a_{ij} \right]$ be an $m\times n$ matrix with
columns $A_1, A_2, \ldots, A_n$, written
$A= \left[ \begin{array}{cccc}
A_1 & A_2 & \cdots & A_n \end{array}\right]$,
and let $X$ be an $n\times 1$ column vector, 
\vspace*{-.1in}

\[ X = \left[ \begin{array}{r} x_1 \\ x_2 \\ \vdots \\ x_n
\end{array} \right] \]
\vspace*{-.1in}

\pause
Then \alert{the product of matrix $A$ and (column) vector $X$}
is the $m\times 1$ column vector given by
\vspace*{-.35in}

\[
\left[ \begin{array}{cccc}
A_1 & A_2 & \cdots & A_n \end{array}\right]
\left[ \begin{array}{r} x_1 \\ x_2 \\ \vdots \\ x_n
\end{array} \right]
=
x_{1}A_{1}+x_{2}A_{2}+\cdots +x_{n}A_{n} = \sum_{j=1}^{n}x_{j}A_{j}  \]
\vspace*{-.1in}

that is, $AX$ is a \textcolor{blue}{linear combination} of the
columns of $A$. Notice how this is a generalization of the dot product between vectors.
\end{definition}
}
%-------------- end slide -------------------------------%

%--------------- start slide ---------------------------%
\frame{\frametitle{Matrix Vector Multiplication}
\begin{problem}\em
Compute the product $AX$ for 
\[ A= \left[ \begin{array}{rr}
1 & 4 \\ 5 & 0 \end{array} \right]\mbox{ and }
X= \left[ \begin{array}{r} 2 \\ 3 \end{array} \right] \]
\end{problem}
\pause
\begin{solution}\em
\[ AX= 
\left[ \begin{array}{rr} 1 & 4 \\ 5 & 0 \end{array} \right]
\left[ \begin{array}{r} 2 \\ 3 \end{array} \right] =
2 \left[ \begin{array}{r} 1 \\ 5 \end{array} \right]
+ 3 \left[ \begin{array}{r} 4 \\ 0 \end{array} \right]
= \left[ \begin{array}{r} 2 \\ 10 \end{array} \right]
+ \left[ \begin{array}{r} 12 \\ 0 \end{array} \right]
= \left[ \begin{array}{r} 14 \\ 10 \end{array} \right]
\]
\end{solution}
}
%----------------end slide ----------------------------%

%--------------- start slide ---------------------------%
\frame{\frametitle{Matrix Vector Multiplication}
\begin{problem}\em
Compute $AY$ for 
\[ A=\left[\begin{array}{rrrr}
1 & 0 & 2 & -1 \\
2 & -1 & 0 & 1 \\
3 & 1 & 3 & 1
\end{array}\right] \mbox{ and }
Y=\left[\begin{array}{r}
2 \\ -1 \\ 1 \\ 4
\end{array}\right]
\]
\end{problem}
\pause
\begin{solution}\em
	$ AY= $
\invisible{$AY = 2\left[\begin{array}{r} 1 \\ 2 \\ 3
\end{array}\right] +
(-1)\left[\begin{array}{r} 0 \\ -1 \\ 1
\end{array}\right] +
1\left[\begin{array}{r} 2 \\ 0 \\ 3
\end{array}\right] +
4\left[\begin{array}{r} -1 \\ 1 \\ 1
\end{array}\right] =
\left[\begin{array}{r} 0 \\ 9 \\ 12
\end{array}\right]$}
\end{solution}
}
%-------------- end slide -------------------------------%

%--------------- start slide--------------------------------%
{\small
\frame{\frametitle{Matrix form of a system of linear equations}
\begin{definition}
Consider the system of linear equations
\[ \begin{array}{ccccccccc}
a_{11}x_{1} & + & a_{12}x_2 & + &  \cdots & + & a_{1n}x_{n} & = & b_{1} \\
a_{21}x_{1} & + & a_{22}x_2 & + &  \cdots & + & a_{2n}x_{n} & = & b_{2} \\
\vdots &  & \vdots & & & & & & \vdots \\
a_{m1}x_{1} & + & a_{m2}x_2 & + &  \cdots & + & a_{mn}x_{n} & = & b_{m} 
\end{array}\]
Such a system can be expressed in
\alert{matrix form} using matrix vector multiplication,
\[ \left[ \begin{array}{cccc}
a_{11} & a_{12} & \cdots & a_{1n} \\
a_{21} & a_{22} & \cdots & a_{2n} \\
\vdots & \vdots &        & \vdots \\
a_{m1} & a_{m2} & \cdots & a_{mn}
\end{array} \right]
\left[ \begin{array}{c} x_{1} \\ x_{2} \\ \vdots \\ x_{n}
\end{array} \right]
=
\left[ \begin{array}{c} b_{1}\\ b_{2}\\ \vdots \\ b_{m}
\end{array} \right] \]
\vspace*{-.15in}
\pause

Thus a system of linear equations can be expresses as a 
\alert{matrix equation $AX=B$},
where \textcolor{blue}{$A$ is the coefficient matrix},
\textcolor{blue}{$B$ is the constant matrix}, and
\alert{$X$ is the matrix of variables.}
\end{definition}
}}
%--------------- end slide -----------------------------%

%---------------------start slide----------------------%
\frame{\frametitle{Matrix form of a system of linear equations}
\begin{problem}\em
Express the following system of linear equations in matrix form.
\[ \begin{array}{ccccccr}
2x_1 & + & 4x_2 & - & 3x_3 & = & -6 \\
  &  - & x_2  & + & 5 x_3 & = & 0 \\
x_1 & + & x_2 & + & 4x_3 & = & 1
\end{array}\]
\end{problem}
\pause
\begin{solution}\em
\[ \left[ \begin{array}{rrr}
2 & 4 & -3\\ 0 & -1 & 5\\ 1 & 1 & 4 \end{array} \right]
\left[ \begin{array}{c} x_1 \\ x_2 \\ x_3 \end{array} \right]
= \left[ \begin{array}{r} -6 \\ 0\\ 1 \end{array} \right]\]
\end{solution}
}
%-----------------------end slide---------------------%

%-------------- start slide -------------------------------%
\frame{\frametitle{Matrix and Vector Equations}
\begin{theorem}\em
\begin{enumerate}
\item
Every system of $m$ linear equations in $n$
variables can be written in the 
form $AX=B$ where $A$ is the coefficient matrix,
$X$ is the matrix of variables, and
$B$ is the constant matrix.  
\pause
\item
The system $AX=B$ is consistent (i.e., has at least one solution)
if and only if $B$ is a linear combination of the columns of $A$.
\pause
\item
The vector $X=\left[ \begin{array}{c}
x_1 \\ x_2 \\ \vdots \\ x_n \end{array}\right]$ is a solution
to the system $AX=B$ if and only if 
$x_1, x_2, \ldots, x_n$ are a solution to the 
vector equation
\[ x_1A_1 + x_2A_2 + \cdots x_nA_n=B \]
where $A_1, A_2, \ldots, A_n$ are the columns of $A$.
\end{enumerate}
\end{theorem}
}
%-------------- end slide -------------------------------%

%-------------------start slide----------------------%
\frame{\frametitle{Proof of the Theorem (a sketch)}

Every statement that deserves to be called a theorem deserves a proof, 
and the theorem from the previous slide is no exception.
 \pause 
In this particular case the proof is straightforward (i.e. uneventful). 
\pause

\begin{block}{Proof.}
(a) One first checks that $(x_1,\dots, x_n)$ is a solution to the original system if and only if
$\tiny X= \left[ \begin{array}{c} x_1 \\ x_2 \\ \vdots \\ x_n\end{array} \right]$ is a solution to $AX=B$. 
\pause

This \alert{depends on the way that the matrix arithmetics (addition, multiplication 
by scalars, multiplication) was defined}.
\end{block}
}
%-------------------end slide----------------------%

%---------------start slide-----------------------%
\frame{\frametitle{Proof continued}
\begin{proof}
(b) Once (a) is taken care of, it gives a one-to-one correspondence between 
the set of solutions to the original system and the set of solutions to $AX=B$: 
\[
\tiny (x_1,\dots, x_n)\mapsto \left[ \begin{array}{c} x_1 \\ x_2 \\ \vdots \\ x_n\end{array} \right].
\]

\pause

This \emph{is}  (3), and it  implies that the two sets have the same cardinality, and (2) follows. 
\end{proof}
}
%------------------end slide--------------------------%


%-------------- start slide -------------------------------%
{\small
\frame{
\begin{problem}\em
Let
\vspace*{-.3in}

\[ A=\left[\begin{array}{rrrr}
1 & 0 & 2 & -1 \\
2 & -1 & 0 & 1 \\
3 & 1 & 3 & 1 
\end{array}\right]
\mbox{ and } B =\left[\begin{array}{r}
1 \\ 1 \\ 1 \end{array}\right] \]
\vspace*{-.1in}

Express $B$ as a linear combination of the columns $A_1, A_2, A_3, A_4$
of $A$, or show that this is impossible.
\end{problem}
\pause
\begin{solution}\em
Solve the system $AX=B$ where $X$ is a column vector with
four entries.
\pause
Do so by putting the {\bf augmented matrix} 
$\left[ \begin{array}{c|c} A & B \end{array}\right]$
in reduced row-echelon form.
\pause
\vspace*{-.1in}

{\footnotesize
\[ \left[\begin{array}{rrrr|r}
1 & 0 & 2 & -1 & 1 \\
2 & -1 & 0 & 1 & 1 \\
3 & 1 & 3 & 1 & 1 
\end{array}\right]
\rightarrow \cdots \rightarrow
\left[\begin{array}{rrrr|r}
1 & 0 & 0 & 1 & \vspace{0.05in}\frac{1}{7} \\
0 & 1 & 0 & 1 & -\vspace{0.05in}\frac{5}{7} \\
0 & 0 & 1 & -1 & \vspace{0.05in}\frac{3}{7} 
\end{array}\right]
\]}
\vspace*{-.1in}

\pause
Since there are infinitely many solutions ($x_4$ is assigned
a parameter), choose any value for $x_4$.
\pause
Choosing $x_4=0$ (which is the simplest thing to do) gives us
\vspace*{-.2in}

\[ B = \left[\begin{array}{r} 1 \\ 1\\ 1 \end{array}\right]
= \frac{1}{7} \left[\begin{array}{r} 1 \\ 2 \\ 3
\end{array}\right] 
-\frac{5}{7}\left[\begin{array}{r} 0 \\ -1 \\ 1
\end{array}\right] +
\frac{3}{7}\left[\begin{array}{r} 2 \\ 0 \\ 3
\end{array}\right] 
=\frac{1}{7} A_{1} - \frac{5}{7} A_{2} + \frac{3}{7} A_{3} + 0 A_{4}.\]
\end{solution}
}}
%-------------- end slide -------------------------------%

%-------------- start slide -------------------------------%
\frame{\frametitle{Matrix Multiplication}
\begin{definition}[Product of two matrices]
 Let $A$ be an $m\times n$ matrix
and let
$B=\left[ \begin{array}{cccc} B_1 & B_2 & \cdots & B_{p}\end{array}
\right]$  be an $n\times p$ matrix, whose
columns are $B_1, B_2, \ldots, B_p$.
The \alert{product of $A$ and $B$} is the matrix
\[ AB=A\left[\begin{array}{cccc}
B_1 & B_2 & \cdots & B_p \end{array}\right]
=\left[\begin{array}{cccc}
AB_1 & AB_2 & \cdots & AB_p \end{array}\right] \]
i.e., the first column of $AB$ is $AB_1$,  the second
column of $AB$ is $AB_2$, etc.
Note that $AB$ has size $m \times p$.
\end{definition}
}
%-------------- end slide -------------------------------%
%\section{The $(i,j)$-Entry of a Product}
%-------------- start slide -------------------------------%
\frame{
	\begin{definition}[The $(i,j)$-entry of a product]
		Let $A=[a_{ij}]$ be an $m\times n$ matrix and $B=[b_{ij}]$ be an
		$n\times p$ matrix.
		Then the \alert{ $(i,j)$-entry of $AB$} is given by \vspace*{-4mm}
		\[ a_{i1}b_{1j}+a_{i2}b_{2j}+\cdots +a_{in}b_{nj}=
		\sum_{k=1}^{n}a_{ik}b_{kj}\]
		(Note: This can simply be viewed as the dot product of the $ i $'th row of $ A $ with the $ j$'th column of $ B $. )
	\end{definition}
	\pause
	\begin{example}
		Using the above definition,
		the $(2,3)$-entry of the product  \vspace*{-2mm}
		\[ 
		\left[\begin{array}{rrr}
		-1 & 0 & 3 \\
		\alert{2} & \alert{-1} & \alert{1}
		\end{array}\right]
		\left[\begin{array}{rrr}
		-1 & 1 & \alert{2} \\
		0 & -2 & \alert{4} \\
		1 & 0 & \alert{0}
		\end{array}\right] 
		\] \vspace*{-4mm}
		\pause
		
		is computed using the \alert{second row} of the first 
		matrix, and the \alert{third column} of the second matrix,
		\pause
		resulting in \vspace*{-4mm}
		\[ 2(2) + (-1) ( 4) + 1 (0) = 4-4+0=0.\]
	\end{example}
}
%-------------- end slide -------------------------------%

%-------------- start slide -------------------------------%
\frame{
	
\begin{problem}\em
Find the product $AB$ of matrices
\vspace*{-.2in}

\[
A= \left[\begin{array}{rrr}
-1 & 0 & 3 \\
2 & -1 & 1 
\end{array}\right]
\mbox{ and }
B= \left[\begin{array}{rrr}
-1 & 1 & 2 \\
0 & -2 & 4 \\
1 & 0 & 0 
\end{array}\right] \]
\end{problem}
\pause
\begin{solution}\em
%$AB$ has columns
%\vspace*{-.2in}
%
%\[
%AB_1 = \left[\begin{array}{rrr}
%-1 & 0 & 3 \\ 2 & -1 & 1 \end{array}\right]
%\left[\begin{array}{r}
%-1\\ 0\\ 1 \end{array}\right],
%AB_2 = \left[\begin{array}{rrr}
%-1 & 0 & 3 \\ 2 & -1 & 1 \end{array}\right]
%\left[\begin{array}{r} 1\\ -2\\ 0 \end{array}\right],
%\]
%\[ \mbox{and }
%AB_3 =
%\left[\begin{array}{rrr}
%-1 & 0 & 3 \\ 2 & -1 & 1 \end{array}\right]
%\left[\begin{array}{r} 2\\ 4\\ 0 \end{array}\right] \]
%\pause
%Thus, $AB=\left[\begin{array}{rrr}
%4 & -1 & -2 \\
%-1 & 4 & 0 
%\end{array}\right]$.
\end{solution}
\vspace*{8cm} }
%-------------- end slide -------------------------------%

%-------------- start slide -------------------------------%
\frame{\frametitle{Compatibility for Matrix Multiplication}
\begin{definition}
Let $A$ and $B$ be matrices, and suppose that $A$ is $m \times n$.
\begin{itemize}
\item
 In order for the product $AB$
to exist, the number of rows in $B$ must be equal to
the number of columns in $A$, implying that
$B$ is an $n\times p$ matrix for some $p$.  
\item
When defined, $AB$ is an \alert{$m\times p$} matrix.
\end{itemize}
If the product is defined, then $A$ and $B$ are said to be
\alert{compatible} for (matrix) multiplication.
\end{definition}
}
%-------------- end slide -------------------------------%

%-------------- start slide -------------------------------%
\frame{
\begin{example}
As we saw in the previous problem
\[
\stackrel{2\times 3}{
\left[\begin{array}{rrr}
-1 & 0 & 3 \\
2 & -1 & 1
\end{array}\right]}
\stackrel{3\times 3}
{\left[\begin{array}{rrr}
-1 & 1 & 2 \\
0 & -2 & 4 \\
1 & 0 & 0
\end{array}\right]}
=
\stackrel{2\times 3}{
\left[\begin{array}{rrr}
4 & -1 & -2 \\
-1 & 4 & 0
\end{array}\right]}
\]
\pause
Note that the product
\[
\stackrel{3\times 3}
{\left[\begin{array}{rrr}
-1 & 1 & 2 \\
0 & -2 & 4 \\
1 & 0 & 0
\end{array}\right]}
\stackrel{2\times 3}{
\left[\begin{array}{rrr}
-1 & 0 & 3 \\
2 & -1 & 1
\end{array}\right]}
\] 
does not exist.
\end{example}
}
%-------------- end slide -------------------------------%

%-------------- start slide -------------------------------%
\frame{\frametitle{Multiplication by the Zero Matrix}
\begin{example}
Compute the product $A0$ for the matrix
\[ A= \left[ \begin{array}{rr}
1 & 2 \\ 3 & 4 \end{array} \right]\]
and the $2 \times 2$ zero matrix given by $0= \left[ \begin{array}{rr}
0 & 0 \\ 0 & 0 \end{array} \right]$
\end{example}
\pause

\begin{solution}
In this product, we compute
\[ \left[ \begin{array}{rr}
1 & 2 \\ 3 & 4 \end{array} \right]
\left[ \begin{array}{rr}
0 & 0 \\ 0 & 0 \end{array} \right]
=
\left[ \begin{array}{rr}
0 & 0 \\ 0 & 0 \end{array} \right]
\]
Hence, $A0=0$.
\end{solution}
}
%-------------- end slide -------------------------------%




\section{Properties of Matrix Multiplication}
%-------------- start slide -------------------------------%
\frame{\frametitle{Questions on Matrix Multiplication}

\begin{alertblock}{}
Given matrices $A$ and $B$, is $AB=BA$? 
\end{alertblock}

\pause

\begin{block}{}
Suppose $A$ is an $m\times n$ matrix and $B$ is an $m'\times n'$ matrix. 
\pause

The product $AB$  is defined if and only if $n=m'$. 

The product $BA$  is defined if and only if $m=n'$. 
\pause

\medskip

Therefore the equation $AB=BA$ makes sense if and only if $A$ is an $m\times n$ matrix and $B$ is an $n\times m$
matrix for some---possibly different---$m$ and $n$. 
\pause

\end{block}

So the right question is:

\begin{alertblock}{}
Given matrices $A$ and $B$ such that both $AB$ and $BA$ are defined, is $AB=BA$? \end{alertblock}



}
%-------------- end slide -------------------------------%

%-------------- start slide -------------------------------%
\frame{\frametitle{Matrix Multiplication is Not Commutative}
\begin{problem}\em
Let 
\[
A= \left[\begin{array}{rr}
1 & 2 \\
-3 & 0 \\
1 & -4 \\
\end{array}\right]
\mbox{ and }
B= \left[\begin{array}{rrrr}
1 & -1 & 2 & 0 \\
3 & -2 & 1 & -3 \\
\end{array}\right] \]

\begin{itemize}
\item Does $AB$ exist?  If so, compute it.
\item Does $BA$ exist?  If so, compute it.
\end{itemize}
\end{problem}
\pause
\begin{solution}\em
\pause
\[ AB = 
\left[\begin{array}{rrrr}
7 & -5 & 4 & -6 \\
-3 & 3 & -6 & 0 \\
-11 & 7 & -2 & 12 
\end{array}\right] \]
\pause

\vspace*{-.1in}

\alert{
\[ BA \mbox{ does not exist }\] }

\vspace*{-.1in}
\end{solution}
}
%-------------- end slide -------------------------------%

%-------------- start slide -------------------------------%
\frame{
\begin{problem}\em
Let 
\[
G= \left[\begin{array}{r}
1 \\ 1 
\end{array}\right]
\mbox{ and }
H= \left[\begin{array}{rr}
1 & 0 
\end{array}\right] \]

\begin{itemize}
\item Does $GH$ exist?  If so, compute it.
\item Does $HG$ exist?  If so, compute it.
\end{itemize}
\end{problem}
\pause
\begin{solution}\em
\pause
%\[ GH = \left[\begin{array}{rr}
%1 & 0 \\ 1 & 0 
%\end{array}\right] \] 
%\pause
%\[ HG = \left[\begin{array}{r}
%1 \end{array}\right] \] 
\pause
\alert{
In this example, $GH$ and $HG$ both exist, but they are not
equal.
They aren't even the same size!}
\end{solution}\vspace*{7cm}
}
%-------------- end slide -------------------------------%

%-------------- start slide -------------------------------%
\frame{
\begin{problem}\em
Let 
\[ P= \left[\begin{array}{rr}
1 & 0 \\ 2 & -1 
\end{array}\right]
\mbox{ and }
Q= \left[\begin{array}{rr}
-1 & 1 \\ 0 & 3 
\end{array}\right] \]
\begin{itemize}
\item Does $PQ$ exist?  If so, compute it.
\item Does $QP$ exist?  If so, compute it.
\end{itemize}
\end{problem}
\pause
\begin{solution}\em
\pause
\[ PQ = \left[\begin{array}{rr}
-1 & 1 \\ -2 & -1 
\end{array}\right] \] 
\pause
\[ QP = \left[\begin{array}{rr}
1 & -1 \\ 6 & -3 
\end{array}\right] \] 
\alert{
In this example, $PQ$ and $QP$ both exist
and are the same size, but $PQ\neq QP$.}
\end{solution}
}
%-------------- end slide -------------------------------%

%-------------- start slide -------------------------------%
\frame{
\begin{fact}\em
The three preceding problems illustrate an important 
property of matrix multiplication.

\begin{quote}
In general, matrix multiplication is \alert{not}
commutative, i.e., the order of the matrices in the
product is important.
\end{quote}

In other words, in general $AB \neq BA$.
\end{fact}
}
%-------------- end slide -------------------------------%

%-------------- start slide -------------------------------%
\frame{
\begin{problem}\em
Let
\[
U= \left[\begin{array}{rr}
2 & 0 \\ 0 & 2
\end{array}\right]
\mbox{ and }
V= \left[\begin{array}{rr}
1 & 2 \\ 3 & 4
\end{array}\right]
\]

\begin{itemize}
\item Does $UV$ exist?  If so, compute it.
\item Does $VU$ exist?  If so, compute it.
\end{itemize}
\end{problem}
\pause
\begin{solution}\em
\pause
\[ UV = \left[\begin{array}{rr}
2 & 4 \\ 6 & 8
\end{array}\right] \] 
\pause
\[ VU = \left[\begin{array}{rr}
2 & 4 \\ 6 & 8
\end{array}\right] \]
\pause
In this particular example, the matrices
\alert{commute,} i.e., $UV=VU$.
\end{solution}
}
%-------------- end slide -------------------------------%

%-------------- start slide -------------------------------%
\frame{\frametitle{Properties of Matrix Multiplication}
\begin{theorem}\em
Let $A$, $B$, and $C$ be matrices of the appropriate sizes, and let
$r\in\RR$ be a scalar.  Then the following properties hold.
\pause
\begin{enumerate}
\item
$A(B+C) = AB + AC$.

\textcolor{blue}{(matrix multiplication distributes over matrix addition).}
\pause
\item
$(B+C)A = BA + CA$.

\textcolor{blue}{(matrix multiplication distributes over matrix addition).}
\pause
\item
$A\left( BC\right) =\left( AB\right) C$.
\textcolor{blue}{(matrix multiplication is associative).}
\pause
\item $r(AB)= (rA)B = A(rB)$.
\end{enumerate}
\end{theorem}
\pause

\bigskip

This applies to matrix-vector multiplication 
as well, since a vector is a row matrix or a column matrix.
}
%-------------- end slide -------------------------------%

%%-------------- start slide -------------------------------%
%\frame{
%\begin{problem} 
%Let $A = \leftB a_{ij} \rightB$, $B = \leftB b_{ij} \rightB$ 
%and $C=\leftB c_{ij} \rightB$ be three $n \times n$ matrices. 
%For $1\leq i, j\leq n$ 
%write down a formula for the $(i,j)$-entry of each of the following matrices. 
%\begin{multicols}{2}
%\begin{enumerate}
%\item AB
%\item BA
%\item A+C
%\item A(BC)
%\item (AB)C 
%\item (A+B) 
%\item C(A+B)
%\end{enumerate}
%\end{multicols}
%\end{problem}  
%}
%%-------------- end slide -------------------------------%



%-------------- start slide -------------------------------%
\frame{\frametitle{Elementary Proofs}
\begin{problem}\em
Let $A$ and $B$ be $m\times n$ matrices, and let $C$ be an
$n\times p$ matrix.
Prove that if $A$ and $B$ commute with $C$, then $A+B$ 
commutes with $C$.
\end{problem}
\pause

\begin{proof}
\invisible{We are given that $AC=CA$ and $BC=CB$.  
Consider $(A+B)C$.
\begin{eqnarray*}
(A+B)C & = & \pause AC + BC \\
\pause
& = & CA + CB \\
\pause
& = & C(A+B)
\end{eqnarray*}
\pause
Since $(A+B)C=C(A+B)$, $A+B$ commutes with $C$.}
\end{proof}
}
%-------------- end slide -------------------------------%

%-------------- start slide -------------------------------%
\frame{
\begin{problem}\em
Let $A,B$ and $C$ be $n\times n$ matrices, and suppose that
both $A$ and $B$ commute with $C$, i.e.,
$AC=CA$ and $BC=CB$.
Show that $AB$ commutes with $C$.
\end{problem}
\pause
\begin{proof}
We must show that $(AB)C=C(AB)$ given that $AC=CA$ and $BC=CB$. \pause 
\begin{eqnarray*}
(AB)C  & = & A(BC) ~~\mbox{\small (matrix multiplication is associative)}\\ \pause% 
 & = & A(CB) ~~\mbox{\small ($B$ commutes with $C$)} \\ \pause%
& = & (AC)B ~~\mbox{\small (matrix multiplication is associative)}\\ \pause%
 & = & (CA)B ~~\mbox{\small ($A$ commutes with $C$)} \\ \pause%
 & = & C(AB) ~~\mbox{\small (matrix multiplication is associative)} \pause%
\end{eqnarray*}
Therefore, $AB$ commutes with $C$.
\end{proof}
}
%-------------- end slide -------------------------------%

% !!! Put in one more elementary proof!


\section{The Transpose}
%-------------- start slide -------------------------------%
\frame{
\begin{definition}[Matrix Transpose]
If $A$ is an $m\times n$ matrix, then its \alert{transpose},
denoted \alert{$A^T$}, is the $n\times m$ 
whose $i^{th}$ row is the $i^{th}$ column of $A$, $1\leq i\leq n$;
i.e., if $A=[a_{ij}]$, then
\[ A^T= [a_{ij}]^T =
[a_{ji}] \]
i.e., the $(i,j)$-entry of $A^T$ is the $(j,i)$-entry of $A$.
\end{definition}
\pause

\begin{theorem}[Properties of the Transpose of a Matrix]\em
Let $A$ and $B$ be $m\times n$ matrices,
$C$ be a $n\times p$ matrix, and $r\in\RR$ a scalar.
Then
\pause
\begin{multicols}{2}
\begin{enumerate}
\item $(A^T)^T=A$
\pause
\item $(rA)^T = rA^T$
\pause
\item $(A+B)^T = A^T + B^T$
\pause
\item $(AC)^T=C^T A^T$
\end{enumerate}
\end{multicols}
\end{theorem}
\pause To prove each these properties, you only need to compute the $(i,j)$-entries of the 
matrices on the left-hand side and the right-hand side.\pause \alert{ And you can do it!}
}
%-------------- end slide -------------------------------%

%-------------- start slide -------------------------------%
\frame{
\begin{problem}\em
Find the matrix $A$ if
$\left( A +
3\left[ \begin{array}{rrr} 1 & -1 & 0 \\ 1 & 2 & 4 \end{array}
\right] \right)^T =
\left[ \begin{array}{rr} 2 & 1 \\ 0 & 5 \\ 3 & 8  \end{array}
\right]$.
\end{problem}
\pause
\begin{solution}
\invisible{\begin{eqnarray*}
\left( A +
3\left[ \begin{array}{rrr} 1 & -1 & 0 \\ 1 & 2 & 4 \end{array}
\right] \right)^T
& = &
\left[ \begin{array}{rr} 2 & 1 \\ 0 & 5 \\ 3 & 8  \end{array}
\right] \;\;\;\textrm{Now transpose both sides:}\\ \pause
\Rightarrow A + 3\left[ \begin{array}{rrr} 1 & -1 & 0 \\ 1 & 2 & 4 \end{array}
\right]
& = &
\left[ \begin{array}{rrr} 2 & 0 & 3 \\ 1 & 5 & 8 \end{array}
\right] \\ \pause
\Rightarrow A & = &
\left[ \begin{array}{rrr} 2 & 0 & 3 \\ 1 & 5 & 8 \end{array}
\right] -
3\left[ \begin{array}{rrr} 1 & -1 & 0 \\ 1 & 2 & 4 \end{array}
\right] \\
 & = &
\left[ \begin{array}{rrr} -1 & 3 & 3 \\ -2 & -1 & -4 \end{array}
\right]
\end{eqnarray*}}
\end{solution}
}
%-------------- end slide -------------------------------%

%-------------- start slide -------------------------------%
\frame{\frametitle{Symmetric Matrices}
\begin{definition}
Let $A=\left[ a_{ij} \right]$ be an $m\times n$ matrix.
The entries $a_{11}, a_{22}, a_{33},\ldots $ are called the
\alert{main diagonal} of $A$.
\end{definition}
\pause
\begin{definition}
The matrix $A$ is called \alert{symmetric} if and only if
$A^T=A$.  Note that this immediately implies that $A$ is
a {\bf square} matrix.
\end{definition}
\pause
\begin{examples}
\[
\left[ \begin{array}{rr}
\textcolor{blue}{2} & -3 \\
-3 & \textcolor{blue}{17}
\end{array} \right],
\left[ \begin{array}{rrr}
\textcolor{blue}{-1} & 0 & 5 \\
0 & \textcolor{blue}{2} & 11 \\
5 & 11 & \textcolor{blue}{-3}
\end{array} \right],
\left[ \begin{array}{rrrr}
\textcolor{blue}{0} & 2 & 5 & -1 \\ 
2 & \textcolor{blue}{1} & -3 & 0 \\ 
5 & -3 & \textcolor{blue}{2} & -7 \\ 
-1 & 0 & -7 & \textcolor{blue}{4} 
\end{array} \right]
\]
are symmetric matrices, and each
\textcolor{blue}{is symmetric about its main diagonal}.
\end{examples}
}
%-------------- end slide -------------------------------%
%-------------- start slide -------------------------------%
\frame{
	\begin{problem}\em
		Show that if $A$ and $B$ are symmetric matrices, then $2A-B$
		is symmetric.
	\end{problem}
	\pause
	\begin{proof}
\invisible{		\begin{eqnarray*}
			(2A-B)^T & = & (2A)^T - (B)^T \\
			\pause
			& = & 2A^T -B^T \\
			\pause
			& = & 2A -B, \mbox{ since } A^T=A\mbox{ and } B^T=B
		\end{eqnarray*}
		\pause
		Since $(2A-B)^T=2A-B$, $2A-B$ is symmetric.}
	\end{proof}
}
%-------------- end slide -------------------------------%
%%-------------- start slide -------------------------------%
%\frame{
%\begin{problem}\em
%Show that if $A$ and $B$ are symmetric matrices, then $A^T+2B$
%is symmetric.
%\end{problem}
%\pause
%\begin{proof}
%\begin{eqnarray*}
%(A^T+2B)^T & = & (A^T)^T + (2B)^T \\
%\pause
%& = & A + 2B^T \\
%\pause
%& = & A^T + 2B, \mbox{ since } A^T=A\mbox{ and } B^T=B
%\end{eqnarray*}
%\pause
%Since $(A^T+2B)^T=A^T+2B$, $A^T+2B$ is symmetric.
%\end{proof}
%}
%%-------------- end slide -------------------------------%

%-----------------------start slide-------------------------%
{\small
\frame{\frametitle{Skew Symmetric Matrices}

\begin{definition}
An $n \times n$ matrix $A$ is said to be \alert{skew symmetric} if 
$A^{T} = -A$.
\end{definition}

\begin{example}[Skew Symmetric Matrices]
\[
\left[ \begin{array}{rr}
\textcolor{blue}{0} & 2 \\
-2 & \textcolor{blue}{0}
\end{array} \right],
\left[ \begin{array}{rrr}
\textcolor{blue}{0} & 9 & 4 \\
-9 & \textcolor{blue}{0} & -3 \\
-4 & 3 & \textcolor{blue}{0} \end{array} \right]
\]
\end{example}
\pause
\begin{problem}\em
Show that if $A$ is a square matrix, then $A-A^T$ is 
skew-symmetric.
\end{problem}
\pause
\begin{solution}\em
We must show that $(A-A^T)^T=-(A-A^T)$.
\pause
Using the properties of matrix addition, scalar multiplication,
and transposition
\[ (A-A^T)^T = A^T - (A^T)^T = A^T - A=-(A-A^T).\] 
\end{solution}
}}
%-------------------------end slide--------------------------%

\section{The Identity and Inverse}
%-------------- start slide -------------------------------%
{\small
\frame{\frametitle{The $n\times n$ Identity Matrix}
\begin{definition}
For each $n\geq 2$,
the \alert{$n\times n$ identity matrix}, denoted \alert{$I_n$},
is the matrix having ones on its main diagonal and zeros 
elsewhere, and is defined for all $n\geq 2$.
\end{definition}
\begin{example}
\[ I_2 = \left[\begin{array}{rr}
1 & 0 \\ 0 & 1
\end{array}\right],
I_3 = \left[\begin{array}{rrr}
1 & 0 & 0 \\ 0 & 1 & 0 \\ 0 & 0 & 1
\end{array}\right] \]
\end{example}
\pause
\begin{definition}
Let $n\geq 2$. 
For each $j$, $1\leq j\leq n$, we denote by $E_j$ the 
$j^{\mbox{th}}$ column of $I_n$.
\end{definition} 
\begin{example}
When $n=3$, 
$E_1 = \left[\begin{array}{r}
1 \\ 0 \\ 0 \end{array}\right], 
E_2 = \left[\begin{array}{r}
0 \\ 1 \\ 0 \end{array}\right], 
E_3 = \left[\begin{array}{r}
0 \\ 0 \\ 1 \end{array}\right]$. 
\end{example}
}
%-------------------------end slide--------------------------%

%-------------- start slide -------------------------------%
\frame{
\begin{theorem}\em
Let $A$ be an $m\times n$ matrix
Then $AI_{n}=A$ and $I_{m}A=A.$
\end{theorem}

\pause

\begin{block}{Proof}
The $(i,j)$-entry of $AI_n$ is the product of the $i^{th}$ row
of $A=[a_{ij}]$, namely
$\left[\begin{array}{cccccc}
a_{i1} & a_{i2} & \cdots & a_{ij} & \cdots & a_{in} \end{array}\right]$
with the $j^{th}$ column of $I_n$, namely $E_j$.
Since $E_j$ has a one in row $j$ and zeros elsewhere,
\[ \left[\begin{array}{cccccc}
a_{i1} & a_{i2} & \cdots & a_{ij} & \cdots & a_{in} \end{array}\right] E_j
=a_{ij}\]
Since this is true for all $i\leq m$ and all $j\leq n$, $A I_n=A$. 

\alert{The proof of  $I_m A=A$ is analogous---work it out!}
\end{block}
}
%---------------------end slide---------------------_%

%---------------------start slide---------------------%
\frame{
\begin{alertblock}{}
Instead of $AI_n$ and $I_mA$ we often write $AI$ and
$IA$, respectively, since the size of the identity matrix is
clear from the context: the sizes of $A$ and $I$ must be 
compatible for matrix multiplication.
\pause

Thus
\[ AI=A\mbox{ and } IA=A \]
which is why $I$ is called an \alert{identity} matrix --
it is an identity for matrix multiplication.
\end{alertblock}
}
%-------------- end slide -------------------------------%

%-------------- start slide -------------------------------%
\frame{\frametitle{Matrix Inverses}
\begin{definition}
Let $A$ be an $n\times n$ matrix.
Then $B$ is \alert{an inverse} of $A$ if and only if
$AB=I_n$ and $BA=I_n$.
\pause
Note that since $A$ and $I_n$ are both $n\times n$, 
$B$ {\bf must also be} an $n\times n$ matrix.
\end{definition}
\pause
\begin{example}
Let
$A=\left[\begin{array}{rr}
1 & 2 \\ 3 & 4 \end{array}\right]$
and
$B=\left[\begin{array}{rr}
-2 & 1 \\ \frac{3}{2} & -\frac{1}{2} \end{array}\right]$.
Then 
\[ AB =
\left[\begin{array}{rr}
1 & 0 \\ 0 & 1 \end{array}\right]
 \]
and
\[ BA = 
\left[\begin{array}{rr}
1 & 0 \\ 0 & 1 \end{array}\right]
 \]
so $B$ is an inverse of $A$.
\end{example}
}
%-------------- end slide -------------------------------%

%-------------- start slide -------------------------------%
\frame{
\begin{alertblock}{Does every square matrix have an inverse?}
\pause No! Take e.g. the zero matrix $\mathbf{0_n}$ (all entries of $\mathbf{O_n}$ are equal to $0$)  
\[
A \mathbf{0_n}=\mathbf{0_n} A=\mathbf{O_n}
\]
for all $n\times n$ matrices $A$: 
\pause The $(i,j)$-entry of $\mathbf{O_n} A$ is equal to $\sum_{k=1}^n 0 a_{kj}=0$. 
\end{alertblock}
\pause

\begin{alertblock}{Does every \textbf{nonzero} square matrix have an inverse?}
\end{alertblock}
}
%-------------- end slide -------------------------------%

%-------------- start slide -------------------------------%
\frame{
\begin{example}
Does the matrix
\[ A=\left[\begin{array}{rr}
0 & 1 \\ 0 & 1 \end{array}\right]\]
have an inverse? \pause

\alert{No!}
To see this, suppose
\[ B=\left[\begin{array}{rr}
a & b \\ c & d \end{array}\right]\]
is an inverse of $A$. 
\pause
Then 
\[ AB=\left[\begin{array}{rr} 0 & 1 \\ 0 & 1 \end{array}\right]
\left[\begin{array}{rr} a & b \\ c & d \end{array}\right]
= \left[\begin{array}{rr} c & d \\ c & d \end{array}\right]
\]
which is never equal to $I_2$.   \alert{(Why?)}
\end{example}
}
%-------------- end slide -------------------------------%



%-------------- start slide -------------------------------%
\frame{\frametitle{Uniqueness of an Inverse}
\begin{theorem}\em
If $A$ is a square matrix and $B$ and $C$
are inverses of $A$, then $B=C$.
\end{theorem}
\pause
\begin{proof}
Since $B$ and $C$ are inverses of $A$,
$AB=I=BA$ and $AC=I=CA$.
Then
\[ B = BI \pause = B(AC) \pause = (BA)C\pause  =IC\pause  =C \]

so $B=C$.
\end{proof}
}
%-------------- end slide -------------------------------%

%-------------- start slide -------------------------------%
\frame{
\begin{example}[revisited]
For
$A=\left[\begin{array}{rr}
1 & 2 \\ 3 & 4 \end{array}\right]$
and
$B=\left[\begin{array}{rr}
-2 & 1 \\ \frac{3}{2} & -\frac{1}{2} \end{array}\right]$, we saw that
\[ AB = \left[\begin{array}{rr}
1 & 0 \\ 0 & 1 \end{array}\right]
\mbox{ and }
BA = \left[\begin{array}{rr}
1 & 0 \\ 0 & 1 \end{array}\right] \]
The preceding theorem tells us that
$B$ is \alert{the inverse} of $A$,
rather than just
\textcolor{blue}{an inverse} of $A$.
\end{example}
}
%-------------- start slide -------------------------------%
\frame{
\begin{definitions}
Let $A$ be a square matrix, i.e., an $n\times n$ matrix.
\begin{itemize}
\item
\alert{The} inverse of $A$, if it exists, is
denoted $A^{-1}$, and
\[ AA^{-1}=I=A^{-1}A \]
\item<2->
If $A$ has an inverse, then we say that $A$ is
\alert{invertible} (or \alert{nonsingular}).
\end{itemize}
\end{definitions}
}
%-------------- end slide -------------------------------%

\section{Finding the Inverse of a Matrix}
%-------------- start slide -------------------------------%
\frame{\frametitle{Finding the inverse of a $2\times 2$ matrix}
\begin{example}
Suppose that $A=\left[\begin{array}{rr}
a & b \\ c & d\end{array}\right]$.
\pause
\alert{If $ad-bc\neq 0$}, then there is a formula for $A^{-1}$:
\[ A^{-1} =
\frac{1}{ad-bc}
\left[\begin{array}{rr}
d & -b \\ -c & a \end{array}\right] \]
\pause
This can easily be verified by computing the products
$A A^{-1}$ and $A^{-1}A$. 
\pause
\begin{eqnarray*}
A A^{-1} & = & 
\left[\begin{array}{rr} a & b \\ c & d\end{array}\right]
\frac{1}{ad-bc} 
\left[\begin{array}{rr} d & -b \\ -c & a \end{array}\right] \\
& = &
\frac{1}{ad-bc} 
\left(
\left[\begin{array}{rr} a & b \\ c & d\end{array}\right]
\left[\begin{array}{rr} d & -b \\ -c & a \end{array}\right] 
\right)\\
& = & 
\frac{1}{ad-bc} 
\left[\begin{array}{cc} ad-bc & 0 \\ 0 & -bc+ad\end{array}\right]
= \left[\begin{array}{rr} 1 & 0 \\ 0 & 1\end{array}\right]
\end{eqnarray*}
\pause
Showing that $A^{-1}A=I_2$ is left as an exercise.
\end{example}
}
%-------------- end slide -------------------------------%

%-------------- start slide -------------------------------%
\frame{\frametitle{Finding the inverse of an $n\times n$ matrix}
\begin{problem}\em
Suppose that $A$ is any $n \times n$ matrix.
\pause
\begin{itemize}
\item
How do we know whether or not $A^{-1}$ exists?
\pause
\item
If $A^{-1}$ exists, how do we find it?
\end{itemize}
\end{problem}
\begin{solution}\em
\alert{The matrix inversion algorithm.}
\end{solution}

\pause

\begin{alertblock}{}
Although the formula for the inverse of a $2\times 2$ matrix is 
quicker and easier to use than the matrix inversion algorithm,
the general formula for the inverse an $n\times n$ matrix, $n\geq 3$
(which we will see later), is more
complicated and difficult to use than the matrix inversion algorithm.
To find inverses of square matrices that are not $2\times 2$, 
the matrix inversion algorithm is the most efficient method to use.
\end{alertblock}
}
%-------------- end slide -------------------------------%

%-------------- start slide -------------------------------%
\frame{
\begin{block}{The Matrix Inversion Algorithm}
Let $A$ be an $n\times n$ matrix.
To find $A^{-1}$, if it exists, 
\begin{itemize}
\item take the $n\times 2n$ matrix
\[ \left[\begin{array}{c|c} A & I_n \end{array}\right] \]
obtained by augmenting $A$ with the $n\times n$
identity matrix, $I_n$.
\item Perform elementary row operations to transform 
$\left[\begin{array}{c|c} A & I_n \end{array}\right]$ into
a reduced row-echelon matrix.
\end{itemize}
\end{block}
\pause
\begin{theorem}[Matrix Inverses]
Let $A$ be an $n\times n$ matrix.
Then the following conditions are equivalent.
\begin{enumerate}
\item $A$ is invertible.
\item the reduced row-echelon form on $A$ is $I$.
\item $\left[\begin{array}{c|c} A & I_n \end{array}\right]$
can be transformed into 
$\left[\begin{array}{c|c} I_n & A^{-1} \end{array}\right]$
using the Matrix Inversion Algorithm.
\end{enumerate}
\end{theorem}
}
%-------------- end slide -------------------------------%

%-------------- start slide -------------------------------%
\frame{
\begin{problem}\em
Find, if possible, the inverse of 
$\left[\begin{array}{rrr}
1 & 0 & -1 \\
-2 & 1 & 3 \\
-1 & 1 & 2
\end{array}\right]$.
\end{problem}
\pause
\begin{solution}\em
Using the matrix inversion algorithm (fill in the operations)
\pause
\[
\left[\begin{array}{rrr|rrr}
1 & 0 & -1 & 1 & 0 & 0 \\
-2 & 1 & 3 & 0 & 1 & 0 \\
-1 & 1 & 2 & 0 & 0 & 1
\end{array}\right] \pause \rightarrow
\left[\begin{array}{rrr|rrr}
1 & 0 & -1 & 1 & 0 & 0 \\
0 & 1 & 1 & 2 & 1 & 0 \\
0 & 1 & 1 & 1 & 0 & 1
\end{array}\right] \pause \rightarrow
\]
\[
\left[\begin{array}{rrr|rrr}
1 & 0 & -1 & 1 & 0 & 0 \\
0 & 1 & 1 & 2 & 1 & 0 \\
0 & 0 & 0 & -1 & -1 & 1
\end{array}\right]
\]
\bigskip

From this, we see that \alert{$A$ has no inverse}.
\end{solution}
}
%-------------- end slide -------------------------------%

%-------------- start slide -------------------------------%
\frame{
\begin{problem}\em
Let $A=
\left[\begin{array}{rrr}
3 & 1 & 2 \\
1 & -1 & 3 \\
1 & 2 & 4
\end{array}\right]$.  Find the inverse of $A$, if it exists.
\end{problem}
}
%-------------- end slide -------------------------------%

%-------------- start slide -------------------------------%
\frame{
\begin{solution}[continued]
Using the matrix inversion algorithm (fill in the operations)
\pause
\[ \left[\begin{array}{c|c} A & I \end{array}\right]
=\left[\begin{array}{rrr|rrr}
3 & 1 & 2 & 1 & 0 & 0 \\
1 & -1 & 3 & 0 & 1 & 0 \\
1 & 2 & 4 & 0 & 0 & 1
\end{array}\right]
\pause
\rightarrow
\left[\begin{array}{rrr|rrr}
1 & -1 & 3 & 0 & 1 & 0 \\
3 & 1 & 2 & 1 & 0 & 0 \\
1 & 2 & 4 & 0 & 0 & 1
\end{array}\right]
\rightarrow
\]
\[
\left[\begin{array}{rrr|rrr}
1 & -1 & 3 & 0 & 1 & 0 \\
0 & 4 & -7 & 1 & -3 & 0 \\
0 & 3 & 1 & 0 & -1 & 1
\end{array}\right]
\rightarrow
\left[\begin{array}{rrr|rrr}
1 & -1 & 3 & 0 & 1 & 0 \\
0 & 1 & -8 & 1 & -2 & -1 \\
0 & 3 & 1 & 0 & -1 & 1
\end{array}\right]
\rightarrow
\]
\[
\left[\begin{array}{rrr|rrr}
1 & 0 & -5 & 1 & -1 & -1 \\
0 & 1 & -8 & 1 & -2 & -1 \\
0 & 0 & 25 & -3 & 5 & 4
\end{array}\right]
\rightarrow
\left[\begin{array}{rrr|rrr}
1 & 0 & -5 & 1 & -1 & -1 \\
0 & 1 & -8 & 1 & -2 & -1 \\
0 & 0 & 1 & -\frac{3}{25} & \frac{5}{25} & \frac{4}{25}
\end{array}\right]
\rightarrow
\]
\[
\left[\begin{array}{rrr|rrr} \vspace*{.02in}
1 & 0 & 0 & \frac{10}{25} & 0 & -\frac{5}{25} \\ \vspace*{.02in}
0 & 1 & 0 & \frac{1}{25} & -\frac{10}{25} & \frac{7}{25} \\
0 & 0 & 1 & -\frac{3}{25} & \frac{5}{25} & \frac{4}{25}
\end{array}\right]
= \left[\begin{array}{c|c} I & A^{-1} \end{array}\right]
\]
\end{solution}
}
%-------------- end slide -------------------------------%

%-------------- start slide -------------------------------%
\frame{
\begin{solution}[continued]\em
Therefore, $A^{-1}$ exists, and
\[ A^{-1}=
\left[\begin{array}{rrr} \vspace*{.02in}
\frac{10}{25} & 0 & -\frac{5}{25} \\ \vspace*{.02in}
\frac{1}{25} & -\frac{10}{25} & \frac{7}{25} \\
-\frac{3}{25} & \frac{5}{25} & \frac{4}{25}
\end{array}\right]
=\frac{1}{25}
\left[\begin{array}{rrr}
10 & 0 & -5 \\
1 & -10 & 7 \\
-3 & 5 & 4
\end{array}\right]
\]
\pause
\alert{You can check your work by computing $AA^{-1}$ and $A^{-1}A$.}
\end{solution}
}
%-------------- end slide -------------------------------%

%-------------- start slide -------------------------------%
\frame{\frametitle{Systems of Linear Equations and Inverses}
Suppose that a system of $n$ linear equations in $n$ variables
is written in matrix form as
\alert{$AX=B$},
and suppose that $A$ is invertible.
\pause
\begin{example}
The system of linear equations
\begin{eqnarray*}
2x-7y & = & 3 \\
5x-18y & = & 8 
\end{eqnarray*}
can be written in matrix form as 
$AX=B$:
\[
\left[\begin{array}{rr}
2 & -7 \\ 5 & -18 \end{array}\right]
\left[\begin{array}{r}
x \\ y \end{array}\right]
=
\left[\begin{array}{r}
3 \\ 8 \end{array}\right] \]
\pause
You can check that
$A^{-1}= \left[\begin{array}{rr}
18 & -7 \\ 5 & -2 \end{array}\right]$.
\end{example}
}
%-------------- end slide -------------------------------%

%-------------- start slide -------------------------------%
\frame{
\begin{example}[continued]
Since $A^{-1}$ exists and has the property
that $A^{-1}A=I$, we obtain the following.
\pause
\vspace*{-.21in}

\begin{eqnarray*}
AX & = & B \\ 
\pause
A^{-1}(AX) & = & A^{-1}B \\
\pause
(A^{-1}A)X & = & A^{-1}B \\
\pause
IX & = & A^{-1}B \\
\pause
X & = & A^{-1}B
\end{eqnarray*}
\vspace*{-.21in}

\pause
i.e., $AX=B$ has the \alert{unique solution} given by
\alert{$X=A^{-1}B$}.
\pause
Therefore, 
\[ X=A^{-1}\left[\begin{array}{r} 3 \\ 8 \end{array}\right]
= \left[\begin{array}{rr}
18 & -7 \\ 5 & -2 \end{array}\right]
\left[\begin{array}{r} 3 \\ 8 \end{array}\right]
= \left[\begin{array}{r} -2 \\ -1 \end{array}\right]
\]
\pause
\alert{You should verify that $x=-2$, $y=-1$ is a solution to the
system.}
\end{example}
}
%-------------------end slide-------------------%

%---------------------start slide-----------------%
\frame{
\begin{alertblock}{}
The last example illustrates another method for solving a system of linear
equations when {\bf the coefficient matrix is square and invertible}.
\pause
Unless that coefficient matrix is $2\times 2$, this is generally
{\bf NOT} an efficient method for solving a system of linear
equations.
\end{alertblock}
}
%-------------- end slide -------------------------------%

%-------------- start slide -------------------------------%
{\small
\frame{
\begin{example}
Let $A, B$ and $C$ be matrices, and suppose
that $A$ is invertible.
\begin{enumerate}
\item
If $AB=AC$, then
\pause
\begin{eqnarray*}
A^{-1}(AB) & = & A^{-1}(AC) \\
\pause
(A^{-1}A)B & = & (A^{-1}A)C \\
\pause
IB & = & IC \\
\pause
B & = & C
\end{eqnarray*}
\pause
\vspace*{-.2in}

\item
If $BA=CA$, then
\pause
\begin{eqnarray*}
(BA)A^{-1} & = & (CA)A^{-1} \\
B(AA^{-1}) & = & C(AA^{-1}) \\
BI & = & CI \\
B & = & C
\end{eqnarray*}
\end{enumerate}
\end{example}
\pause
\begin{problem}\em
Find square matrices $A, B$ and $C$ for which $AB=AC$ but $B\neq C$.
\end{problem}
}}
%-------------- end slide -------------------------------%

\section{Properties of the Inverse}
%-----------------------start slide-------------------%
\frame{\frametitle{Inverses of Transposes and Products}
\begin{example}
Suppose $A$ is an invertible matrix.  
Then
\[ A^T(A^{-1})^T =
\pause (A^{-1}A)^T = 
\pause I^T = 
\pause I \]
and
\[ (A^{-1})^TA^T 
\pause = (AA^{-1})^T 
\pause = I^T 
\pause = I \]
This means that \alert{$(A^T)^{-1}=(A^{-1})^T$}.
\end{example}
\pause
\begin{example}
Suppose $A$ and $B$ are invertible $n\times n$ matrices.  Then
\[ (AB)(B^{-1}A^{-1})
\pause = A(BB^{-1})A^{-1}
\pause =AIA^{-1}
\pause =AA^{-1}
\pause =I \]
\pause
and
\[ (B^{-1}A^{-1})(AB)
\pause =B^{-1}(A^{-1}A)B
\pause =B^{-1}IB
\pause =B^{-1}B
\pause =I \]
\pause
This means that \alert{$(AB)^{-1}=B^{-1}A^{-1}$}.
\end{example}
}
%-------------- end slide -------------------------------%

%-------------- start slide -------------------------------%
\frame{\frametitle{Inverses of Transposes and Products}
\begin{alertblock}{}
The previous two examples prove the first two parts of the
following theorem.
\end{alertblock}
\pause
\begin{theorem}\em
\begin{enumerate}
\item
If $A$ is an invertible matrix, then $(A^T)^{-1}=(A^{-1})^T$.
\pause
\item
If $A$ and $B$ are invertible matrices, then $AB$ is invertible
and
\[ (AB)^{-1}=B^{-1}A^{-1} \]
\pause
\item If $A_1, A_2, \ldots, A_k$ are invertible, then
$A_1A_2\cdots A_k$ is invertible and
\[ (A_1A_2\cdots A_k)^{-1}=
A_k^{-1} A_{k-1}^{-1}\cdots A_2^{-1} A_1^{-1} \]
\textcolor{blue}{(the third part is proved by iterating the above, or, more formally, by using \alert{mathematical induction})}
\end{enumerate}
\end{theorem}
}
%-----------------------end slide------------------------%

%-------------- start slide -------------------------------%
\frame{\frametitle{Properties of Inverses}
\begin{theorem}\em
\begin{enumerate}
\item
$I$ is invertible, and $I^{-1}=I$.
\pause
\item
If $A$ is invertible, so is $A^{-1}$, and
$(A^{-1})^{-1}=A$.
\pause
\item
If $A$ is invertible, so is $A^k$, and
$(A^k)^{-1}=(A^{-1})^k$.

\textcolor{blue}{($A^k$ means $A$ multiplied by itself $k$ times)}
\pause
\item
If $A$ is invertible and $p\in\RR$ is nonzero, then
$pA$ is invertible, and $(pA)^{-1}=\frac{1}{p}A^{-1}$.
\end{enumerate}
\end{theorem}
}
%-------------- end slide -------------------------------%

%-------------- start slide -------------------------------%
\frame{
\begin{example}
Given $(3I - A^T)^{-1}=2
\left[\begin{array}{rr}
1 & 1 \\ 2 & 3 \end{array}\right]$,
we wish to find the matrix $A$.
\pause
Taking inverses of both sides of the equation:
\invisible{\begin{eqnarray*}
3I - A^T & = & \left(2\left[\begin{array}{rr}
1 & 1 \\ 2 & 3 \end{array}\right] \right)^{-1} 
\pause \\
& = & \frac{1}{2} 
\left[\begin{array}{rr}
1 & 1 \\ 2 & 3 \end{array}\right]^{-1}
\pause \\
& = & \frac{1}{2} 
\left[\begin{array}{rr}
3 & -1 \\ -2 & 1 \end{array}\right]
\pause \\
& = & 
\left[\begin{array}{rr} \vspace*{.02in}
\frac{3}{2} & -\frac{1}{2} \\ 
-1 & \frac{1}{2} \end{array}\right]
\end{eqnarray*}}
\end{example}
}
%-------------- end slide -------------------------------%

%-------------- start slide -------------------------------%
\frame{
\begin{example}[continued]
\invisible{\begin{eqnarray*}
3I - A^T & = &
\left[\begin{array}{rr} \vspace*{.02in}
\frac{3}{2} & -\frac{1}{2} 
\\ 
-1 & \frac{1}{2} \end{array}\right]
\pause \\
-A^T & = &
\left[\begin{array}{rr} \vspace*{.02in}
\frac{3}{2} & -\frac{1}{2} \\ 
-1 & \frac{1}{2} \end{array}\right] - 3I
\pause \\
-A^T & = &
\left[\begin{array}{rr} \vspace*{.02in}
\frac{3}{2} & -\frac{1}{2} \\ 
-1 & \frac{1}{2} \end{array}\right] -
\left[\begin{array}{rr} 
3 & 0 \\ 0 & 3
\end{array}\right]
\pause \\
-A^T & = &
\left[\begin{array}{rr} \vspace*{.02in}
-\frac{3}{2} & -\frac{1}{2} \\ 
-1 & -\frac{5}{2} \end{array}\right] 
\pause \\
A & = & 
\left[\begin{array}{rr} \vspace*{.02in}
\frac{3}{2} & 1 \\ 
\frac{1}{2} & \frac{5}{2} \end{array}\right] \\
\end{eqnarray*}}
\end{example}
}
%-------------- end slide -------------------------------%

%-------------- start slide -------------------------------%
\frame{
\begin{problem}\em
True or false?  Justify your answer.
\begin{center}
If $A^3=4I$, then $A$ is invertible.
\end{center}
\end{problem}
\pause
\begin{solution}\em
\invisible{If $A^3=4I$, then
\[ \frac{1}{4} A^3 = I \]
\pause
so
\[ (\frac{1}{4} A^2)A = I \mbox{ and }
\pause
A(\frac{1}{4} A^2) = I \]
\pause

Therefore $A$ is invertible, and $A^{-1}=\frac{1}{4}A^2$.}
\end{solution}
}
%-------------- end slide -------------------------------%

%-------------- start slide -------------------------------%
\frame{\frametitle{A Fundamental Result}

\begin{theorem}
Let $A$ be an $n\times n$ matrix, and
let $X$, $B$ be $n \times 1$ vectors.
The following conditions are equivalent.
\pause
\begin{enumerate}
\item The rank of $A$ is $n$. 
\item
$A$ can be transformed to $I_n$ by elementary row operations.
\pause
\item
$A$ is invertible.
\pause
\item
There exists an $n\times n$ matrix $C$ with the property that
$CA=I_n$.
\pause
\item
The system $AX=B$ has a unique solution $X$ for any
choice of $B$.
\item
$AX=0$ has only the trivial solution, $X=0$.
\pause
\item
There exists an $n\times n$ matrix $C$ with the property that
$AC=I_n$.
\end{enumerate}
\end{theorem}
 }
%-----------------end slide---------------------------------%

%-------------- start slide -------------------------------%
{\small
\frame{

\begin{block}{Proof of Theorem:} 
(1) $\Rightarrow$ (2) The rank of $A$ is the number of leading 1s in the RREF of $A$. 
Since the size of $A$ is $n\times n$,   $\rank(A)=n$ \alert{is equivalent to} $A$ being row-equivalent to $I_n$. 


\pause

(2) $\Rightarrow$ (3): Matrix inversion algorithm.



\pause
(3) $\Rightarrow$ (4): $C=A^{-1}$. 


\pause
(4) $\Rightarrow$ (5): $X=CB$. 

\pause
(5) $\Rightarrow$ (6): Take $B=0$. 

\pause
(6) $\Rightarrow$ (1): If rank of $A$ is $<n$, then there are non-leading variables in the RREF of $[A|0]$. 
Hence $AX=0$ has  infinitely many solutions. 

\pause

(4) $\Leftrightarrow$ (7): 
 $CA=I$ if and only if $A^TC^T=I$; hence (4) for $A$ is equivalent to (7) for $A^T$. 
\pause

We already know that $A^{-1}$ exists if and only if $(A^T)^{-1}$ exists. 

\end{block} 

 }
%-----------------end slide---------------------------------%

%-------------- start slide -------------------------------%
{\small
\frame{
\begin{alertblock}{}
The following is an important and useful consequence of the 
theorem.
\end{alertblock}
\pause
\begin{theorem}
If $A$ and $B$ are $n\times n$ matrices such that $AB=I$, then
$BA=I$.
Furthermore, $A$ and $B$ are invertible, with $B=A^{-1}$ and
$A=B^{-1}$.
\end{theorem}
\pause
\begin{alertblock}{Important Fact}
In the second Theorem, it is essential that the matrices be square.
\end{alertblock}
}}
%-------------- end slide -------------------------------%


%-------------- start slide -------------------------------%
\frame{
\begin{theorem}
If $A$ and $B$ are matrices such that $AB=I$ and $BA=I$, then 
$A$ and $B$ are square matrices (of the same size).
\end{theorem}
}
%-------------- end slide -------------------------------%

%-------------- start slide -------------------------------%
{\small
\frame{
\begin{example}
Let
$A=\left[\begin{array}{rrr}
1 & 1 & 0 \\ -1 & 4 & 1 \end{array}\right]
\mbox{ and }
B=\left[\begin{array}{rr}
1 & 0 \\ 0 & 0 \\ 1 & 1 \end{array}\right]$.
\pause
Then
\[ AB= 
\left[\begin{array}{rrr}
1 & 1 & 0 \\ -1 & 4 & 1 \end{array}\right]
\left[\begin{array}{rr}
1 & 0 \\ 0 & 0 \\ 1 & 1 \end{array}\right] 
\pause 
= \left[ \begin{array}{rr}
1 & 0 \\ 0 & 1 \end{array} \right] =I_2 \]
\pause
\vspace*{-.2in}

and
\vspace*{-.2in}

\[ BA= 
\left[\begin{array}{rr}
1 & 0 \\ 0 & 0 \\ 1 & 1 \end{array}\right]
\left[\begin{array}{rrr}
1 & 1 & 0 \\ -1 & 4 & 1 \end{array}\right]
\pause
= \left[ \begin{array}{rrr}
1 & 1 & 0 \\
0 & 0 & 0 \\
0 & 5 & 1
\end{array} \right] \neq I_3 \]
\end{example}
\pause
\begin{alertblock}{}
This example illustrates why ``an inverse'' of a non-square matrix
doesn't make sense.
If $A$ is $m\times n$ and $B$ is $n\times m$, where $m\neq n$,
then even if $AB=I$, it will never be the case that $BA=I$.
\end{alertblock}
}}
%-------------- end slide -------------------------------%

%
%\section{Elementary Matrices}
%%-------------- start slide -------------------------------%
%\frame{\frametitle{Elementary Matrices}
%\begin{definition}
%An \alert{elementary matrix} is a matrix obtained from an
%identity matrix by performing \alert{a single} elementary 
%row operation.
%\end{definition}
%\pause
%\bigskip
%
%The \alert{type} of an elementary matrix is given by the type of row operation 
%used to obtain the elementary matrix. 
%Recall the elementary row operations.
%\pause
%\bigskip
%
%\begin{block}{Elementary Row Operations.}
%\pause
%\begin{itemize}
%\item {\bf Type I:} Interchange two rows.
%\pause
%\item {\bf Type II:} Multiply a row by a nonzero number.
%\pause
%\item {\bf Type III:} 
%Add a (nonzero) multiple of one row to a different row.
%\end{itemize}
%\end{block}
%}
%%-----------------start slide------------------------%
%\frame{
%\begin{example}
%\[ E =
%\left[ \begin{array}{cccc}
%1 & 0 & 0 & 0 \\
%0 & 0 & 0 & 1 \\
%0 & 0 & 1 & 0 \\
%0 & 1 & 0 & 0
%\end{array}\right],
%F=
%\left[ \begin{array}{rrrr}
%1 & 0 & 0 & 0 \\
%0 & 1 & 0 & 0 \\
%0 & 0 & -2 & 0 \\
%0 & 0 & 0 & 1 
%\end{array}\right],
%G=
%\left[ \begin{array}{rrrr}
%1 & 0 & 0 & 0 \\
%0 & 1 & 0 & 0 \\
%-3 & 0 & 1 & 0 \\
%0 & 0 & 0 & 1 
%\end{array}\right],
%\]
%are examples of elementary matrices of types I, II and III, 
%respectively.
%\pause
%
%Let 
%\[A= \left[ \begin{array}{rr}
%1 & 1 \\ 2 & 2 \\ 3 & 3 \\ 4 & 4 
%\end{array}\right] \]
%\pause
%We are interested in the effect that (left) multiplication of
%$A$ by $E$, $F$ and $G$ has on the matrix $A$.
%\pause
%Computing $EA$, $FA$, and $GA$ \ldots
%\end{example}
%}
%%-------------- end slide -------------------------------%
%
%%-------------- start slide -------------------------------%
%{\small
%\frame{
%\begin{example}[continued]
%\[
%EA=
%\left[ \begin{array}{cccc}
%1 & 0 & 0 & 0 \\
%0 & 0 & 0 & 1 \\
%0 & 0 & 1 & 0 \\
%0 & 1 & 0 & 0
%\end{array}\right]
%\left[ \begin{array}{rr}
%1 & 1 \\ 2 & 2 \\ 3 & 3 \\ 4 & 4 
%\end{array}\right]=
%\left[ \begin{array}{rr}
%1 & 1 \\
%4 & 4 \\
%3 & 3 \\
%2 & 2 
%\end{array}\right]
%\]
%\pause
%\[
%FA=
%\left[ \begin{array}{rrrr}
%1 & 0 & 0 & 0 \\
%0 & 1 & 0 & 0 \\
%0 & 0 & -2 & 0 \\
%0 & 0 & 0 & 1 
%\end{array}\right]
%\left[ \begin{array}{rr}
%1 & 1 \\ 2 & 2 \\ 3 & 3 \\ 4 & 4
%\end{array}\right]=
%\left[ \begin{array}{rr}
%1 & 1 \\ 
%2 & 2 \\
%-6 & -6 \\ 
%4 & 4 \\
%\end{array}\right]
%\]
%\pause
%\[
%GA=
%\left[ \begin{array}{rrrr}
%1 & 0 & 0 & 0 \\
%0 & 1 & 0 & 0 \\
%-3 & 0 & 1 & 0 \\
%0 & 0 & 0 & 1 
%\end{array}\right]
%\left[ \begin{array}{rr}
%1 & 1 \\ 2 & 2 \\ 3 & 3 \\ 4 & 4
%\end{array}\right]=
%\left[ \begin{array}{rr}
%1 & 1 \\
%2 & 2 \\
%0 & 0 \\
%4 & 4 \\
%\end{array}\right]
%\]
%
%\pause
%\alert{
%Notice that $EA$ is the matrix obtained from $A$ by interchanging row 2
%and row 4,
%\pause
%which is the same row operation used to obtain $E$ from $I_4$.}
%\pause
%What about $FA$ and $GA$?
%\end{example}
%}}
%%-------------- end slide -------------------------------%
%
%%-------------- start slide -------------------------------%
%\frame{\frametitle{Multiplication by an Elementary Matrix}
%\begin{theorem}\em
%Let $A$ be an $m\times n$ matrix, and suppose that $B$ is obtained
%from $A$ by performing \alert{a single elementary} row operation.
%\pause
%Then $B=EA$ where $E$ is the elementary matrix obtained from $I_m$ 
%by performing the same elementary operation on $I_m$ as was 
%performed on $A$.
%\end{theorem}
%}
%%-------------- end slide -------------------------------%
%
%%-------------- start slide -------------------------------%
%{\small
%\frame{
%\begin{problem}\em
%Let
%\vspace*{-.3in}
%
%\[ A=\left[\begin{array}{rr}
%4 & 1 \\ 1 & 3 \end{array}\right]
%\mbox{ and }
%C=\left[\begin{array}{rr}
%1 & 3 \\ 2 & -5 \end{array}\right] \]
%Find elementary matrices $E$ and $F$ so that 
%$C=FEA$.
%\end{problem}
%\pause
%\begin{solution}\em
%{\bf Note.}
%The statement of the problem implies that \alert{$C$ can be
%obtained from $A$ by a sequence of two elementary row 
%operations}, represented by elementary matrices $E$ and $F$.
%\pause
%\[
%A = \left[\begin{array}{rr}
%4 & 1 \\ 1 & 3 
%\end{array}\right]
%\pause
%\stackrel{\rightarrow}{_E}
%\left[\begin{array}{rr}
%1 & 3 \\
%4 & 1 
%\end{array}\right]
%\pause
%\stackrel{\rightarrow}{_F}
%\left[\begin{array}{rr}
%1 & 3 \\
%2 & -5 
%\end{array}\right] = C 
%\]
%\pause
%where
%$E=\left[\begin{array}{rr}
%0 & 1 \\ 1 & 0 \end{array}\right]$
%\pause
%and $F=\left[\begin{array}{rr}
%1 & 0 \\ -2 & 1 \end{array}\right]$.
%\pause
%Thus we have the sequence
%$A\rightarrow EA \rightarrow F(EA)=C$,
%\pause
%so $C=FEA$, i.e.,
%\[ \left[\begin{array}{rr} 1 & 3 \\ 2 & -5 \end{array}\right]
%=
%\left[\begin{array}{rr}
%1 & 0 \\ -2 & 1 \end{array}\right]
%\left[\begin{array}{rr}
%0 & 1 \\ 1 & 0 \end{array}\right]
%\left[\begin{array}{rr}
%4 & 1 \\ 1 & 3 
%\end{array}\right] \]
%\pause
%\alert{You can check your work by doing the matrix multiplication.}
%\end{solution}
%}}
%%-------------- end slide -------------------------------%
%
%
%%-------------- start slide -------------------------------%
%\frame{\frametitle{Inverses of Elementary Matrices}
%\begin{example}
%Without using the matrix inversion algorithm, find the
%inverse of the elementary matrix
%\[ G=
%\left[ \begin{array}{rrrr}
%1 & 0 & 0 & 0 \\
%0 & 1 & 0 & 0 \\
%-3 & 0 & 1 & 0 \\
%0 & 0 & 0 & 1
%\end{array}\right] 
%\]
%\pause
%{\bf Hint.}
%What row operation can be applied to $G$ to transform it
%to $I_4$?
%\pause
%The row operation $G\rightarrow I_4$ is to \alert{add} three times
%row one to row three,
%\pause
%and thus
%\[ G^{-1} =
%\left[ \begin{array}{cccc}
%1 & 0 & 0 & 0 \\
%0 & 1 & 0 & 0 \\
%3 & 0 & 1 & 0 \\
%0 & 0 & 0 & 1
%\end{array}\right] \]
%
%\pause
%\alert{Check by computing $G^{-1}G$.}
%\end{example}
%}
%%-------------- end slide -------------------------------%
%
%%-------------- start slide -------------------------------%
%\frame{
%\begin{example}[continued]
%Similarly,
%\[ E^{-1} = 
%\left[ \begin{array}{cccc}
%1 & 0 & 0 & 0 \\
%0 & 0 & 0 & 1 \\
%0 & 0 & 1 & 0 \\
%0 & 1 & 0 & 0
%\end{array}\right]^{-1} =
%\left[ \begin{array}{cccc}
%1 & 0 & 0 & 0 \\
%0 & 0 & 0 & 1 \\
%0 & 0 & 1 & 0 \\
%0 & 1 & 0 & 0
%\end{array}\right]
%\]
%\uncover<2->{
%and 
%\[ F^{-1} =
%\left[ \begin{array}{rrrr}
%1 & 0 & 0 & 0 \\
%0 & 1 & 0 & 0 \\
%0 & 0 & -2 & 0 \\
%0 & 0 & 0 & 1
%\end{array}\right]^{-1}
%= \left[ \begin{array}{rrrr}
%1 & 0 & 0 & 0 \\
%0 & 1 & 0 & 0 \\
%0 & 0 & -\frac{1}{2} & 0 \\
%0 & 0 & 0 & 1
%\end{array}\right]
%\]}
%\end{example}
%}
%%-------------- end slide -------------------------------%
%
%
%%-------------- start slide -------------------------------%
%\frame{\frametitle{The Form $B=UA$}
%Suppose $A$ is an $m\times n$ matrix and that $B$ can be 
%obtained from $A$ by a sequence of $k$ elementary row operations.
%\pause
%Then there exist elementary matrices $E_1, E_2,\ldots E_k$
%such that
%\[ B=E_k(E_{k-1}(\cdots(E_2(E_1A))\cdots)) \]
%\pause
%Since matrix multiplication is associative, we have
%\[ B=(E_k E_{k-1} \cdots E_2 E_1)A\]
%\pause
%or, more concisely, $B=UA$ where
%$U=E_k E_{k-1} \cdots E_2 E_1$.
%\pause
%\bigskip
%
%To find $U$ so that $B=UA$, we \alert{could} find
%$E_1, E_2, \ldots, E_k$ and multiply these together
%(in the correct order), but there is an easier method for
%finding $U$.
%}
%%-------------- end slide -------------------------------%
%
%%-------------- start slide -------------------------------%
%\frame{
%\begin{definition}
%Let $A$ be an $m\times n$ matrix.  We write
%\[ A\rightarrow B \]
%if $B$ can be obtained from $A$ by a sequence of elementary
%row operations.
%\end{definition}
%\pause
%\begin{theorem}\em
%Suppose $A$ is an $m\times n$ matrix and that $A\rightarrow B$.
%Then
%\pause
%\begin{enumerate}
%\item
%there exists an invertible $m\times m$ matrix $U$ such that
%$B=UA$;
%\pause
%\item 
%$U$ can be computed by performing elementary row operations on
%$\left[\begin{array}{c|c} A & I_m \end{array}\right]$ 
%to transform it into 
%$\left[\begin{array}{c|c} B & U \end{array}\right]$;
%\pause
%\item
%$U=E_k E_{k-1} \cdots E_2 E_1$, where
%$E_1, E_2, \ldots, E_k$ are elementary matrices
%corresponding, in order,
%to the elementary row operations
%used to obtain $B$ from $A$.
%\end{enumerate}
%\end{theorem}
%}
%%-------------- end slide -------------------------------%
%
%%-------------- start slide -------------------------------%
%\frame{
%\begin{problem}\em
%Let $A=\left[\begin{array}{rrr}
%3 & 0 & 1 \\ 2 & -1 & 0 \end{array}\right]$, 
%and let $R$ be the reduced row-echelon form of $A$.
%
%Find a matrix $U$ so that $R=UA$.
%\end{problem}
%%%Also, find a matrix $Q$ so that $A=QR$.
%\pause
%\begin{solution}\em
%{\small
%\[ 
%\left[\begin{array}{rrr|rr}
%3 & 0 & 1 & 1 & 0 \\
%2 & -1 & 0 & 0 & 1
%\end{array}\right]
%\rightarrow
%\left[\begin{array}{rrr|rr}
%1 & 1 & 1 & 1 & -1 \\
%2 & -1 & 0 & 0 & 1
%\end{array}\right]
%\rightarrow
%\left[\begin{array}{rrr|rr}
%1 & 1 & 1 & 1 & -1 \\
%0 & -3 & -2 & -2 & 3
%\end{array}\right]
%\]
%\[ 
%\rightarrow
%\left[\begin{array}{rrr|rr}
%1 & 1 & 1 & 1 & -1 \\
%0 & 1 & \frac{2}{3} & \frac{2}{3} & -1
%\end{array}\right]
%\rightarrow
%\left[\begin{array}{rrr|rr}\vspace*{.02in}
%1 & 0 & \frac{1}{3} & \frac{1}{3} & 0 \\
%0 & 1 & \frac{2}{3} & \frac{2}{3} & -1
%\end{array}\right]
%\]}
%\pause
%Starting with 
%$\left[\begin{array}{c|c}
%A & I \end{array}\right]$, we've obtained
%$\left[\begin{array}{c|c}
%R & U \end{array}\right]$.
%
%\pause
%Therefore $R=UA$, where
%\[ U=\left[\begin{array}{rr}\vspace*{.02in}
%\frac{1}{3} & 0 \\
%\frac{2}{3} & -1
%\end{array}\right] \]
%%%\uncover<5->{
%%%Since $R=UA$ and $U$ is invertible \alert{(why?)}, we have}
%\end{solution}
%}
%%-------------- end slide -------------------------------%
%
%%-------------- start slide -------------------------------%
%%%ADD THIS SLIDE FOR QR-FACTORIZATION
%%%\frame{
%%%\begin{example}[continued]
%%%\begin{eqnarray*}
%%%R & = & UA \\
%%%U^{-1} R & = & U^{-1}(UA) \\
%%%& = & (U^{-1}U)A \\
%%%& = & IA \\
%%%& = & A.
%%%\end{eqnarray*}
%%%\uncover<2->{
%%%Thus $A=U^{-1} R$, and
%%%\[ Q=U^{-1}=
%%%\left[\begin{array}{rr}\vspace*{.02in}
%%%\frac{1}{3} & 0 \\
%%%\frac{2}{3} & -1
%%%\end{array}\right]^{-1}
%%%=\left[\begin{array}{rr}
%%%3 & 0 \\
%%%2 & -1
%%%\end{array}\right].
%%%\]}
%%%\end{example}
%%%}
%%-------------- end slide -------------------------------%
%
%%-------------- start slide -------------------------------%
%{\small
%\frame{\frametitle{A Matrix as a Product of Elementary Matrices}
%\begin{example}
%Let 
%\[ A= \left[\begin{array}{rrr}
%1 & 2 & -4 \\
%-3 & -6 & 13 \\
%0 & -1 & 2 \end{array}\right] \]
%\pause
%Suppose we do row operations to put $A$ in reduced row-echelon form,
%and write down the corresponding elementary matrices.
%\pause
%\[
%\left[\begin{array}{rrr}
%1 & 2 & -4 \\
%-3 & -6 & 13 \\
%0 & -1 & 2
%\end{array}\right]
%\pause
%\stackrel{\longrightarrow}{_{E_1}}
%\left[\begin{array}{rrr}
%1 & 2 & -4 \\
%0 & 0 & 1 \\
%0 & -1 & 2
%\end{array}\right]
%\pause
%\stackrel{\longrightarrow}{_{E_2}}
%\left[\begin{array}{rrr}
%1 & 2 & -4 \\
%0 & -1 & 2\\
%0 & 0 & 1 
%\end{array}\right]
%\pause
%\stackrel{\longrightarrow}{_{E_3}}
%\]
%\[
%\left[\begin{array}{rrr}
%1 & 2 & -4 \\
%0 & 1 & -2\\
%0 & 0 & 1 
%\end{array}\right]
%\pause
%\stackrel{\longrightarrow}{_{E_4}}
%\left[\begin{array}{rrr}
%1 & 0 & 0 \\
%0 & 1 & -2\\
%0 & 0 & 1 
%\end{array}\right]
%\pause
%\stackrel{\longrightarrow}{_{E_5}}
%\left[\begin{array}{rrr}
%1 & 0 & 0 \\
%0 & 1 & 0\\
%0 & 0 & 1 
%\end{array}\right]
%\]
%\pause
%Notice that the reduced row-echelon form of $A$ equals $I_3$.
%Now find the matrices $E_1, E_2, E_3, E_4$ and $E_5$.
%\end{example}
%}}
%%-------------- end slide -------------------------------%
%
%%-------------- start slide -------------------------------%
%\frame{
%\begin{example}[continued]
%\[ E_1 =\left[\begin{array}{rrr}
%1 & 0 & 0 \\
%3 & 1 & 0\\
%0 & 0 & 1
%\end{array}\right],
%\pause
%E_2 =\left[\begin{array}{rrr}
%1 & 0 & 0 \\
%0 & 0 & 1\\
%0 & 1 & 0\\
%\end{array}\right],
%\pause
%E_3 =\left[\begin{array}{rrr}
%1 & 0 & 0 \\
%0 & -1 & 0\\
%0 & 0 & 1
%\end{array}\right]
%\]
%\pause
%\[
%E_4 =\left[\begin{array}{rrr}
%1 & -2 & 0 \\
%0 & 1 & 0\\
%0 & 0 & 1
%\end{array}\right],
%\pause
%E_5 =\left[\begin{array}{rrr}
%1 & 0 & 0 \\
%0 & 1 & 2\\
%0 & 0 & 1
%\end{array}\right]
%\]
%\pause
%It follows that 
%\begin{eqnarray*}
%(E_5(E_4(E_3(E_2(E_1A))))) & = & I\\
%(E_5E_4E_3E_2E_1)A & = & I
%\end{eqnarray*}
%and therefore
%\[ A^{-1}=E_5E_4E_3E_2E_1 \]
%\end{example}
%}
%%-------------- end slide -------------------------------%
%
%%-------------- start slide -------------------------------%
%\frame{
%\begin{example}[continued]
%Since $A^{-1}=E_5E_4E_3E_2E_1$, 
%\begin{eqnarray*}
%A^{-1} & = & E_5E_4E_3E_2E_1 \\
%(A^{-1})^{-1} & = & (E_5E_4E_3E_2E_1)^{-1} \\
%A & = & E_1^{-1}E_2^{-1}E_3^{-1}E_4^{-1}E_5^{-1}
%\end{eqnarray*}
%\end{example}
%
%This example illustrates the following result.
%
%\uncover<2->{
%\begin{theorem}\em
%Let $A$ be an $n \times n$ matrix. Then, $A^{-1}$ exists if and only if $A$ can be written as the product of elementary matrices.
%\end{theorem}}
%}
%%-------------- end slide -------------------------------%
%
%%-------------- start slide -------------------------------%
%\frame{
%\begin{problem}\em
%Express $A=
%\left[\begin{array}{rr}
%4 & 1 \\
%-3 & 2 
%\end{array}\right]$ as a product of elementary matrices.
%\end{problem}
%\pause
%\begin{solution}\em
%\[
%\left[\begin{array}{rr}
%4 & 1 \\ -3 & 2 
%\end{array}\right]
%\pause
%\stackrel{\longrightarrow}{_{E_1}}
%\left[\begin{array}{rr}
%1 & 3 \\ -3 & 2 
%\end{array}\right]
%\pause
%\stackrel{\longrightarrow}{_{E_2}}
%\left[\begin{array}{rr}
%1 & 3 \\ 0 & 11 
%\end{array}\right]
%\pause
%\stackrel{\longrightarrow}{_{E_3}}
%\left[\begin{array}{rr}
%1 & 3 \\ 0 & 1 
%\end{array}\right]
%\pause
%\stackrel{\longrightarrow}{_{E_4}}
%\left[\begin{array}{rr}
%1 & 0 \\ 0 & 1 
%\end{array}\right]
%\]
%\pause
%and 
%\[
%E_1 =\left[\begin{array}{rr}
%1 & 1 \\ 0 & 1 
%\end{array}\right],
%\pause
%E_2 =\left[\begin{array}{rr}
%1 & 0 \\ 3 & 1 
%\end{array}\right],
%\pause
%E_3 =\left[\begin{array}{rr}
%1 & 0 \\ 0 & \frac{1}{11} 
%\end{array}\right],
%\pause
%E_4 =\left[\begin{array}{rr}
%1 & -3 \\ 0 & 1 
%\end{array}\right]
%\]
%\pause
%Since $E_4E_3E_2E_1A=I$, $A^{-1}=E_4E_3E_2E_1$, and hence
%\[ A= E_1^{-1} E_2^{-1} E_3^{-1} E_4^{-1} \]
%\end{solution}
%}
%%-------------- end slide -------------------------------%
%
%%-------------- start slide -------------------------------%
%\frame{
%\begin{solution}[continued]\em
%Therefore, 
%\[ A = 
%\left[\begin{array}{rr}
%1 & 1 \\
%0 & 1 
%\end{array}\right]^{-1}
%\left[\begin{array}{rr}
%1 & 0 \\
%3 & 1
%\end{array}\right]^{-1}
%\left[\begin{array}{rr}
%1 & 0 \\
%0 & \frac{1}{11}
%\end{array}\right]^{-1}
%\left[\begin{array}{rr}
%1 & -3 \\
%0 & 1
%\end{array}\right]^{-1}
%\]
%\pause
%i.e.,
%\[ A = 
%\left[\begin{array}{rr}
%1 & -1 \\
%0 & 1
%\end{array}\right]
%\left[\begin{array}{rr}
%1 & 0 \\
%-3 & 1
%\end{array}\right]
%\left[\begin{array}{rr}
%1 & 0 \\
%0 & 11
%\end{array}\right]
%\left[\begin{array}{rr}
%1 & 3 \\
%0 & 1
%\end{array}\right]
%\]
%\pause
%\alert{Check your work by computing the product.}
%\end{solution}
%}
%%-------------- end slide -------------------------------%
%
%%-------------- start slide -------------------------------%
%\frame{
%\begin{problem}\em
%Is the $n\times n$ identity matrix an elementary matrix? 
%Justify your answer.
%\end{problem}
%\pause
%\bigskip
%
%One result that we have assumed in all our work involving
%reduced row-echelon matrices is the following.
%\pause
%\begin{theorem}\em
%If $A$ is an $m\times n$ matrix and $R$ and $S$ are reduced
%row-echelon forms of $A$, then $R=S$.
%\end{theorem}
%\medskip
%\pause
%
%\alert{This theorem ensures that the reduced row-echelon form of
%a matrix is unique,}
%\pause
%and its proof follows from the results about elementary matrices.
%}
%%-------------- end slide -------------------------------%

\end{document}
