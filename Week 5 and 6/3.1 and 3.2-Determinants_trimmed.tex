%%%%%%%%%%%%%%%%%%%%%%
%%%%%%%%%%%%%%%%%%%%%%
%%Options for presentations (in-class) and handouts (e.g. print). 
\documentclass[pdf
%,handout
]{beamer}
%\usepackage{pgfpages}
%\pgfpagesuselayout{2 on 1}[letterpaper,border shrink=5mm]

%%%%%%%%%%%%%%%%%%%%%%
%% Change this for different slides so it appears in bar
\usepackage{authoraftertitle}
\date{Determinants: Basic Techniques and Properties}

%%%%%%%%%%%%%%%%%%%%%%
%% Upload common style file
\usepackage{../LyryxLinearAlgebraSlidesStyle}

\begin{document}
	
	%%%%%%%%%%%%%%%%%%%%%%%
	%% Title Page and Copyright Common to All Slides
	
	%Title Page
	\input ../frontmatter/titlepage.tex
	
	%LOTS Page
	%\input frontmatter/lyryxopentexts.tex
	
	%Copyright Page
	\input ../frontmatter/copyright.tex
	
	%%%%%%%%%%%%%%%%%%%%%%%%%


\section{Cofactors and $n \times n$ Determinants}

%-------------- start slide -------------------------------%
\frame{\frametitle{Determinant of a $2 \times 2$ Matrix}

\begin{definition}
Let
$A = \left[ \begin{array}{cc}
a & b \\ c & d \end{array}\right]$. Then the 
\alert{determinant} of $A$ is defined as
\[
\det A = ad-bc
\]
\end{definition}

\bigskip

\pause
{\bf Notation.}
For $\det \left[ \begin{array}{cc}
a & b \\ c & d \end{array}\right]$, we often write
$\left| \begin{array}{cc}
a & b \\ c & d \end{array}\right|$, i.e., use
\alert{vertical bars} instead of \alert{square brackets.}

\bigskip

%\pause
%What about the determinant of an $n \times n$ matrix for other values of $n$?

}
%-------------- end slide -------------------------------%

%-------------- start slide -------------------------------%
\frame{

\begin{alertblock}{How do we find the determinant of an
$n\times n$ matrix?}
The determinant of an $n\times n$ matrix is defined recursively,
using determinants of $(n-1)\times(n-1)$ submatrices,
and requires some new definitions and notation.
\end{alertblock}

\bigskip

\pause

\begin{definition}
Let $A=\left[ a_{ij}\right]$ be an $n\times n$ matrix.
The \alert{sign} of the $(i,j)$ position is $(-1)^{i+j}$.


Thus the sign is $1$ if $(i+j)$ is even, and $-1$ if 
$(i+j)$ is odd.
\end{definition}
}
%-------------------end slide------------------%

%-------------------start slide------------------%
\frame{\frametitle{The Minor of a Matrix}
\begin{definition}
Let $A = \left[a_{ij} \right]$ be an $n \times n$ matrix. 
The \alert{$ij^{th}$ minor} of $A$, denoted as
 $minor\left( A\right) _{ij},$ is the determinant
of the $n-1 \times n-1$ matrix which results from deleting the $i^{th}$ row and
the $j^{th}$ column of $A$.
\end{definition}


\[
A=
\left[
\begin{array}{rrrrrr}
a_{11} & a_{12} & \cdots & \textcolor{red}{a_{1j}} & \cdots & a_{1n} \\
a_{21} & a_{22} & \cdots & \textcolor{red}{a_{2j}} & \cdots & a_{2n} \\
\vdots & \vdots & & \vdots & & \vdots \\
\textcolor{blue}{a_{i1}} & \textcolor{blue}{a_{i2}} & \cdots & \textcolor{purple}{a_{ij}} & \cdots & \textcolor{blue}{a_{in}} \\
\vdots & \vdots & & \vdots & & \vdots \\
a_{n1} & a_{n2} & \cdots & \textcolor{red}{a_{nj}} & \cdots & a_{nn}
\end{array}
\right] 
\]



For any matrix $A$, $minor \left(A \right)_{ij}$ is found by first removing the \textcolor{blue}{$i^{th}$} \textcolor{blue}{row} and \textcolor{red}{$j^{th}$} \textcolor{red}{column}, and taking the determinant of the remaining matrix. 

}
%--------------------end slide -----------------------%

%---------------------start slide ---------------------%
\frame{\frametitle{The Minor of a Matrix}

\begin{example}
Let 
\[
A = \left[
\begin{array}{rrr}
1 & 1 & 3 \\
2 & 4 & 1 \\
5 & 2 & 6
\end{array}
\right]
\]
Find $minor(A)_{12}$.\end{example}

\pause
\begin{solution}\em
First, remove the \textcolor{blue}{$1^{st}$ row} and \textcolor{red}{$2^{nd}$ column} from $A$.
\[
A = \left[
\begin{array}{rrr}
\textcolor{blue}{1} & \textcolor{purple}{1} & \textcolor{blue}{3} \\
2 & \textcolor{red}{4} & 1 \\
5 & \textcolor{red}{2} & 6
\end{array}
\right]
\]


\pause
 The resulting matrix is $ A= \left[
\begin{array}{rr}
2 & 1 \\
5 & 6
\end{array}
\right]$
\end{solution}
}
%--------------------end slide-------------------------------%

%-------------------start slide--------------------------------%
\frame{
\begin{solution}[continued]\em
\[
A= \left[
\begin{array}{rr}
2 & 1 \\
5 & 6
\end{array}
\right]
\]

\pause
Using our previous definition, we can calculate the determinant of this matrix to be
\[
(2)(6) - (5)(1) = 12 - 5 = 7
\]


\pause
Therefore, \alert{$minor(A)_{12} = 7$.}


\end{solution}
}
%--------------------end slide ------------------------%

%---------------------start slide ---------------------%
\frame{\frametitle{The Cofactors of a Matrix}

\begin{definition}
The \alert{$ij^{th}$ cofactor} of $A$ is 
\[ \mbox{cof}(A)_{ij}=(-1)^{i+j} minor \left(A\right)_{ij} \]
\end{definition}

\pause
\begin{example}[continued]
Let 
\[
A = \left[
\begin{array}{rrr}
1 & 1 & 3 \\
2 & 4 & 1 \\
5 & 2 & 6
\end{array}
\right]
\]
Find $\mbox{cof}(A)_{12}$.

\end{example}
}
%----------------end slide----------------------%

%------------------start slide ------------------%
\frame{
\begin{solution}\em
By the definition, we know that 
$\mbox{cof}(A)_{12} = (-1)^{1+2} minor \left(A\right)_{12}$

\bigskip

\pause
From earlier, we know that $minor\left(A\right)_{12} = 7$.

\bigskip

\pause
Therefore, 
$\mbox{cof}(A)_{12} = (-1)^{1+2} minor \left(A\right)_{12}
 = (-1)^{3}7 = -7$

\end{solution}
}
%-------------- end slide -------------------------------%

%-------------- start slide -------------------------------%
\frame{\frametitle{Cofactor Expansion}
Using these definitions, we can now define
the \alert{determinant of the $n \times n$ matrix $A$}:

\begin{definition}
$\det A = a_{11} \mbox{cof}(A)_{11} + a_{12}\mbox{cof}(A)_{12} +
a_{13}\mbox{cof}(A)_{13} + \cdots + a_{1n}\mbox{cof}(A)_{1n}$

This is called the \alert{cofactor expansion of $\det A$ along 
row $1$}.
\end{definition}

\pause

We may also write: 
\begin{equation*}
\det \left( A\right) =\sum_{j=1}^{n}a_{ij}\mbox{cof}\left( A\right)
_{ij}=\sum_{i=1}^{n}a_{ij}\mbox{cof}\left( A\right) _{ij}
\end{equation*}

\pause
The first formula consists of expanding the determinant along the $i^{th}$
row and the second expands the determinant along the $j^{th}$ column.


\medskip

\pause
Cofactor expansion is also called \alert{Laplace Expansion}.
}

%-------------- end slide -------------------------------%

%-------------- start slide -------------------------------%
\frame{\frametitle{Cofactor Expansion}
\begin{example}
Let $A=
\left[\begin{array}{rrr}
1 & 1 & 3 \\
2 & 4 & 1 \\
5 & 2 & 6
\end{array}\right]$.  
Find $\det A$.
\end{example}

\pause

\begin{solution}\em
Using cofactor expansion along row 1,
\begin{eqnarray*}
\det A & = & 1\mbox{cof}_{11}(A) + 1\mbox{cof}_{12}(A) + 3\mbox{cof}_{13}(A) \\ 
\pause
& = & 1(-1)^2 
\left|\begin{array}{rr}
4 & 1 \\ 2 & 6 \end{array}\right| +
1(-1)^3\left|\begin{array}{rr}
2 & 1 \\ 5 & 6 \end{array}\right| +
3(-1)^4\left|\begin{array}{rr}
2 & 4 \\ 5 & 2 \end{array}\right| \\
\pause & = & 1(24-2) -1(12-5) +3(4-20) \\
\pause & = & 22 -7 + 3(-16) \\
\pause & = & 22 -7  - 48 \\
\pause & = & -33
\end{eqnarray*}
\end{solution}
}
%-------------- end slide -------------------------------%

%-------------- start slide -------------------------------%
\frame{
\begin{solution}[continued]\em
$A= \left[\begin{array}{rrr}
1 & 1 & 3 \\
2 & 4 & 1 \\
5 & 2 & 6
\end{array}\right]$
\bigskip

Now try cofactor expansion along column 2.
\pause
\begin{eqnarray*}
\det A & = & 1\mbox{cof}_{12}(A) + 4\mbox{cof}_{22}(A) + 2\mbox{cof}_{32}(A) \\
& = & 1(-1)^3 
\left|\begin{array}{rr}
2 & 1 \\ 5 & 6 \end{array}\right| +
4(-1)^4\left|\begin{array}{rr}
1 & 3 \\ 5 & 6 \end{array}\right| +
2(-1)^5\left|\begin{array}{rr}
1 & 3 \\ 2 & 1 \end{array}\right| \\
& = & -1(12-5) +4(6-15) -2(1-6) \\
& = & -(7) +4(-9) -2(-5) \\
& = & -7 - 36 + 10 \\
& = & -33
\end{eqnarray*}

\pause
We get the same answer!
\end{solution}
}
%-------------- end slide -------------------------------%


%-------------- start slide -------------------------------%
\frame{\frametitle{The Determinant is Well Defined}
\begin{theorem}\em
The determinant of an $n\times n$ matrix $A$ can be computed using
cofactor expansion along \alert{any row or column} of $A$.
\end{theorem}
\pause
\begin{alertblock}{What is the significance of this theorem?}
This theorem allows us to choose any row or column for 
computing cofactor expansion, which gives us the opportunity
to save ourselves some work!
\end{alertblock}
}
%-------------- end slide -------------------------------%

%-------------- start slide -------------------------------%
{\small
\frame{
\begin{problem}\em
Let 
$A= \left[\begin{array}{rrrr}
0 & 1 & -2 & 1 \\
5 & 0 & 0 & 7 \\
0 & 1 & -1 & 0 \\
3 & 0 & 0 & 2
\end{array}\right]$.  
Find $\det A$.
\end{problem}
\pause
\begin{solution}\em
Cofactor expansion along row 1 yields
\begin{eqnarray*}
\det A & = & 0\times \mbox{cof}(A)_{11} + 1\times \mbox{cof}(A)_{12} + (-2)\times \mbox{cof}(A)_{13}
+ 1\times \mbox{cof}(A)_{14} \\
\pause
& = & \mbox{cof}(A)_{12} - 2\times \mbox{cof}(A)_{13} + \mbox{cof}(A)_{14},
\end{eqnarray*}
\pause
whereas cofactor expansion along, row 3 yields
\begin{eqnarray*}
\det A & = & 0\times \mbox{cof}(A)_{31} + 1\times \mbox{cof}(A)_{32} + (-1)\times \mbox{cof}(A)_{33}
+ 0\times \mbox{cof}(A)_{34} \\
\pause
& = & 1\mbox{cof}(A)_{32} + (-1)\mbox{cof}(A)_{33}, 
\end{eqnarray*}
\pause
i.e., in the first case we must compute \alert{three} cofactors, but in
the second case we need only compute \alert{two} cofactors.
\end{solution}
}}
%-------------- end slide -------------------------------%

%-------------- start slide -------------------------------%
{\small
\frame{
\begin{solution}[continued]\em
Therefore, we can save ourselves some work by using
cofactor expansion along row 3 rather than row 1.
\[ A= \left[\begin{array}{rrrr}
0 & 1 & -2 & 1 \\
5 & 0 & 0 & 7 \\
0 & 1 & -1 & 0 \\
3 & 0 & 0 & 2
\end{array}\right]\]
\vspace*{-.25in}

\begin{eqnarray*}
\det A & = & 1\times \mbox{cof}(A)_{32} + (-1)\times \mbox{cof}(A)_{33}\\
\pause
& = & 1\times  (-1)^5
\left|\begin{array}{rrr}
0 & -2 & 1 \\
5 & 0 & 7 \\
3 & 0 & 2
\end{array}\right| +
(-1)\times (-1)^6
\left|\begin{array}{rrr}
0 & 1 & 1\\
5 & 0 & 7 \\
3 & 0 & 2
\end{array}\right| \\
\pause
& = & 
-\left|\begin{array}{rrr}
0 & -2 & 1 \\
5 & 0 & 7 \\
3 & 0 & 2
\end{array}\right| 
-\left|\begin{array}{rrr}
0 & 1 & 1\\
5 & 0 & 7 \\
3 & 0 & 2
\end{array}\right| \\
\pause
\end{eqnarray*}
\vspace*{-.2in}

Each of the two determinants above can easily be evaluated using
\alert{cofactor expansion along column 2}.
\end{solution}
}}
%-------------- end slide -------------------------------%

%-------------- start slide -------------------------------%
{\small
\frame{
\begin{solution}[continued]\em
\begin{eqnarray*}
\det A & = & -\left|\begin{array}{rrr}
0 & -2 & 1 \\ 5 & 0 & 7 \\ 3 & 0 & 2 \end{array}\right| 
-\left|\begin{array}{rrr}
0 & 1 & 1\\ 5 & 0 & 7 \\ 3 & 0 & 2 \end{array}\right| \\
\pause
& = & -(-2)(-1)^3
\left|\begin{array}{rr}
5 & 7 \\ 3 & 2
\end{array}\right| 
- 1(-1)^3
\left|\begin{array}{rr}
5 & 7 \\ 3 & 2
\end{array}\right| \\
\pause
& = & -2(10-21) + 1(10-21)\\
\pause
& = & -2(-11) + (-11)\\
\pause
& = & 22 -11\\
\pause
& = & 11.
\end{eqnarray*}
Therefore, $\det A=11$.
\end{solution}
}}
%-------------- end slide -------------------------------%

%-------------- start slide -------------------------------%
\frame{\frametitle{A Row or Column of Zeros}
\begin{example}
Let
\[ A=\left[\begin{array}{rrrr}
-8 & 1 & 0 & -4 \\
5 & 7 & 0 & -7 \\
12 & -3 & 0 & 8 \\
-3 & 11 & 0 & 2
\end{array}\right].\]
\pause

By choosing column 3 for cofactor expansion, we get
$\det A = 0$,
\pause
i.e., 
\[ \det A =0\times \mbox{cof}(A)_{13} + 0\times \mbox{cof}(A)_{23} +0\times \mbox{cof}(A)_{33}
+0\times \mbox{cof}(A)_{43} \pause = 0.\]
\end{example}
\pause
\begin{alertblock}{Important Fact}
If $A$ is an $n\times n$ matrix with a row or column of zeros,
then $\det A = 0$.
\end{alertblock}
}
%-------------- end slide -------------------------------%

\section{Triangular Matrices}

%-------------- start slide -------------------------------%
\frame{\frametitle{Determinants of a Triangular Matrices}
\begin{definitions}
\begin{enumerate}
\item
An $n\times n$ matrix $A$ is called \alert{upper triangular}
if all entries {\bf below} the main diagonal are zero.
\pause
\item
An $n\times n$ matrix $A$ is called \alert{lower triangular}
if all entries {\bf above} the main diagonal are zero.
\pause
\item
An $n\times n$ matrix $A$ is called \alert{triangular}
if it is upper triangular or lower triangular.
\end{enumerate}
\end{definitions}
\pause
\begin{theorem}\em
If $A=\left[ a_{ij} \right]$
is an $n\times n$ triangular matrix, then
\[ \det A = a_{11}\times a_{22}\times a_{33}\times \cdots\times a_{nn},\]
i.e., $\det A$ is the product of the entries of the main diagonal 
of $A$.
\end{theorem}
}
%-------------- end slide -------------------------------%

%-------------- start slide -------------------------------%
\frame{
\begin{example}
\begin{eqnarray*}
\det \left[
\begin{array}{rrr}
1 & 2 & 3 \\
0 & 5 & 6 \\
0 & 0 & 9
\end{array}
\right]
\pause
& = & 1\times\det \left[
\begin{array}{rr}
5 & 6 \\
0 & 9
\end{array}
\right]\\
\pause
& = & 1\times 5\times \det\left[
\begin{array}{r}
9
\end{array}
\right]\\
\pause
& = & 1\times 5\times 9 \\
\pause
& = & 45.
\end{eqnarray*}
\pause
Notice that $45$ is the product of the entries on the main diagonal.
\[
\left[
\begin{array}{rrr}
\textcolor{blue}{1} & 2 & 3 \\
0 & \textcolor{blue}{5} & 6 \\
0 & 0 & \textcolor{blue}{9}
\end{array}
\right]
\]
\end{example}
}
%-------------- end slide -------------------------------%

\section{Properties of the Determinant}

%-------------- start slide -------------------------------%
{\small
\frame{\frametitle{Elementary Row Operations and Determinants}
%\begin{example}
%Let 
%{\footnotesize \[ A=\left[\begin{array}{rrr}
%2 & 0 & -3 \\
%0 & 4 & 0 \\
%1 & 0 & -2
%\end{array}\right].\]}
%\pause
%
%Computing $\det A$
%by cofactor expansion along row (or column) 2 yields
%\pause
%\vspace*{-.1in}
%
%\[ \det A=4(-1)^4\left|
%\begin{array}{rr}
%2 & -3 \\ 1 & -2 \end{array}\right|
%= 4(-1)=-4.\]
%\vspace*{-.15in}
%
%\pause
%Let \textcolor{violet}{$B_1$}, 
%\textcolor{blue}{$B_2$},
%and \alert{$B_3$} be obtained from $A$ by 
%\textcolor{violet}{interchanging rows 1 and 2}, 
%\textcolor{blue}{multiplying row 3 by $-3$},
%and
%\alert{adding twice row 1 to row 3}, 
%respectively, i.e.,
%\pause
%\[
%B_1=
%\left[\begin{array}{rrr}
%2 & 0 & -3 \\
%1 & 0 & -2\\
%0 & 4 & 0 \\
%\end{array}\right],
%B_2 =
%\left[\begin{array}{rrr}
%2 & 0 & -3 \\
%0 & 4 & 0 \\
%-3 & 0 & 6
%\end{array}\right],
%B_3=
%\left[\begin{array}{rrr}
%2 & 0 & -3 \\
%0 & 4 & 0 \\
%5 & 0 & -8
%\end{array}\right].\]
%\pause
%We are interested in how elementary operations affect the determinant.
%\pause
%Computing $\det B_1$, $\det B_2$, and $\det B_3$:
%\end{example}
%}}
%%-------------- end slide -------------------------------%
%
%%-------------- start slide -------------------------------%
%\frame{
%\begin{example}[continued]
%\begin{eqnarray*}
%\det B_1 & = &  4(-1)^5\left|
%\begin{array}{rr}
%2 & -3 \\ 1 & -2 \end{array}\right|
%= (-4)(-1)=4 = (-1)\det A. \\
%\pause
%\det B_2  & = &  4(-1)^4\left|
%\begin{array}{rr}
%2 & -3 \\ -3 & 6 \end{array}\right|
%= 4(12-9)=4\times 3=12= -3\det A. \\
%\pause
%\det B_3 & = & 4(-1)^4\left|
%\begin{array}{rr}
%2 & -3 \\ 5 & -8 \end{array}\right|
%= 4(-16+15)=4(-1)=-4= \det A.
%\end{eqnarray*}
%\end{example}
%\pause
%\begin{alertblock}{}
%The general effects of elementary row operations on the determinant
%are summarized in the next theorem.
%\end{alertblock}
%}
%%-------------- end slide -------------------------------%
%
%%-------------- start slide -------------------------------%
%\frame{
\begin{theorem}\em
Let $A$ be an $n\times n$ matrix and $B$ be an $n \times n$ matrix as defined below.
\pause
\begin{enumerate}
\item Let $B$ be a matrix
which results from switching two rows of $A.$ Then $\det \left( B\right)
= - \det \left( A\right) .$ 

\item Let $B$ be a matrix
which results from multiplying some row of $A$ by a scalar $k$. Then $\det
\left( B\right) = k \det \left( A\right). $

\item
Let $B$ be a matrix
which results from adding a multiple of a row to another row.
 Then $\det \left( A\right) =\det
\left( B \right) $.

\item
If $A$ contains a row which is a multiple of another row of $A$, then $\det \left( A \right) = 0$
\end{enumerate}
\end{theorem}
\pause
\begin{alertblock}{}
An analogous theorem holds for \alert{elementary column operation}.
If $A$ is a matrix, then an 
\alert{elementary column operation}
on $A$ is simply the corresponding elementary row operation
performed on the transpose of $A$, $A^T$.
\end{alertblock}
}
%-------------- end slide -------------------------------%

%-------------- start slide -------------------------------%
{\small
\frame{\frametitle{Computing the Determinant}
\begin{example}
\begin{eqnarray*}
\det\left[\begin{array}{rrr}
-3 & 5 & -6 \\
1 & -1 & 3 \\
2 & -4 & 1 \end{array}\right]
\pause
& = &
\invisible{\left| \begin{array}{rrr}
0 & 2 & 3 \\
1 & -1 & 3 \\
2 & -4 & 1 \end{array} \right| 
\mbox{\textcolor{blue}{\scriptsize ~~~row $1+3\times$(row 2)}} \\
\pause
& = & 
\left| \begin{array}{rrr}
0 & 2 & 3 \\
1 & -1 & 3 \\
0 & -2 & -5 \end{array} \right| 
\mbox{\textcolor{blue}{\scriptsize ~~~row $3-2\times$(row 2)}} \\
\pause
& = &
(1)(-1)^{2+1}
\left| \begin{array}{rr}
2 & 3 \\ -2 & -5
\end{array} \right|
\mbox{\textcolor{blue}{\scriptsize ~~cofactor expansion: column 1}} \\
\pause
& = & -(-10+6)\\
\pause
& = & 4.}
\end{eqnarray*}
\end{example}
}}
%-------------- end slide -------------------------------%

%-------------- start slide -------------------------------%
{\small
\frame{
\begin{example}
\begin{eqnarray*}
\det\left[\begin{array}{rrrr}
3 & 1 & 2 & 4\\
-1 & -3 & 8 & 0\\
1 & -1 & 5 & 5\\
1 & 1 & 2 & -1 \end{array}\right]
\pause
& = &
\invisible{\left| \begin{array}{rrrr}
7 & 5 & 10 & 0 \\
-1 & -3 & 8 & 0\\
6 & 4 & 15 & 0\\
1 & 1 & 2 & -1 \end{array}\right]\\
\pause
& = &
(-1)(-1)^8 \left| \begin{array}{rrr}
7 & 5 & 10 \\
-1 & -3 & 8\\
6 & 4 & 15
\end{array} \right| \\
\pause
& = &
-\left| \begin{array}{rrr}
0 & -16 & 66 \\
-1 & -3 & 8\\
0 & -14 & 63
\end{array} \right| \\
\pause
& = & 
-(-1)(-1)^3 \left| \begin{array}{rr}
-16 & 66 \\
-14 & 63
\end{array} \right| \\
\pause
& = &
- \left|\begin{array}{rr}
-2 & 3 \\
-14 & 63
\end{array} \right| \\
\pause
& = &
-(-126 + 42) \\
\pause
& = &
 84.}
\end{eqnarray*}
\end{example}
}}
%-------------- end slide -------------------------------%

%-------------- start slide -------------------------------%
{\small
\frame{
\begin{problem}\em
If
$\det \left[\begin{array}{ccc}
a_1 & a_2  & a_3 \\
b_1 & b_2 & b_3 \\
c_1 & c_2 & c_3 \end{array}\right] = 4$, 
find
$\det \left[\begin{array}{ccc}
-b_1 & -b_2 & -b_3 \\
a_1+2b_1 & a_2+2b_2 & a_3+2b_3 \\
3c_1 & 3c_2 & 3c_3 \\ \end{array}\right]$. 
\end{problem}
\pause
\begin{solution}\em
{\footnotesize
\begin{eqnarray*}
\left| \begin{array}{ccc}
-b_1 & -b_2 & -b_3 \\
a_1+2b_1 & a_2+2b_2 & a_3+2b_3\\
3c_1 & 3c_2 & 3c_3 \end{array} \right| 
\pause
& = & (-1)(3) \left| \begin{array}{ccc}
b_1 & b_2 & b_3 \\
a_1+2b_1 & a_2+2b_2 & a_3+2b_3 \\
c_1 & c_2 & c_3 \end{array} \right| \\
& = & (-3)\left| \begin{array}{ccc}
b_1 & b_2 & b_3 \\
a_1 & a_2 & a_3 \\
c_1 & c_2 & c_3 \end{array} \right| \\
\pause
& = &
(-3)(-1)\left| \begin{array}{ccc}
a_1 & a_2 & a_3 \\
b_1 & b_2 & b_3 \\
c_1 & c_2 & c_3 \end{array} \right| \\
& = & (-3)(-1)\times 4\\
\pause
& = & 12.
\end{eqnarray*}}
\end{solution}
}}
%-------------- end slide -------------------------------%


%-------------- start slide -------------------------------%
{\small
\frame{{\alert{Lecture II}}
\begin{problem}\em
Let $A=
\left[\begin{array}{ccc}
2 & 3 & 5 \\
3 & 5 & 9 \\
1 & 2 & 4 \end{array}\right]$.
Find $\det A$.
\end{problem}
\pause
\begin{solution}\em
\[
\det A  =  \left|\begin{array}{ccc}
2 & 3 & 5 \\
3 & 5 & 9 \\
1 & 2 & 4 \end{array}\right|
\pause
 =  \left|\begin{array}{ccc}
3 & 5 & 9 \\
3 & 5 & 9 \\
1 & 2 & 4 \end{array}\right|
\pause
 =  \left|\begin{array}{ccc}
0 & 0 & 0 \\
3 & 5 & 9 \\
1 & 2 & 4 \end{array}\right|
 =  0.\]
\pause
\alert{Notice:
\[ \mbox{row}2 + \mbox{row}3 - 2(\mbox{row}1)
=\left[\begin{array}{ccc}
0 & 0 & 0 \end{array}\right]\]
Hence the determinant equals $0$.}
\end{solution}
}
%-------------- end slide -------------------------------%

%%-------------- start slide -------------------------------%
%\frame{
%\begin{problem}\em
%Let
%\[ A=\left[\begin{array}{ccc}
%a & b & c \\
%p & q & r \\
%x & y & z \end{array}\right]
%\mbox{ and }
%B=\left[\begin{array}{ccc}
%2a+p & 2b+q & 2c+r \\
%2p+x & 2q+y & 2r+z \\
%2x+a & 2y+b & 2z+c \end{array}\right].
%\]
%Show that $\det B= 9\det A$.
%\end{problem}
%}
%%-------------- end slide -------------------------------%
%
%%-------------- start slide -------------------------------%
%\frame{
%\begin{solution}\em
%$\det B 
%= \left|\begin{array}{ccc}
%2a+p & 2b+q & 2c+r \\
%2p+x & 2q+y & 2r+z \\
%2x+a & 2y+b & 2z+c
%\end{array}\right|
%\pause
%\xrightarrow{R_1 + (-2)R_3}
%\left|\begin{array}{ccc}
%p-4x & q-4y & r-4z \\
%2p+x & 2q+y & 2r+z \\
%2x+a & 2y+b & 2z+c
%\end{array}\right| $
%\medskip
%\pause
%
%$ 
%\xrightarrow{R_2 + (-2)R_1}
% \left|\begin{array}{ccc}
%p-4x & q-4y & r-4z \\
%9x & 9y & 9z \\
%2x+a & 2y+b & 2z+c
%\end{array}\right|
%\pause
%\xrightarrow{\frac{1}{9}R_2}
% 9\left|\begin{array}{ccc}
%p-4x & q-4y & r-4z \\
%x & y & z \\
%2x+a & 2y+b & 2z+c
%\end{array}\right|$
%\medskip
%\pause
%
%$
%\xrightarrow{R_1 + (4)R_2}
%9\left|\begin{array}{ccc}
%p & q & r \\
%x & y & z \\
%2x+a & 2y+b & 2z+c
%\end{array}\right|
%\pause
%\xrightarrow{R_3 + (-2)R_2}
% 9\left|\begin{array}{ccc}
%p & q & r \\
%x & y & z \\
%a & b & c
%\end{array}\right|$
%\medskip
%\pause
%
%$\xrightarrow{R_1 <-> R_3}
%-9\left|\begin{array}{ccc}
%a & b & c\\
%x & y & z \\
%p & q & r 
%\end{array}\right|$
%\medskip
%\pause
%$ \xrightarrow{R_2 <-> R_3}
% 9\left|\begin{array}{ccc}
%a & b & c\\
%p & q & r \\
%x & y & z \\
%\end{array}\right|
%=9\det A$.
%\end{solution}
%}
%%-------------- end slide -------------------------------%

%-------------- start slide -------------------------------%
{\small
\frame{\frametitle{Determinants and Scalar Multiplication}
\begin{problem}\em
Suppose $A$ is a $3\times 3$ matrix with $\det A=7$.
What is $\det(-2A)$?
\end{problem}
\pause
\begin{solution}\em
Write
$A=\left[\begin{array}{rrr}
a_{11} & a_{12} & a_{13} \\
a_{21} & a_{22} & a_{23} \\
a_{31} & a_{32} & a_{33}
\end{array}\right]$.
\pause
Then
$-2A=\left[\begin{array}{rrr}
-2a_{11} & -2a_{12} & -2a_{13} \\
-2a_{21} & -2a_{22} & -2a_{23} \\
-2a_{31} & -2a_{32} & -2a_{33}
\end{array}\right]$.
\pause
\[ \det(-2A) =
\left|
\begin{array}{rrr}
-2a_{11} & -2a_{12} & -2a_{13} \\
-2a_{21} & -2a_{22} & -2a_{23} \\
-2a_{31} & -2a_{32} & -2a_{33}
\end{array}
\right|
\pause
=
(-2) \left|
\begin{array}{rrr}
a_{11} & a_{12} & a_{13} \\
-2a_{21} & -2a_{22} & -2a_{23} \\
-2a_{31} & -2a_{32} & -2a_{33}
\end{array}
\right|
\]
\pause
\[
=(-2)(-2) \left|
\begin{array}{rrr}
a_{11} & a_{12} & a_{13} \\
a_{21} & a_{22} & a_{23} \\
-2a_{31} & -2a_{32} & -2a_{33}
\end{array}
\right|
\pause
=(-2)(-2)(-2) \left|
\begin{array}{rrr}
a_{11} & a_{12} & a_{13} \\
a_{21} & a_{22} & a_{23} \\
a_{31} & a_{32} & a_{33}
\end{array}
\right|
\]

\pause
\hspace*{.18in}$= (-2)^3 \det A 
\pause
= (-8)\times 7 
\pause
= -56$.
\end{solution}
}}
%-------------- end slide-----------------%

%-------------- start slide -------------------------------%
\frame{
\begin{solution}[continued]\em
Think about the matrix $-2A$ 
as the matrix obtained from $A$ be multiplying
\alert{each of its rows by $-2$}.
\pause
This involves \alert{three elementary row operations}, each
of which contributes a factor of $-2$ to the determinant,
and thus
\pause
\alert{$\det A = (-2)\times (-2)\times (-2) \times 7
\pause
=(-2)^3 \times 7$.}
\end{solution}
\pause

\begin{theorem}\em
If $A$ is an $n\times n$ matrix and $k$ is any scalar, then
\[ \det(kA)=k^n\det A.\]
\end{theorem}
}
%-------------- end slide -------------------------------%




%%-------------- start slide -------------------------------%
%{\small
%\frame{\frametitle{Determinants of Inverses}
%
%
%\pause
%\begin{example}[Illustration of the above Theorem.]
%Let
%$A=\left[\begin{array}{rr}
%2 & -3 \\ -6 & 4 \end{array}\right]$.
%\pause
%Then $\det A=2(4)-(-3)(-6)=-10$.
%\pause
%Since $\det A\neq 0$,
%the above Theorem tell us that $A$ is invertible, 
%and that $\det(A^{-1})$ 
%\alert{should be equal to $-\frac{1}{10}$}.
%\pause
%
%We can verify this directly.
%\pause 
%Using the formula for the inverse of a $2\times 2$ matrix
%\pause
%\vspace*{-.1in}
%
%\[
%A^{-1}=\frac{1}{-10}
%\left[\begin{array}{rr}
%4 & 3 \\ 6 & 2\end{array}\right].\]
%\pause
%Therefore,
%\vspace*{-.2in}
%
%\[
%\det A^{-1}=
%\left(-\frac{1}{10}\right)^2
%\left|\begin{array}{rr}
%4 & 3 \\ 6 & 2\end{array}\right|
%\pause
%= \left(-\frac{1}{10}\right)^2(8-18)
%\pause 
% = 
%\frac{1}{100}\times (-10) 
%\pause
%=-\frac{1}{10}.
%\]
%\end{example}
%}}
%%-------------- end slide -------------------------------%



%-------------- start slide -------------------------------%
\frame{\frametitle{Determinants of Inverses, Products, and Transposes}

\begin{theorem}\em
	An $n\times n$ matrix $A$ is invertible if and only if
	\alert{$\det A\neq 0$}.
	In this case,
	\vspace*{-.1in}
	
	\[ \det(A^{-1})=\frac{1}{\det A}.\]
\end{theorem}

\pause

\begin{theorem}
Let $A$ and $B$ be $n\times n$ matrices. Then 
\[ \det \left( AB\right) =\det A \times\det B.\]
\end{theorem}
\pause
\bigskip\bigskip

\begin{theorem}\em
If $A$ is an $n \times n$ matrix, then
the determinant of its transpose is given by
\[ \det(A^T) = \det A.\]
\end{theorem}
}
%-------------- end slide -------------------------------%

%-------------- start slide -------------------------------%
{\small
	\frame{
		\begin{problem}\em
			Find all values of $c$ for which
			$A=\left[\begin{array}{rrr}
			c & 1 & 0 \\
			0 & 2 & c \\
			-1 & c & 5
			\end{array}\right]$ is invertible.
		\end{problem}
		\pause
		\begin{solution}\em
			\begin{eqnarray*}
				\det A = \left|\begin{array}{rrr}
					c & 1 & 0 \\
					0 & 2 & c \\
					-1 & c & 5 \end{array}\right| 
				\pause
				& = & c  \left|\begin{array}{rr}
					2 & c \\
					c & 5 \end{array}\right| + 
				(-1) \left|\begin{array}{rr}
					1 & 0 \\
					2 & c
				\end{array}\right| \\
				\pause
				& = & c(10-c^2)-c\\
				\pause
				& = & c(9-c^2) \\
				\pause
				& = & c(3-c)(3+c).
			\end{eqnarray*}
			\pause
			Since $A$ is invertible \alert{when $\det (A) \neq 0$},
			\pause
			$A$ is invertible for all $c\neq 0, 3, -3$.
		\end{solution}
}}
%-------------- end slide -------------------------------%
%-------------- start slide -------------------------------%
\frame{
\begin{problem}\em
Suppose $A$, $B$ and $C$ are $4\times 4$ matrices with
\[ \det A = -1, \det B = 2, \mbox{ and } \det C=1.\]
Find $\det(2A^2(B^{-1})(C^T)^3 B(A^{-1}))$.
\end{problem}
\pause
\begin{solution}
\begin{eqnarray*}
\det(2A^2(B^{-1})(C^T)^3 B(A^{-1}))
\pause
& = & 2^4 (\det A)^2 \frac{1}{\det B}(\det C)^3 
(\det B)\frac{1}{\det A}\\
\pause
& = & 16(\det A)(\det C)^3 \\
\pause
& = & 16\times (-1)\times 1^3\\
\pause
& = & -16.
\end{eqnarray*}
\end{solution}
}
%-------------- end slide -------------------------------%

%%-------------- start slide -------------------------------%
%{\small
%\frame{
%\begin{problem}\em
%Suppose $A$ is a $3\times 3$ matrix.
%Find $\det A$ and $\det B$ if
%\vspace*{-.15in}
%
%\[ \det(2A^{-1})=-4=\det(A^3(B^{-1})^T).\]
%\end{problem}
%\pause
%\begin{solution}\em
%First,
%\vspace*{-.45in}
%
%\begin{eqnarray*}
%\det(2A^{-1}) & = & -4 \\
%\pause
%2^3\det(A^{-1}) & = & -4 \\
%\pause
%\frac{1}{\det A} & = & \frac{-4}{8} = -\frac{1}{2}.
%\end{eqnarray*}
%\pause
%Therefore, $\det A=-2$.
%\pause
%Using this fact,
%\vspace*{-.25in}
%
%\begin{eqnarray*}
%\det(A^{3}(B^{-1})^T) & = & -4 \\
%\pause
%(\det A)^3\det(B^{-1}) & = & -4 \\
%\pause
%(-2)^3\det(B^{-1}) & = & -4 \\
%\pause
%(-8)\det(B^{-1}) & = & -4 \\
%\pause
%\frac{1}{\det B} & = & \frac{-4}{-8} = \frac{1}{2}
%\end{eqnarray*}
%\pause
%\vspace*{-.25in}
%
%Therefore, $\det B=2$.
%\end{solution}
%}}
%%-------------- end slide -------------------------------%

%-------------- start slide -------------------------------%
\frame{
\begin{problem}\em
Let $A$ be an $n\times n$ matrix.  
Find all  conditions that ensure
$\det(-A)=\det A$.
\end{problem}
\pause
\begin{solution}\em
Since $\det(-A)=(-1)^n\det A$, 
\pause
$\det(-A)=\det A$ if and only if 
\[ (-1)^n\det A = \det A.\]
\pause
\alert{When is this possible?}
\pause
\begin{enumerate}
\item $(-1)^n\det A = \det A$ whenever $\det A=0$.
\pause
\item If $\det A\neq 0$, then $(-1)^n\det A = \det A$ 
only if $(-1)^n=1$, 
\pause
i.e., only if $n$ is even.
\end{enumerate}
\pause
Therefore, $\det(-A)=\det A$ only if 
\alert{$\det A=0$ or $n$ is even}.
\end{solution}
}
%-------------- end slide -------------------------------%

\section{Using Row Operations}


%%--------------------start slide-------------------------%
%\frame{\frametitle{Using Row Operations}
%\begin{problem}\em
%Let
%\[
%A = \left[
%\begin{array}{rrr}
%5 & 1 & 2 \\
%1 & 3 & 2 \\
%2 & 6 & 0 
%\end{array}
%\right]
%\]
%
%Find $\det(A)$.
%\end{problem}
%
%\uncover<2->{
%\begin{solution}\em
%We could solve this using cofactor expansion. However, we can also use row operations to simplify $A$ first.}
%
%\uncover<3->{
%First, switch rows $1$ and $2$ to obtain matrix $B$.
%\[
%B = \left[
%\begin{array}{rrr}
%1 & 3 & 2 \\
%5 & 1 & 2 \\
%2 & 6 & 0 
%\end{array}
%\right]
%\]
%
%Then, $\det(B) = - \det(A)$, which we can write as $\det(A) = -\det(B)$.
%}
%\end{solution}
%}
%%-------------------------end slide------------------------%
%
%%-----------------------start slide------------------------%
%\frame{
%\begin{solution}[continued]\em
%Now, multiply row $3$ by $\frac{1}{2}$ to obtain matrix $C$.
%\[
%C =  \left[
%\begin{array}{rrr}
%1 & 3 & 2 \\
%5 & 1 & 2 \\
%1 & 3 & 0 
%\end{array}
%\right]
%\]
%
%Then, $\det(C) = \frac{1}{2} \det(B) = -\frac{1}{2}\det(A)$. 
%
%\uncover<2->{
%Now, subtract $5$ times row $1$ from row $2$, and $1$ times row $1$ from row $3$ to obtain matrix $D$.
%\[
%D =  \left[
%\begin{array}{rrr}
%1 & 3 & 2 \\
%0 & -14 & -8 \\
%0 & 0 & -2 
%\end{array}
%\right]
%\]
%
%Then, $\det(D) = \det(C) = -\frac{1}{2}\det(A)$. Hence, $\det(A) = -2\det(D)$.
%}
%\end{solution}
%}
%%-----------------end slide----------------------%
%
%%-------------------start slide-----------------%
%\frame{
%\begin{solution}[continued]\em
%Now we can use cofactor expansion to find $\det(D)$. 
%\[
%\det(D) = (1)(-1)^{1+1} \left|
%\begin{array}{rr}
%-14 & -8 \\
%0 & -2
%\end{array}
%\right|
%=
%28
%\]
%
%Similarly, since $D$ is triangular, we can find the determinant by multiplying the entries on the main diagonal. 
%
%\uncover<2->{
%Then
%\[
%\det(A) = -2\det(D) = -2 (28) = -56
%\]
%}
%\end{solution}
%}
%%------------------end slide------------------%

%%%%%%%%%%%%%%%%%%%%%%%%%

\section{A Formula for the Inverse}

%-------------- start slide -------------------------------%
\frame{\frametitle{The Cofactor Matrix}
	\begin{definition}
		Let $A=\left[ a_{ij}\right]$ be an $n\times n$ matrix.
		The \alert{cofactor matrix of $A$}, 
		is the matrix 
		\[ \left[ \mbox{cof}(A)_{ij} \right],\]
		i.e., the matrix whose $(i,j)$-entry is the $(i,j)$-cofactor
		of $A$.
	\end{definition}
	\pause
	\begin{alertblock}{Reminder: the $(i,j)$-cofactor}
		\[ \mbox{cof}(A)_{ij} = (-1)^{i+j} minor\left(A\right)_{ij},\]
		where $minor\left(A\right)_{ij}$ is the determinant of the matrix obtained from $A$ by
		deleting row $i$ and column $j$.
	\end{alertblock}
}
%-----------------------end slide---------------------%

%-------------------start slide----------------------%
{\small
	\frame{
		\begin{problem}\em
			Find the cofactor matrix $\left[ \mbox{cof}(A)_{ij}\right]$ of the matrix
			\[
			A = \left[ \begin{array}{rrr}
			4 & 0 & 3 \\
			1 & 9 & 7 \\
			0 & 6 & 4 \end{array} \right]. \]
		\end{problem}
		\pause
		\begin{solution}\em
			For each $i$ and $j$, $1\leq i,j\leq 3$, we need to compute
			cof$(A)_{ij}$, so there are \alert{9 cofactors} to compute.
			\pause
			\begin{eqnarray*}
				\mbox{cof}(A)_{11} & = & (-1)^{1+1} \det A_{11} 
				\pause
				= \left|\begin{array}{rr} 9 & 7 \\ 6 & 4 \end{array}\right|
				\pause
				= 9\times 4 - 6\times 7 
				\pause
				= 36-42 
				\pause
				= -6. \\
				\pause
				\mbox{cof}(A)_{12} & = & (-1)^{1+2} \det A_{12} 
				\pause
				= -\left|\begin{array}{rr} 1 & 7 \\ 0 & 4 \end{array}\right|
				\pause
				= -(4-0)
				\pause
				= -4. \\ 
				\pause
				\mbox{cof}(A)_{13} \pause & = & (-1)^{1+3} \det A_{12} 
				= \left|\begin{array}{rr} 1 & 9 \\ 0 & 6 \end{array}\right|
				= (6-0)
				= 6. 
			\end{eqnarray*}
		\end{solution}
}}
%--------------------end slide---------------------------%

%-----------------------start slide------------------------%
\frame{
	\begin{solution}[continued]\em
		Computing the six remaining cofactors results in the 
		cofactor matrix
		\pause
		\[
		\left[ \begin{array}{rrr}
		-6 & -4 & 6 \\
		18 & 16 & -24 \\
		-27 & -25 & 36 \end{array} \right].  \]
	\end{solution}
}
%-----------------------end slide--------------------%


%------------------------start slide-------------------%
{\small
	\frame{\frametitle{The Adjugate}
		\begin{definition}
			If $A$ is an $n\times n$ matrix, then the \alert{adjugate of $A$} is defined by
			\alert{
				\[ \adj~A =  \left[\begin{array}{c} \mbox{cof}(A)_{ij}\end{array}\right]^T,
				\]}
			\vspace*{-.15in}
			
			where $\mbox{cof}(A)_{ij}$ is the $(i,j)$-cofactor of $A$,
			i.e., \textcolor{blue}{$adj~A$ is the 
				transpose of the cofactor matrix}.
		\end{definition}
		\pause
		
		\begin{example}
			$A = 
			\left[ \begin{array}{rrr}
			4 & 0 & 3 \\
			1 & 9 & 7 \\
			0 & 6 & 4 \end{array} \right]$, has cofactor matrix 
			$\left[ \begin{array}{rrr}
			-6 & -4 & 6 \\
			18 & 16 & -24 \\
			-27 & -25 & 36 \end{array} \right]$.
			\medskip
			\pause
			
			Therefore, the adjugate of $A$ is
			\[ \adj~A = \left[ \begin{array}{rrr}
			-6 & -4 & 6 \\
			18 & 16 & -24 \\
			-27 & -25 & 36 \end{array} \right]^T
			=
			\left[ \begin{array}{rrr}
			-6 & 18 & -27 \\
			-4 & 16 & -25 \\
			6 & -24 & 36 \end{array} \right]. \]
		\end{example}
}}
%-------------- end slide -------------------------------%

%-------------------start slide -------------------%
\frame{
	\begin{problem}\em
		Find $\adj~A$ when
		$A=\left[\begin{array}{rrr}
		2 & 1 & 3 \\
		5 & -7 & 1 \\
		3 & 0 & -6
		\end{array}\right]$.
	\end{problem}
	\pause
	\begin{solution}\em
		\[
		\adj~A =
		\left[\begin{array}{rrr}
		42 & 6 & 22 \\
		33 & -21 & 13 \\
		21 & 3 & -19
		\end{array}\right].
		\]
	\end{solution}
}
%-------------- end slide -------------------------------%

%-------------------start slide -------------------%
{\small
	\frame{\frametitle{The Adjugate of a $2 \times 2$ Matrix}
		\begin{example}
			Let $A$ be a $2\times 2$ matrix, say
			$A=\left[\begin{array}{cc}
			a & b \\ c & d \end{array}\right]$.
			\pause
			Then
			\begin{eqnarray*}
				\adj~A
				\pause
				=\left[\begin{array}{cc}
					\mbox{cof}(A)_{11} & \mbox{cof}(A)_{12} \\
					\mbox{cof}(A)_{21} & \mbox{cof}(A)_{22} \end{array}\right]^T 
				\pause
				&=&
				\left[\begin{array}{cc}
					(-1)^2  d & (-1)^3 c  \\
					(-1)^3  b & (-1)^4 a \end{array}\right]^T \\
				\pause
				&=&
				\left[\begin{array}{cc}
					d & -b \\
					-c & a \end{array}\right].
			\end{eqnarray*}
			\pause
			\alert{We've seen this matrix before:} 
			if $\det A\neq 0$, then 
			\[ A^{-1}=\frac{1}{\det A}
			\left[\begin{array}{cc}
			d & -b \\
			-c & a \end{array}\right]
			\pause 
			=\frac{1}{\det A} \adj~A.\]
		\end{example}
}}
%-------------- end slide -------------------------------%

%-------------- start slide -------------------------------%
\frame{
	\begin{example}[continued]
		Observe that, regardless of the value of $\det A$,
		\begin{eqnarray*}
			A(\adj~A) & = & 
			\left[\begin{array}{cc}
				a & b \\ c & d \end{array}\right]
			\left[\begin{array}{rr}
				d & -b \\ -c & a \end{array}\right] \\
			\pause
			& = & 
			\left[\begin{array}{cc}
				ad-bc & 0 \\ 0 & ad-bc \end{array}\right] \\
			\pause
			& = & 
			(ad-bc)\left[\begin{array}{cc}
				1 & 0 \\ 0 & 1 \end{array}\right] \\
			\pause
			& = & (\det A)I_2.
		\end{eqnarray*}
	\end{example}
}
%-------------- end slide -------------------------------%


%-------------- start slide -------------------------------%
{\small
	\frame{
		\begin{example}
			In an earlier example, we saw that 
			$A = \left[ \begin{array}{rrr}
			4 & 0 & 3 \\
			1 & 9 & 7 \\
			0 & 6 & 4
			\end{array} \right]$,
			\pause
			has adjugate 
			
			$\adj~A = \left[ \begin{array}{rrr}
			-6 & 18 & -27 \\
			-4 & 16 & -25 \\
			6 & -24 & 36 
			\end{array} \right]$.
			\pause
			Computing $A(\adj~A)$ we see that
			\[ A(\adj~A)
			= 
			\left[\begin{array}{rrr}
			4 & 0 & 3 \\
			1 & 9 & 7 \\
			0 & 6 & 4 \end{array}\right]
			\left[ \begin{array}{rrr}
			-6 & 18 & -27 \\
			-4 & 16 & -25 \\
			6 & -24 & 36 
			\end{array} \right]
			\pause
			=
			\left[\begin{array}{rrr}
			-6 & 0 & 0 \\
			0 & -6 & 0 \\
			0 & 0 & -6
			\end{array}\right].\]
			\pause
			Note that
			\[\det A
			=
			\left|\begin{array}{rrr}
			4 & 0 & 3 \\
			1 & 9 & 7 \\
			0 & 6 & 4 \end{array}\right|
			\pause
			=
			\left|\begin{array}{rrr}
			0 & -36 & -25 \\
			1 & 9 & 7 \\
			0 & 6 & 4
			\end{array}\right| 
			\pause
			=
			-\left|\begin{array}{rrr}
			-36 & -25 \\
			6 & 4
			\end{array}\right| 
			\pause
			= -6.\]
			\alert{Therefore we have $A(\adj~A) = (\det A)I$.}
		\end{example}
}}
%-------------- end slide -------------------------------%

%-------------- start slide -------------------------------%
%-------------- end slide -------------------------------%


%-------------- start slide -------------------------------%
\frame{\frametitle{The Adjugate Formula}
	\begin{theorem}\em
		If $A$ is an $n\times n$ matrix, then
		\[ A(\adj~A)=(\det A) I = (\adj A)A.\]
		Furthermore, if $\det A\neq 0$, then we get a formula
		for $A^{-1}$, i.e.,
		\pause
		\[
		A^{-1}= \frac{1}{\det A} \adj~A.\]
	\end{theorem}
	\pause
	\begin{alertblock}{Inverting a matrix using the adjugate}
		Except in the case of a $2\times 2$ matrix, the adjugate formula
		is a very inefficient method for computing the inverse of a matrix;
		the matrix inversion algorithm is much more practical.
		However, the adjugate formula is of theoretical significance.
	\end{alertblock}
}
%-------------- end slide -------------------------------%

%---------------------start slide----------------------%
{\small
	\frame{\frametitle{Proof of the Adjugate Formula}
		\begin{example}
			
			Recall that the $(i,j)$-entry of $\adj(A)$ is equal to $\mbox{cof}(A)_{ji}$. 
			\pause
			Let us compute the $(i,j)$-entry of 
			$B=A\cdot \adj(A)$: 
			\[
			b_{ij}=\sum_{k=1}^n a_{ik} \mbox{cof}(A)_{ki}
			\]\pause
			By the cofactor expansion theorem, $b_{ij}$ is equal to the determinant of
			matrix $C$ obtained from $A$ by replacing its
			$j$th  column by $a_{i1}, a_{i2}, \dots a_{in}$ --- i.e., its $i$th column. 
			\pause
			
			If $i=j$ then this matrix is $A$ and therefore 
			\[
			a_{ii}=\det A
			\]
			for all $i$. 
			\pause
			If $i\neq j$ then this matrix has its $i$th column equal to its $j$th column, and therefore 
			\[
			a_{ij}=0\qquad \text { if } i\neq j. 
			\]
		\end{example}
	}
	%-------------- end slide -------------------------------%
	
	
	%---------------------start slide----------------------%
	{\small
		\frame{\frametitle{Using the Adjugate to Find the Inverse of a Matrix}
			\begin{example}
				Let 
				$A = \left[ \begin{array}{rrr}
				4 & 0 & 3 \\
				1 & 9 & 7 \\
				0 & 6 & 4 \end{array} \right]$.
				\pause
				As we saw earlier, $\det A=-6\neq 0$, so $A$ is invertible, and
				\pause
				\[ \adj~A=\left[ \begin{array}{rrr}
				-6 & 18 & -27 \\
				-4 & 16 & -25 \\
				6 & -24 & 36 \end{array} \right]. \]
				\pause
				\[
				A^{-1} =\frac{1}{\det A}\adj~A
				\pause =\frac{1}{-6}
				\left[ \begin{array}{rrr}
				-6 & 18 & -27 \\
				-4 & 16 & -25 \\
				6 & -24 & 36 \end{array} \right]
				\pause
				=
				\left[ \begin{array}{rrr}
				1 & -3 & \vspace{0.05in}\frac{9}{2} \\
				\vspace{0.05in}\frac{2}{3} & -\vspace{0.05in}\frac{8}{3}
				& \vspace{0.05in}\frac{25}{6} \\
				-1 & 4 & -6
				\end{array} \right].  \]
				\pause
				\alert{You can check this by computing $AA^{-1}$.}
				\pause
				You could also check by using the Matrix Inversion Algorithm
				to find $A^{-1}$ (though this is more work).
			\end{example}
	}}
	%-------------- end slide -------------------------------%
	
	%-------------- start slide -------------------------------%
	{\small
		\frame{
			\begin{problem}\em
				Let $A$ be an $n\times n$ invertible matrix. 
				Show that $\det(\adj~A)=(\det A)^{n-1}$.
			\end{problem}
			\pause
			\begin{solution}\em
				Using the adjugate formula,
				\begin{eqnarray*}
					A(\adj~A) & = & (\det A)I \\
					\pause
					\det(A(\adj~A)) & = & \det((\det A)I) \\
					\pause
					(\det A) \times \det(\adj~A) & = & (\det A)^{n}(\det I) \\
					\pause
					(\det A) \times \det(\adj~A) & = & (\det A)^{n}
				\end{eqnarray*}
				\pause
				Since $A$ is invertible, $\det A\neq 0$,  so we
				divide both sides of the last equation
				by $\det A$ to obtain
				\pause
				\[ \det(\adj~A)=(\det A)^{n-1}.\]
			\end{solution}
			\pause
			\begin{alertblock}{}
				Even if $A$ is not invertible, $\det(\adj~A)=(\det A)^{n-1}$, but
				the proof is more complicated.
			\end{alertblock}
	}}
	%------------------------end slide-------------------------% 
	
	\section{Cramer's Rule}
	%-------------- start slide -------------------------------%
	\frame{\frametitle{Cramer's Rule}
		\begin{alertblock}{}
			If $A$ is an $n\times n$ \alert{invertible} matrix, then
			the solution to $AX=B$ can be given in terms of determinants
			of matrices.
		\end{alertblock}
		\pause
		\begin{theorem}[Cramer's Rule]\em
			Let $A$ be an $n\times n$ invertible matrix, and consider
			the system $AX=B$, where
			$X=\left[\begin{array}{cccc}
			x_1 & x_2 & \cdots & x_n
			\end{array}\right]^T$.
			\pause
			We define $A_i$ to be the matrix obtained from $A$ by
			replacing column $i$ with $B$.
			\pause
			Then for each value of $i$, $1\leq i\leq n$,
			\[ x_i = \frac{\det A_i}{\det A}\]
		\end{theorem}
	}
	%-------------- end slide -------------------------------%
	
	%-------------- start slide -------------------------------%
	\frame{
		\begin{example}[Cramer's Rule]
			Solve the following system of linear equations using Cramer's Rule.
			\[ \begin{array}{rrrrrrr}
			3x_1 & + & x_2 & - & x_3 & = & -1 \\
			5x_1 & + & 2x_2 & & & = & 2 \\
			x_1 & + & x_2 & - & x_3 & = & 1 
			\end{array}\]
			\pause
			First, $x_1 = \frac{\det A_1}{\det A}$, 
			\pause
			where
			\[ A=\left[\begin{array}{rrr}
			3 & 1 & -1 \\ 
			5 & 2 & 0 \\
			1 & 1 & -1
			\end{array}\right]
			\pause
			\mbox{ and }
			A_1=\left[\begin{array}{rrr}
			-1 & 1 & -1 \\ 
			2 & 2 & 0 \\
			1 & 1 & -1
			\end{array}\right]. \]
			\pause
			Computing the determinants of these two matrices,
			\[ \det A = -4 \mbox{ and } \det A_1 = 4, \]
			\pause
			and thus $x_1 = \frac{4}{-4} = -1$.
		\end{example}
	}
	%---------------------------end slide--------------------%
	
	%---------------------start slide------------------------%
	{\small
		\frame{
			\begin{example}[continued]
				Secondly,
				$x_2 = \frac{\det A_2}{\det A}$ where $\det A = -4$ and 
				\[\det A_2=\left|\begin{array}{rrr}
				3 & -1 & -1 \\ 
				5 & 2 & 0 \\
				1 & 1 & -1
				\end{array}\right| = -14,\]
				\pause
				and thus $x_2 = \frac{-14}{4} = \frac{7}{2}.$
				\pause
				Finally, 
				$x_3=\frac{\det A_3}{\det A}$, where $\det A=-4$ and
				\[ \det A_3=\left|\begin{array}{rrr}
				3 & 1 & -1 \\ 
				5 & 2 & 2 \\
				1 & 1 & 1
				\end{array}\right|= -6,\]
				\pause
				and thus $x_3=\frac{-6}{-4}=\frac{3}{2}$.
				\pause
				Therefore, the solution to the system is given by
				\vspace*{-.10in}
				
				\[ X = \left[ \begin{array}{r}
				x_1 \\ x_2 \\ x_3 \\
				\end{array} \right]
				=
				\left[ \begin{array}{r}
				-1 \\ \vspace{0.05in}\frac{7}{2} \\ \vspace{0.05in}\frac{3}{2}
				\end{array} \right].  \]
				\vspace*{-.10in}
				
				\pause
				\alert{You can check this by substituting these values into
					the original system.}
			\end{example}
		}
		%-------------- end slide -------------------------------%
		
		\section{Polynomial Interpolation}
		%-------------- start slide -------------------------------%
		\frame{\frametitle{Polynomial Interpolation}
			\begin{problem}\em
				Given data points $(0,1)$, $(1,2)$, $(2,5)$ and $(3,10)$,
				find an interpolating polynomial 
				$p(x)$ of degree at most three, 
				and then estimate the value of $y$ corresponding to $x=\frac{3}{2}$.
			\end{problem}
			\pause
			\begin{solution}\em
				We want to find the coefficients $r_0$, $r_1$, $r_2$ and $r_3$ of
				\[ p(x) = r_0 + r_1 x + r_2 x^2 + r_3 x^3\]
				\pause
				so that $p(0)=1$, $p(1)=2$, $p(2)=5$, and $p(3)=10$.
				\pause
				\begin{eqnarray*}
					p(0) & = & r_0 = 1\\
					p(1) & = & r_0 + r_1 + r_2 + r_3 = 2 \\
					p(2) & = & r_0 + 2r_1 + 4r_2 + 8r_3 = 5 \\
					p(3) & = & r_0 + 3r_1 + 9r_2 + 27r_3 = 10 
				\end{eqnarray*}
			\end{solution}
		}
		%-------------- end slide -------------------------------%
		
		%-------------- start slide -------------------------------%
		\frame{
			\begin{solution}[continued]\em
				Solve this system of four equations in the four variables
				$r_0$, $r_1$, $r_2$ and $r_3$.
				\pause
				\[ \left[\begin{array}{rrrr|r}
				1 & 0 & 0 & 0 & 1 \\
				1 & 1 & 1 & 1 & 2 \\
				1 & 2 & 4 & 8 & 5 \\
				1 & 3 & 9 & 27 & 10
				\end{array}\right]
				\pause
				\rightarrow \cdots \rightarrow
				\left[\begin{array}{rrrr|r}
				1 & 0 & 0 & 0 & 1 \\
				0 & 1 & 0 & 0 & 0 \\
				0 & 0 & 1 & 0 & 1 \\
				0 & 0 & 0 & 1 & 0
				\end{array}\right] \]
				\pause
				Therefore $r_0=1$, $r_1=0$, $r_2=1$, $r_3=0$, and so
				\[ p(x)= 1 + x^2.\]
				\pause
				The estimate for the value of $y$ corresponding to $x=\frac{3}{2}$ is
				\pause
				\[ y=p\left(\frac{3}{2}\right) =
				1 + \left(\frac{3}{2}\right)^2=\frac{13}{4},\]
				\pause
				resulting in the point $(\frac{3}{2},\frac{13}{4})$.
			\end{solution}
		}
		%-------------- end slide -------------------------------%
		
		%-------------- start slide -------------------------------%
		{\small
			\frame{
				\begin{theorem}\em
					Given $n$ data points $(x_1,y_1), (x_2,y_2),\ldots ,
					(x_n,y_n)$ with the $x_i$ \alert{distinct}, 
					\pause there is a 
					unique polynomial 
					$p(x)= r_0 + r_1x + r_2x^2 +\cdots + r_{n-1}x^{n-1}$
					such that $p(x_i)=y_i$ for $i=1,2,\ldots,n$.
					\pause
					The polynomial $p(x)$ is called the \alert{interpolating polynomial}
					for the data points.
				\end{theorem}
				\pause
				\begin{alertblock}{}
					To find $p(x)$, set up a system of $n$ linear equations in the $n$
					variables $r_0, r_1, r_2, \ldots, r_{n-1}$.
					\pause
					\vspace*{-.2in}
					
					{\footnotesize
						\[ \begin{array}{ccc}
						r_0 + r_1x_1 + r_2x_1^2 + \cdots + r_{n-1}x_1^{n-1} & = & y_1 \\
						r_0 + r_1x_2 + r_2x_2^2 + \cdots + r_{n-1}x_2^{n-1} & = & y_2 \\
						\vdots & \vdots & \vdots \\
						r_0 + r_1x_n + r_2x_n^2 + \cdots + r_{n-1}x_n^{n-1} & = & y_n 
						\end{array} \]}
					\pause
					\vspace*{-.05in}
					
					The fact that the \alert{$x_i$ are distinct} guarantees that
					the coefficient matrix 
					\vspace*{-.1in}
					
					{\footnotesize\[ \left[\begin{array}{ccccc}
						1 & x_1 & x_1^2 & \cdots & x_1^{n-1}\\
						1 & x_2 & x_2^2 & \cdots & x_2^{n-1}\\
						\vdots & \vdots & \vdots & \vdots & \vdots \\
						1 & x_n & x_n^2 & \cdots & x_n^{n-1}
						\end{array}\right]\]}
					\vspace*{-.1in}
					
					has determinant 
					\alert{not equal to zero}, 
					\pause and so the system has a unique
					solution, i.e., there is a unique interpolating polynomial
					for the data points.
					%\pause
					%The determinant of a matrix of this form is called a
					%\alert{Vandermonde} determinant.
				\end{alertblock}
		}}
		%-------------- end slide -------------------------------%

\end{document}



