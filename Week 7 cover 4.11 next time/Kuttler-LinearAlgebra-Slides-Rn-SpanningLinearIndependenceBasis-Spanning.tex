
%%Options for presentations (in-class) and handouts (e.g. print). 
\documentclass[pdf]{beamer}
%\documentclass[pdf,handout]{beamer}

%%%%%%%%%%%%%%%%%%%%%%
%Change this for different slides so it appears in bar
\usepackage{authoraftertitle}
\date{$\mathbb{R}^n$: Spanning Sets of Vectors}

%%%%%%%%%%%%%%%%%%%%%%
%% Upload common style file
\usepackage{LyryxLinearAlgebraSlidesStyle}

\begin{document}

%%%%%%%%%%%%%%%%%%%%%%%
%% Title Page and Copyright Common to All Slides

%Title Page
\input frontmatter/titlepage.tex

%LOTS Page
\input frontmatter/lyryxopentexts.tex

%Copyright Page
\input frontmatter/copyright.tex

%%%%%%%%%%%%%%%%%%%%%%%%%

{\small

\section{Preliminary Notation}

%-------------- start slide -------------------------------%
\frame{\frametitle{Notation}
\begin{definition}
$\mathbb{R}$ denotes the set of \alert{real} numbers. $\mathbb{R}^n$ is the set of all \alert{$n$-tuples} of real numbers, i.e.,
\[ \mathbb{R}^n=\left\{ (x_1, x_2, \ldots, x_n) ~|~ x_i\in\mathbb{R}, 1\leq i\leq n
\right\}.\]
\end{definition}

\pause

Vectors are denoted as follows:
$\vect{u}, \vect{v}, \vect{x}$, etc.
\begin{example}
$\vect{u}=\leftB \begin{array}{r} -2 \\3\\ 0.7 \\ 5\\ \pi \end{array}\rightB$
is a vector in $\mathbb{R}^5$, written $\vect{u}\in\mathbb{R}^5$.

\pause

To save space on the page, the same vector $\vect{u}$ 
may be written instead as a row matrix by taking the transpose of
the column:
\[ \vect{u}=\leftB\begin{array}{ccccc}
-2, & 3, & 0.7, & 5, & \pi \end{array}\rightB^T.\]
\end{example}

}
%-------------- end slide -------------------------------%

\section{Spanning}
%-------------- start slide -------------------------------%
\frame{

\begin{definition}[Recall: Linear Combination]
Let $\vect{u}_1,\cdots ,\vect{u}_n, \vect{v}$ be vectors. Then 
$\vect{v}$ is said to be a \textbf{linear combination }
of the vectors $\vect{u}_1,\cdots , \vect{u}_n $ 
if there exist scalars, $a_{1},\cdots ,a_{n}$ such
that
\begin{equation*}
\vect{v} = a_1 \vect{u}_1 + \cdots + a_n \vect{u}_n
\end{equation*}
\end{definition}

\medskip

\pause

\begin{definition}[Span of a Set of Vectors]
The collection of all linear combinations of a set of vectors $\{ \vect{u}_1,
\cdots ,\vect{u}_k\}$ in $\mathbb{R}^{n}$ is known as the span of these
vectors and is written as $\func{span} \{\vect{u}_1, \cdots , \vect{u}_k\}$.
\end{definition}

\pause
\medskip

{\bf Additional Terminology.}
If $U=\Span \{\vect{u}_1, \vect{u}_2, \ldots, \vect{u}_k\}$, then 
\begin{itemize}
\item \textcolor{red}{$U$ is spanned by} the vectors 
$\vect{u}_1, \vect{u}_2, \ldots, \vect{u}_k$.
\item the vectors $\vect{u}_1, \vect{u}_2, \ldots, \vect{u}_k$ 
\textcolor{red}{span $U$}.
\item the set of vectors $\{ \vect{u}_1, \vect{u}_2, \ldots, \vect{u}_k\}$ is
a \textcolor{red}{spanning set} for $U$.
\end{itemize}
}
%-------------- end slide -------------------------------%

%-------------- start slide -------------------------------%
\frame{
\begin{example}
Let $\vect{x}\in\mathbb{R}^3$ be a nonzero vector. 
Then $\Span\{\vect{x}\} = \{ k\vect{x} ~|~ k\in\mathbb{R}\}$
is a line through the origin having direction vector $\vect{x}$.
\end{example}
\pause

\begin{example}
Let $\vect{x},\vect{y}\in\mathbb{R}^3$ be nonzero vectors that are not
parallel.  
Then 
\[ \Span\{\vect{x},\vect{y}\} = \{ k\vect{x} + t\vect{y} ~|~ k,t\in\mathbb{R}\} \]
is a plane through the origin containing $\vect{x}$ and $\vect{y}$.
\pause
\medskip

\textcolor{blue}{How would you find the equation of this plane?}
\end{example}
}
%-------------- end slide -------------------------------%

%-------------- start slide -------------------------------%
\frame{
\begin{problem}
Let $\vect{u}=\leftB 
\begin{array}{rrr}
1  & 1 & 0
\end{array}
\rightB^T$ and
$\vect{v}=\leftB 
\begin{array}{rrr}
3  & 2 & 0
\end{array}
\rightB^T \in \mathbb{R}^{3}$. Show that $\vect{w} = \leftB 
\begin{array}{rrr}
4 & 5 & 0 
\end{array}
\rightB^{T}$ is in $\func{span} \left\{ \vect{u}, \vect{v} \right\}$.
\end{problem}

\pause
\begin{block}{Solution}
For a vector to be in $\func{span} \left\{ \vect{u}, \vect{v} \right\}$, it must be a linear combination of these vectors. If $\vect{w} \in \func{span} \left\{ \vect{u}, \vect{v} \right\}$, we must be able to find scalars $a,b$ such that\[
\vect{w} = a \vect{u} +b \vect{v}
\]

We proceed as follows.
\[
\leftB \begin{array}{r}
4 \\
5 \\
0
\end{array}
\rightB
=
a 
\leftB \begin{array}{r}
1 \\
1 \\
0
\end{array}
\rightB
+
b
\leftB \begin{array}{r}
3 \\
2 \\
0
\end{array}
\rightB
\]
This is equivalent to the following system of equations
\begin{eqnarray*}
a + 3b &=& 4 \\
a + 2b &=& 5
\end{eqnarray*}
\end{block}
}
%-------------- end slide -------------------------------%

%-------------- start slide -------------------------------%
\frame{
\begin{block}{Solution (continued)}
We solving this system the usual way, constructing the augmented matrix and row reducing to find the \rref.
\[
\leftB \begin{array}{rr|r}
1 & 3 & 4 \\
1 & 2 & 5 
\end{array}
\rightB
\rightarrow \cdots \rightarrow
\leftB \begin{array}{rr|r}
1 & 0 & 7 \\
0 & 1 & -1
\end{array}
\rightB
\]
The solution is $a=7, b=-1$. This means that 
\[
\vect{w} = 7 \vect{u} - \vect{v}
\] 
Therefore we can say that $\vect{w}$ is in $\func{span} \left\{ \vect{u}, \vect{v} \right\}$. 
\end{block}
}
%-------------- end slide -------------------------------%

%-------------- start slide -------------------------------%
\frame{
\begin{problem}\em
Let $\vect{u}=\leftB 
\begin{array}{rrr}
1  & 1 & \alert{1}
\end{array}
\rightB^T$ and
$\vect{v}=\leftB 
\begin{array}{rrr}
3  & 2 & 0
\end{array}
\rightB^T \in \mathbb{R}^{3}$. Show that $\vect{w} = \leftB 
\begin{array}{rrr}
4 & 5 & 0 
\end{array}
\rightB^{T}$ is in $\func{span} \left\{ \vect{u}, \vect{v} \right\}$.

\medskip

\alert{This is almost identical to the previous, except that 
$\vect{u}$ (above) has one entry that is different.}
\end{problem}

\pause

\begin{block}{Solution}
In this case, the system of linear equations is inconsistent which you can verify. Therefore $\vect{w} \notin \func{span}\left\{ \vect{u}, \vect{v} \right\}$. 
\end{block}
}
%-------------- end slide -------------------------------%


%-------------- start slide -------------------------------%
\frame{
\begin{problem}\em
Let $\vect{x},\vect{y}\in\mathbb{R}^n$,  
$U_1= \Span\{\vect{x},\vect{y}\}$, and $U_2 = \Span\{2\vect{x}-\vect{y}, 2\vect{y}+\vect{x}\}$.
Prove that $U_1=U_2$.
\end{problem}
\pause
\begin{block}{Solution}
\alert{To show that $U_1=U_2$,
prove that 
$U_1\subseteq U_2, \mbox{ and }
U_2\subseteq U_1$.}
\pause
\medskip

Since $2\vect{x}-\vect{y}, 2\vect{y}+\vect{x}\in U_1$,
it follows
that $\Span\{2\vect{x}-\vect{y}, 2\vect{y}+\vect{x}\}\subseteq U_1$, i.e.,
$U_2\subseteq U_1$.
\medskip

Also, since
\begin{eqnarray*}
\vect{x} & = & \frac{2}{5}\left( 2\vect{x}-\vect{y}\right) +
\frac{1}{5}\left(2\vect{y}+\vect{x}\right), \\
\vect{y} & = & -\frac{1}{5}\left( 2\vect{x}-\vect{y}\right) +
\frac{2}{5}\left(2\vect{y}+\vect{x}\right),
\end{eqnarray*}
\vspace*{-.2in}

$\vect{x},\vect{y}\in U_2$. 
Therefore,
$\Span\{\vect{x},\vect{y}\}\subseteq U_2$,
i.e., $U_1\subseteq U_2$.  
The result now follows.
\end{block}
}
%-------------- end slide -------------------------------%

%-------------- start slide -------------------------------%
\frame{
\begin{definition}
Let $\vect{e}_j$ denote the $j^{th}$ column of $I_n$, the
$n\times n$ identity matrix;
$\vect{e}_j$ is called the \alert{$j^{th}$ coordinate vector}
of $\mathbb{R}^n$.
\end{definition}
\pause


\begin{block}{Claim}
$\mathbb{R}^n=\Span\{ \vect{e}_1, \vect{e}_2, \ldots, \vect{e}_n \}$.
\end{block}
\pause

\begin{proof}
Let $\vect{x}=\leftB \begin{array}{c}
x_1 \\ x_2 \\ \vdots \\ x_n \end{array}\rightB\in \mathbb{R}^n$.  
Then 
$\vect{x} = x_1\vect{e}_1 + x_2\vect{e}_2 + \cdots + x_n\vect{e}_n$,
where $x_1, x_2, \ldots, x_n\in\mathbb{R}$. 
Therefore,
$\vect{x}\in \Span\{ \vect{e}_1, \vect{e}_2, \ldots, \vect{e}_n \}$,
and thus
$\mathbb{R}^n\subseteq \Span\{ \vect{e}_1, \vect{e}_2, \ldots, \vect{e}_n \}$.
\medskip

Conversely, since $\vect{e}_i\in\mathbb{R}^n$ for each $i$, $1\leq i\leq n$
(and $\mathbb{R}^n$ is a vector space), it follows that
$\Span\{ \vect{e}_1, \vect{e}_2, \ldots, \vect{e}_n \} \subseteq \mathbb{R}^n$.
The equality now follows.
\end{proof}
}
%-------------- end slide -------------------------------%

%-------------- start slide -------------------------------%
\frame{
\begin{problem}\em
Let
$\vect{u}_1=\leftB\begin{array}{r} 1 \\ -1\\ 1\\ -1 \end{array}\rightB,
\vect{u}_2=\leftB\begin{array}{r} -1 \\ 1\\ 1\\ 1 \end{array}\rightB,
\vect{u}_3=\leftB\begin{array}{r} 1 \\ -1\\ -1\\ 1 \end{array}\rightB,
\vect{u}_4=\leftB\begin{array}{r} 1 \\ -1\\ 1\\ 1 \end{array}\rightB$.
\medskip

Show that $\Span \{ \vect{u}_1, \vect{u}_2, \vect{u}_3, \vect{u}_4 \}\neq \mathbb{R}^4$.
\end{problem}
\pause
\begin{block}{Solution}
If you check, you'll find that $\vect{e}_2$ can not be written
as a linear combination of $\vect{u}_1, \vect{u}_2, \vect{u}_3$, 
and $\vect{u}_4$.
\end{block}
}
%-------------- end slide -------------------------------%


}\end{document}
