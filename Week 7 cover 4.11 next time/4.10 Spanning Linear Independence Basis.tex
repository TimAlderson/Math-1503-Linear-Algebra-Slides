%%%%%%%%%%%%%%%%%%%%%%
%%%%%%%%%%%%%%%%%%%%%%
%%Options for presentations (in-class) and handouts (e.g. print). 
\documentclass[pdf
%,handout
]{beamer}
%\usepackage{pgfpages}
%\pgfpagesuselayout{2 on 1}[letterpaper,border shrink=5mm]

%%%%%%%%%%%%%%%%%%%%%%
%% Change this for different slides so it appears in bar
\usepackage{authoraftertitle}
\date{Section 4.10\\$\mathbb{R}^n$: An Overview of Spanning, Linear Independence, and Basis}

%%%%%%%%%%%%%%%%%%%%%%
%% Upload common style file
\usepackage{../LyryxLinearAlgebraSlidesStyle}

\begin{document}
	
	%%%%%%%%%%%%%%%%%%%%%%%
	%% Title Page and Copyright Common to All Slides
	
	%Title Page
	\input ../frontmatter/titlepage.tex
	
	%LOTS Page
	%\input frontmatter/lyryxopentexts.tex
	
	%Copyright Page
	\input ../frontmatter/copyright.tex
	
	%%%%%%%%%%%%%%%%%%%%%%%%%

\section{Spanning}

%-------------- start slide -------------------------------%
\frame{

\begin{definition}[Recall: Linear Combination]
Let $\vect{u}_1,\cdots ,\vect{u}_n, \vect{v}$ be vectors. Then 
$\vect{v}$ is said to be a \textbf{linear combination }
of the vectors $\vect{u}_1,\cdots , \vect{u}_n $ 
if there exist scalars, $a_{1},\cdots ,a_{n}$ such
that
\begin{equation*}
\vect{v} = a_1 \vect{u}_1 + \cdots + a_n \vect{u}_n
\end{equation*}
\end{definition}

\medskip

\pause

\begin{definition}[Span of a Set of Vectors]
The collection of all linear combinations of a set of vectors $\{ \vect{u}_1,
\cdots ,\vect{u}_k\}$ in $\mathbb{R}^{n}$ is known as the span of these
vectors and is written as $\func{span} \{\vect{u}_1, \cdots , \vect{u}_k\}$.
\end{definition}

\pause
\medskip

{\bf Additional Terminology.}
If $U=\Span \{\vect{u}_1, \vect{u}_2, \ldots, \vect{u}_k\}$, then 
\begin{itemize}
\item \textcolor{red}{$U$ is spanned by} the vectors 
$\vect{u}_1, \vect{u}_2, \ldots, \vect{u}_k$.
\item the vectors $\vect{u}_1, \vect{u}_2, \ldots, \vect{u}_k$ 
\textcolor{red}{span $U$}.
\item the set of vectors $\{ \vect{u}_1, \vect{u}_2, \ldots, \vect{u}_k\}$ is
a \textcolor{red}{spanning set} for $U$.
\end{itemize}
}
%-------------- end slide -------------------------------%
%-------------- start slide -------------------------------%
\frame{
\begin{problem}
Let $\vect{u}=\leftB 
\begin{array}{rrr}
1  & 1 & 0
\end{array}
\rightB^T$ and
$\vect{v}=\leftB 
\begin{array}{rrr}
3  & 2 & 0
\end{array}
\rightB^T \in \mathbb{R}^{3}$. Show that $\vect{w} = \leftB 
\begin{array}{rrr}
4 & 5 & 0 
\end{array}
\rightB^{T}$ is in $\func{span} \left\{ \vect{u}, \vect{v} \right\}$.
\end{problem}

\pause
\begin{block}{Solution}
For a vector to be in $\func{span} \left\{ \vect{u}, \vect{v} \right\}$, it must be a linear combination of these vectors. If $\vect{w} \in \func{span} \left\{ \vect{u}, \vect{v} \right\}$, we must be able to find scalars $a,b$ such that\[
\vect{w} = a \vect{u} +b \vect{v}
\]

%We proceed as follows.
\[
\leftB \begin{array}{r}
4 \\
5 \\
0
\end{array}
\rightB
=
a 
\leftB \begin{array}{r}
1 \\
1 \\
0
\end{array}
\rightB
+
b
\leftB \begin{array}{r}
3 \\
2 \\
0
\end{array}
\rightB
\]
This is equivalent to the following system of equations
\begin{eqnarray*}
a + 3b &=& 4 \\
a + 2b &=& 5
\end{eqnarray*}
\end{block}
}
%-------------- end slide -------------------------------%

%-------------- start slide -------------------------------%
\frame{
\begin{block}{Solution (continued)}
We solving this system the usual way, constructing the augmented matrix and row reducing to find the \rref.
\[
\leftB \begin{array}{rr|r}
1 & 3 & 4 \\
1 & 2 & 5 
\end{array}
\rightB
\rightarrow \cdots \rightarrow
\leftB \begin{array}{rr|r}
1 & 0 & 7 \\
0 & 1 & -1
\end{array}
\rightB
\]
The solution is $a=7, b=-1$. This means that 
\[
\vect{w} = 7 \vect{u} - \vect{v}
\] 
Therefore we can say that $\vect{w}$ is in $\func{span} \left\{ \vect{u}, \vect{v} \right\}$. 
\end{block}
}
%-------------- end slide -------------------------------%
%-----------------start slide-----------------------%
\frame{\frametitle{Span of a Set of Vectors}

%\begin{definition}
%Let $\left\{ \vect{u}_1, \vect{u}_2, ..., \vect{u}_k \right\}$ be a set of vectors in $\mathbb{R}^n$. Then the collection of all linear combinations of these vectors is called the \alert{span} of these vectors, written span$\left\{ \vect{u}_1, \vect{u}_2, ..., \vect{u}_k \right\}$.
%\end{definition}
%
%\medskip

\begin{example}
Let $\vect{x}\in\mathbb{R}^3$ be a nonzero vector. 
Then $\Span\{\vect{x}\} = \{ k\vect{x} ~|~ k\in\mathbb{R}\}$
is a line through the origin having direction vector $\vect{x}$.
\end{example}
\pause


\uncover<2->
{
\begin{problem}\em
Describe the span of the vectors $\vect{u} = \left[ \begin{array}{r}
0 \\
1 \\
2 
\end{array}
\right]$ and 
$\vect{v} = \left[ 
\begin{array}{r}
0 \\
2 \\
3
\end{array}
\right]$. 
\end{problem}
}
}
%---------------------end slide--------------------%

%---------------------- start slide----------------------%
\frame{
\begin{solution}\em
Notice that any linear combination of the vectors $\vect{u}$ and $\vect{v}$ yields a vector $\left[ \begin{array}{r}
0 \\
y \\
z 
\end{array}
\right]$ in the $YZ$-plane. 

\medskip

\uncover<2->{
Suppose we take an arbitrary vector $\left[ \begin{array}{r}
0 \\
y \\
z 
\end{array}
\right]$ in the $YZ$-plane. It turns out we can write any such vector as a linear combination of $\vect{u}$ and $\vect{v}$. 
}

\medskip

\uncover<3->{
\[
\left[ \begin{array}{r}
0 \\
y \\
z 
\end{array}
\right]
=
(-3y + 2z) \left[
\begin{array}{r}
0 \\
1 \\
2 
\end{array}
\right]
+
(2y - z) \left[ 
\begin{array}{r}
0 \\
2 \\
3 
\end{array}
\right]
\]
}

\medskip

\uncover<4->{
Hence, span$\left\{ \vect{u}, \vect{v} \right\}$ is the $YZ$-plane.}

\end{solution}
}
%------------------------end slide------------------------%

%------------------------ start slide--------------------%
\frame{\frametitle{Span of a Set of Vectors}
Consider the previous example where the span of $\vect{u}$ and $\vect{v}$ was the $YZ$-plane. Suppose we add another vector $\vect{w}$, and consider the span of $\vect{u}, \vect{v}$, and $\vect{w}$. What would happen to the span?

\bigskip

\uncover<2->{
\textbf{Scenario 1}
Suppose $\vect{w}$ is a vector in the $YZ$-plane. For example, $\vect{w} = \left[
\begin{array}{r}
0 \\
4 \\
1 
\end{array}
\right]$. Then $\vect{w}$ is in the span of $\vect{u}, \vect{v}$. Adding $\vect{w}$ to the set doesn't change the span at all. 
\[
\mbox{span}\left\{ \vect{u}, \vect{v}, \vect{w} \right\} = \mbox{span}\left\{ \vect{u}, \vect{v} \right\}
\]
}
}
%------------------------end slide---------------------%

%-----------------------start slide--------------------%
\frame{\frametitle{Span of a Set of Vectors}
\textbf{Scenario 2}
Suppose $\vect{w}$ is not in the $YZ$-plane. For example, suppose $\vect{w} = \left[
\begin{array}{r}
1 \\
0 \\
2 
\end{array}
\right]
$. 

\uncover<2->{
Notice that now, the three vectors span $\mathbb{R}^3$. Any vector in $\mathbb{R}^3$ can be written as a linear combination of $\vect{u}, \vect{v}, \vect{w}$ as follows:
\[
\left[
\begin{array}{r}
x \\
y \\
z 
\end{array}
\right]
=
(-4x + 5y + 2z)
\left[
\begin{array}{r}
0 \\
1 \\
2 
\end{array}
\right]
+ (2x + 2y - z)
\left[
\begin{array}{r}
0 \\
2 \\
3 
\end{array}
\right]
+
(x)
\left[
\begin{array}{r}
1 \\
0 \\
2
\end{array}
\right]
\]
}

\uncover<3->{
You can see that the span of these three vectors depended on whether $\vect{w}$ was in span$\left\{ \vect{u}, \vect{v} \right\}$ or not. In the next section, we will examine the distinction between these two scenarios using the concept of linear independence. 
}
}
%----------------------end slide-------------------%

\section{Independence}

%-------------------start slide---------------------%
\frame{\frametitle{Linearly Independent Set of Vectors}

\begin{definition}
Let $\left\{ \vect{u}_1, \vect{u}_2, ..., \vect{u}_k \right\}$ be a set of vectors in $\mathbb{R}^n$. This set is \alert{linearly independent} if no vector in the set is in the span of the other vectors of that set. 
\end{definition}


\uncover<2->{
If a set of vectors is not linearly independent, we call it \alert{linearly dependent}. 
}
}
%--------------------end slide------------------------%

%--------------------start slide------------------------%
\frame{\frametitle{A Linearly Dependent Set}

\begin{problem}\em
Consider the vectors $
\vect{u} = \left[
\begin{array}{r}
0 \\
1  \\
2 
\end{array}
\right], 
\vect{v} = \left[
\begin{array}{r}
0 \\
2 \\
3
\end{array}
\right], \vect{w} = \left[
\begin{array}{r}
0 \\
4 \\
1 
\end{array}
\right]$.  
Is the set $\left\{ \vect{u}, \vect{v}, \vect{w} \right\}$ linearly independent?
\end{problem}


\uncover<2->{
\begin{solution}\em
Notice that we can write $\vect{w}$ as a linear combination of $\vect{u}, \vect{v}$ as follows:
\[
\left[
\begin{array}{r}
0 \\
4 \\
1 
\end{array}
\right]
=
(-10)
\left[
\begin{array}{r}
0 \\
1 \\
2 
\end{array}
\right]
+
(7)
\left[
\begin{array}{r}
0 \\
2 \\
3 
\end{array}
\right]
\]
Hence, $\vect{w}$ is in span$\left\{ \vect{u}, \vect{v} \right\}$. By the definition, this set is not linearly independent (it is linearly dependent). 
}
\end{solution}
}
%----------------------end slide--------------------%

%----------------------start slide-------------------%
\frame{\frametitle{A Linearly Independent Set}

\begin{problem}\em
Consider the vectors
\[
\vect{u} = \left[
\begin{array}{r}
0 \\
1 \\
2 
\end{array}
\right], 
\vect{v} = \left[
\begin{array}{r}
0 \\
2 \\
3
\end{array}
\right],
\vect{w} = \left[
\begin{array}{r}
1 \\
0 \\
2 
\end{array}
\right]
\] 
Is the set $\left\{ \vect{u}, \vect{v}, \vect{w} \right\}$ linearly independent?
\end{problem}

\medskip

\uncover<2->{
\begin{solution}\em
We cannot write any of the three vectors as a linear combination of the other two. (We will see how to show this soon.) Therefore the set $\left\{ \vect{u}, \vect{v}, \vect{w} \right\}$ is \alert{linearly independent}. 
}
\end{solution}
}
%------------------------end slide--------------------%

%------------------start slide----------------------------%
\frame{\frametitle{Linear Independence as a Linear Combination}

The following theorem provides a familiar way to check if a set of vectors is linearly independent.

\medskip

\uncover<2->{
\begin{theorem}\em
The collection of vectors, $\left\{\vect{u}_{1},\cdots ,\vect{u}_{k}\right\}$ in
$\mathbb{R}^{n}$  is linearly independent if
and only if whenever
\begin{equation*}
\sum_{i=1}^{n}a_{i}\vect{u}_{i}=\vect{0}
\end{equation*}
it follows that each $a_{i}=0$.
}

\uncover<3->{
Thus \alert{$\left\{\vect{u}_{1},\cdots ,\vect{u}_{k}\right\} $ in
$\mathbb{R}^{n}$ is linearly independent exactly when the system of
linear equations $AX=0$ has only the trivial solution}, where $A$ is
the $n \times k$ matrix having these vectors as columns. 
\end{theorem}
}

}
%-----------------------end slide------------------------%

%-------------------start slide---------------------%
\frame{\frametitle{Linear Independence}
We can state the conclusion of this theorem in another way: The set of vectors $\left\{ \vect{u}_1, ..., \vect{u}_k \right\}$ is linearly independent if and only if there is no nontrivial linear combination which equals zero. If a linear combination of the vectors equals zero, then all the coefficients of the combination are zero. 

\medskip

\uncover<2->{
If the set is linearly independent, then 
\[
a_1 \vect{u}_1 + ... + a_k \vect{u}_k = 0
\]
implies that 
\[
a_1 = a_2 = ... = a_k = 0
\]
}

}
%---------------------end slide--------------------------%

%--------------------start slide----------------------%
\frame{\frametitle{Linear Independence}

We can use the reduced row-echelon form of the matrix to determine if the columns form a linearly independent set of vectors. 

\medskip

\uncover<2->{
\begin{problem}\em
Determine whether the following set of vectors are linearly independent. 
\[
\left\{
\left[
\begin{array}{c}
1 \\
2 \\
3
\end{array}
\right],
\left[
\begin{array}{c}
2 \\
1 \\
0
\end{array}
\right],
\left[
\begin{array}{c}
0 \\
1 \\
1
\end{array}
\right]
\right\}
\]
\end{problem}
}


}
%-----------------------end slide----------------------%

%---------------------start slide--------------------%
\frame{
\begin{solution}\em
Construct the $3$ x $3$ matrix $A$ having these vectors as columns:
\[
A = \left[
\begin{array}{rrr}
1 & 2 & 0 \\
2 & 1 & 1 \\
3 & 0 & 1 
\end{array}
\right]
\]

\uncover<2->{
By the above theorem, the set of vectors is linearly independent if the system $AX=0$ has only the trivial solution. We can see this from the reduced row-echelon form of the matrix $A$. 
}

\uncover<3->{
\[
\left[
\begin{array}{rrr}
1 & 0 & 0 \\
0 & 1 & 0 \\
0 & 0 & 1 
\end{array}
\right]
\]
}
\uncover<4->{
Since all columns are pivot columns (and the rank of $A$ is $3$), the vectors are linearly independent. 
}
\end{solution}

}
%------------------------end slide----------------%

%------------------------start slide-------------%
\frame{
\begin{problem}\em
Determine whether the following vectors are linearly independent. If they are linearly dependent, write one of the vectors as a linear combination of the others. 
\[
\left\{
\left[
\begin{array}{c}
1 \\
2 \\
4 \\
1
\end{array}
\right],
\left[
\begin{array}{c}
2 \\
7 \\
17 \\
2
\end{array}
\right],
\left[
\begin{array}{c}
0 \\
1 \\
3 \\
0
\end{array}
\right],
\left[
\begin{array}{c}
8 \\
5 \\
11 \\
11
\end{array}
\right]
\right\}
\]
\end{problem}

\uncover<2->{
\begin{solution}\em
Construct the matrix $A$ using these vectors as columns. 
\[
A = 
\left[
\begin{array}{rrrr}
1 & 2 & 0 & 8 \\
2 & 0 & 1 & 5 \\
4 & 0 & 3 & 11 \\
1 & 3 & 0 & 11 
\end{array}
\right]
\]
}
\end{solution}
}
%-------------------------end slide---------------%

%-------------------start slide----------------------%
\frame{
\begin{solution}[continued]\em
The reduced row-echelon form of this matrix is 
\[
\left[
\begin{array}{rrrr}
1 & 0 & 0 & 2 \\
0 & 1 & 0 & 1 \\
0 & 0 & 1 & 3 \\
0 & 0 & 0 & 0 
\end{array}
\right]
\]

\uncover<2->{
Since the rank of $A$ is $3 < 4$, the vectors are linearly dependent. 
}

\uncover<3->{
Therefore, there are infinitely many solutions to $AX=0$, one of which is 
\[
\left[
\begin{array}{r}
-2 \\
-1 \\
-3 \\
1
\end{array}
\right]
\]
}
\end{solution}
}
%----------------end slide------------------------%

%--------------------start slide--------------%
\frame{
\begin{solution}[continued]\em
Therefore we can write:
\[
-2 \left[
\begin{array}{r}
1 \\
2 \\
4 \\
1
\end{array}
\right]
- 1 \left[
\begin{array}{r}
0 \\
1 \\
3 \\
0
\end{array}
\right]
-3 \left[
\begin{array}{r}
2 \\
0 \\
0 \\
3
\end{array}
\right]
+ 1  \left[
\begin{array}{r}
8 \\
5 \\
11 \\
11
\end{array}
\right]
=
\left[
\begin{array}{r}
0 \\
0 \\
0 \\
0
\end{array}
\right]
\]

\uncover<2->{
We can rewrite this as:
\[
2 \left[
\begin{array}{r}
1 \\
2 \\
4 \\
1
\end{array}
\right]
+ 1 \left[
\begin{array}{r}
0 \\
1 \\
3 \\
0
\end{array}
\right]
+ 3 \left[
\begin{array}{r}
2 \\
0 \\
0 \\
3
\end{array}
\right]
=  \left[
\begin{array}{r}
8 \\
5 \\
11 \\
11
\end{array}
\right]
\]
This shows that one of the vectors can be written as a linear combination of the other three vectors. While here we chose the fourth vector, we could have chosen any of the vectors to isolate. 
}
\end{solution}
}
%---------------------end slide--------------------%

%%-----------------------start slide----------------%
%\frame{\frametitle{Linear Dependence in $\mathbb{R}^n$}
%
%\begin{theorem}\em
%Let $\left\{ \vect{u}_1, \vect{u}_2, ..., \vect{u}_k \right\}$ be a set of vectors in $\mathbb{R}^n$. If $k > n$ then the set is linearly dependent. 
%\end{theorem}
%
%}
%%------------------end slide-------------------------%

\section{Subspaces and Basis}

%-------------------start slide-----------------------%
\frame{\frametitle{Subspaces}

\begin{definition}
Let $V$ be a nonempty collection of vectors in $\mathbb{R}^n$. Then $V$ is a \alert{subspace} if whenever $a$ and $b$ are scalars and $\vect{u}$ and $\vect{v}$ are vectors in $V$, $a\vect{u} + b\vect{v}$ is also in $V$.
\end{definition}

\medskip

\uncover<2->{
Subspaces are closely related to the span of a set of vectors which we discussed earlier.
}

\medskip

\uncover<3->{
\begin{theorem}\em
Let $V$ be a nonempty collection of vectors in $\mathbb{R}^n$. Then $V$ is a subspace of $\mathbb{R}^n$ if and only if there exist vectors $\left\{ \vect{u}_1, ..., \vect{u}_k\right\}$ in $V$ such that 
\[
V = \mbox{span}\left\{\vect{u}_1, ..., \vect{u}_k\right\}
\]
\end{theorem}
}

}
%---------------------end slide--------------------%

%----------------------start slide---------------%
\frame{\frametitle{Subspaces}

Subspaces are also related to the property of linear independence.

\medskip

\uncover<2->{
\begin{theorem}\em
If $V$ is a subspace of $\mathbb{R}^n$, then there exist linearly independent vectors $\left\{\vect{u}_1, ..., \vect{u}_k\right\}$ of $V$ such that 
\[
V = \mbox{span}\left\{\vect{u}_1, ..., \vect{u}_k\right\}
\]
\end{theorem}
}

\uncover<3->{
In other words, subspaces of $\mathbb{R}^n$ consist of spans of finite, linearly independent collections of vectors in $\mathbb{R}^n$. 
}
}
%---------------------end slide------------------%

%----------------------start slide----------------%
\frame{\frametitle{Basis of a Subspace}

\begin{definition}
Let $V$ be a subspace of $\mathbb{R}^n$. Then $\left\{\vect{u}_1, ..., \vect{u}_k\right\}$ is called a \alert{basis} for $V$ if the following conditions hold:
\begin{itemize}
\item<2->
span$\left\{\vect{u}_1, ..., \vect{u}_k\right\} = V$
\item<3->
$\left\{\vect{u}_1, ..., \vect{u}_k\right\}$ is linearly independent.
\end{itemize}
\end{definition}

\uncover<4->{
The following theorem claims that any two bases of a subspace must be of the same size. 
}

\uncover<5->{
\begin{theorem}\em
Let $V$ be a subspace of $\mathbb{R}^n$ and suppose $\left\{\vect{u}_1, ..., \vect{u}_k\right\}$ and $\left\{ \vect{v}_1, ..., \vect{v}_m\right\}$ are two bases for $V$. Then $k=m$. 
\end{theorem}
}

}
%------------------------end slide------------------%

%-------------------------start slide---------------%
\frame{\frametitle{Dimension}
The previous theorem shows than all bases of a subspace will have the same size. This size is called the \alert{dimension} of the subspace.

\medskip

\uncover<2->{
\begin{definition}
Let $V$ be a subspace of $\mathbb{R}^n$. Then the \alert{dimension} of $V$ is the number of a vectors in a basis of $V$.
\end{definition}
}

}
%-----------------------end slide ---------------------%

%------------------------start slide---------------%
\frame{\frametitle{Properties of $\mathbb{R}^n$}

Note that the dimension of $\mathbb{R}^n$ is $n$. 

\medskip

\uncover<2->{
There are some other important properties of vectors in $\mathbb{R}^n$. 
}
\uncover<3->{

%\begin{theorem}\em
%Let $\left\{ \vect{u}_1, \vect{u}_2, ..., \vect{u}_k \right\}$ be a set of vectors in $\mathbb{R}^n$. If $k > n$ then the set is linearly dependent. 
%\end{theorem}

\begin{theorem}\em
\begin{itemize}
\item<4->
If $\left\{ \vect{u}_1, ..., \vect{u}_{\alert{n}} \right\}$ is a linearly independent set of a vectors in $\mathbb{R}^{\alert{n}}$, then $\left\{ \vect{u}_1, ..., \vect{u}_n \right\}$ is a basis for $\mathbb{R}^n$. \\
\item<5->
Suppose $\left\{ \vect{u}_1, ..., \vect{u}_{\alert{m}} \right\}$ spans $\mathbb{R}^{\alert{n}}$. Then \alert{$m \ge n$}.\\ 
\item<6->
If $\left\{ \vect{u}_1, ..., \vect{u}_{\alert{n}} \right\}$ spans $\mathbb{R}^{\alert{n}}$, then $\left\{ \vect{u}_1, ..., \vect{u}_n \right\}$ is linearly independent.\\
\item<6-> If $\left\{ \vect{u}_1, \vect{u}_2, ..., \vect{u}_{\alert k} \right\}$ is a set of vectors in $\mathbb{R}^{\alert n}$ with { $\alert{ k > n}$}, then the set is linearly dependent. 
\end{itemize}
\end{theorem}
}
\uncover<7->{ \textit{It follows then that a basis is a minimal spanning set. If a subspace has dimension $ d $, then any spanning set has size at least $ d $, and any spanning set of size $ d $ must be a basis (and is therefore independent).}  
}
}
%---------------------end slide---------------------%

\section{Special Spaces}

%-------------------start slide-----------------------%
\frame{\frametitle{Row and Column Space}

\begin{definition}
Let $A$ be an $m \times n$ matrix. The \alert{column space} of $A$ is the span of the columns of $A$. The \alert{row space} of $A$ is the span of the rows of $A$. 
\end{definition}

\medskip

\uncover<2->{
\begin{problem}\em
Find the rank of the matrix $A$ and describe the column and row spaces efficiently. 

\[
A = \left[
\begin{array}{rrrrr}
1 & 2 & 1 & 3 & 2 \\
1 & 3 & 6 & 0 & 2 \\
3 & 7 & 8 & 6 & 6 
\end{array}
\right]
\]
\end{problem}
}

}
%------------------end slide-----------------------------%

%------------------start slide-----------------------%
\frame{\frametitle{Example: Column Space}
\begin{solution}\em
To find the column space, we first find the reduced row-echelon form of $A$:
\[
\left[ 
\begin{array}{rrrrr}
1 & 0 & -9 & 9 & 2 \\
0 & 1 & 5 & -3 & 0 \\
0 & 0 & 0 & 0 & 0 
\end{array}
\right]
\]

\uncover<2->{
Therefore rank$(A)=2$. 
}

\uncover<3->{
Note the first two columns are the pivot columns. All columns of the above reduced row-echelon matrix are in 
\[
\mbox{span} \left\{ 
\left[
\begin{array}{r}
1 \\
0 \\
0 
\end{array}
\right], 
\left[
\begin{array}{r}
0 \\
1 \\
0
\end{array}
\right]
\right\}
\]
}

\end{solution}
}
%--------------------end slide------------------------%

%--------------------start slide-----------------------%
\frame{\frametitle{Example: Column Space}

\begin{solution}[continued]\em
To construct the column space, we use the pivot columns of the original matrix - in this case, the first and second columns. Therefore the column space of $A$ is 
\[
\mbox{span} \left\{ 
\left[
\begin{array}{r}
1 \\
1 \\
3
\end{array}
\right],
\left[
\begin{array}{r}
2 \\
3 \\
7
\end{array}
\right]
\right\}
\]

\end{solution}
}
%-------------------end slide--------------%

%--------------------start slide------------%
\frame{\frametitle{Example: Row Space}

\begin{solution}[continued]\em
To find the row space of $A$ we again look at the reduced row-echelon form of the matrix. 

\[
\left[ 
\begin{array}{rrrrr}
1 & 0 & -9 & 9 & 2 \\
0 & 1 & 5 & -3 & 0 \\
0 & 0 & 0 & 0 & 0 
\end{array}
\right]
\]

\uncover<2->{
The row space of $A$ is the span of the non-zero rows of the above matrix:
\[
\mbox{span} \left\{
\left[ 
\begin{array}{rrrrr}
1 & 0 & -9 & 9 & 2 
\end{array}
\right],
\left[
\begin{array}{rrrrr}
0 & 1 & 5 & -3 & 0 
\end{array}
\right]
\right\}
\]
}

\uncover<3->{
Notice that the vectors used in the description of the \alert{column space are from the original matrix}, while those in the \textcolor{blue}{row space are from the reduced row-echelon form of the original matrix}.}
\end{solution}
}
%-------------------end slide-------------------%

%----------------start slide-------------------%
\frame{\frametitle{Null Space}

\begin{definition}
The \alert{null space} of $A$, or \alert{kernel} of $A$ is defined as:
\[
\mbox{ker}(A) = \left\{ X: AX=0 \right\}
\]
\end{definition}

\medskip

\uncover<2->{
We also speak of the image of $A$, Im$(A)$, which is all vectors of the form $AX$ where $X$ is in $\mathbb{R}^n$. 
}

\uncover<3->{
To find ker$(A)$, we solve the system of equations $AX=0$. 
}

\uncover<4->{
\begin{problem}\em
Find ker$(A)$ for the matrix $A$:
\[
A = 
\left[
\begin{array}{rrr}
1 & 2 & 1 \\
0 & -1 & 1 \\
2 & 3 & 3 
\end{array}
\right]
\]
\end{problem}
}
}
%-------------------end slide---------------------%

%--------------------start slide--------------------%
\frame{\frametitle{Null Space}

\begin{solution}\em
The first step is to set up the augmented matrix:
\[ 
\left[
\begin{array}{rrr|r}
1 & 2 & 1 & 0 \\
0 & -1 & 1 & 0 \\
2 & 3 & 3 & 0 
\end{array}
\right]
\]

\uncover<2->{
Place this matrix in reduced row-echelon form:
\[ 
\left[
\begin{array}{rrr|r}
1 & 0 & 3 & 0 \\
0 & 1 & -1 & 0 \\
0 & 0 & 0 & 0 
\end{array}
\right]
\]
}

\end{solution}

}
%---------------------end slide----------------------%

%-----------------start slide---------------------%
\frame{\frametitle{Null Space}

\begin{solution}[continued]\em
The solution to this system of equations is
\[
\left\{ \left[
\begin{array}{r}
3t \\
t \\
t 
\end{array}
\right]  : 
t \in \mathbb{R} \right\}
\]

\uncover<2->{
Therefore the null space of $A$ is the span of this vector:
\[
\mbox{ker}(A) = \mbox{span} \left\{
\left[
\begin{array}{r}
3 \\
1 \\
1
\end{array}
\right]
\right\}
\]
}

\end{solution}
}
%---------------------end slide---------------------%

%---------------------start slide--------------------%
\frame{\frametitle{Nullity}

\begin{definition}
The dimension of the null space of a matrix is called the \alert{nullity}, denoted null$(A)$. 
\end{definition}

\medskip

\uncover<2->{
\begin{theorem}\em
Let $A$ be an $m \times n$ matrix. Then, 
\[
\mbox{rank}(A) + \mbox{null}(A) = n
\]
\end{theorem}
}

\medskip

\uncover<3->{
For instance, in the last example, $A$ was a $3 \times 3$ matrix. The rank was $2$ and the nullity was $1$ (since the null space had dimension $1$).
\[
\mbox{rank}(A) + \mbox{null}(A) = 2 + 1 = 3 = n
\]
}
}
%-----------------------end slide------------------%
\end{document}
