%%%%%%%%%%%%%%%%%%%%%%
%%Options for presentations (in-class) and handouts (e.g. print). 
\documentclass[pdf
%,handout
]{beamer}
\usepackage{pgfpages}
%\pgfpagesuselayout{2 on 1}[letterpaper,border shrink=5mm]

\graphicspath{{../}}
%%%%%%%%%%%%%%%%%%%%%%
%% Change this for different slides so it appears in bar
\usepackage{authoraftertitle}
\date{5.9: The General Solution to a Linear System}

%%%%%%%%%%%%%%%%%%%%%%
%% Upload common style file
\usepackage{../LyryxLinearAlgebraSlidesStyle}

\begin{document}
	
	%%%%%%%%%%%%%%%%%%%%%%%
	%% Title Page and Copyright Common to All Slides	
	%Title Page
	\input ../frontmatter/titlepage.tex	
	%LOTS Page
	%\input frontmatter/lyryxopentexts.tex	
	%Copyright Page
	\input ../frontmatter/copyright.tex	
	%%%%%%%%%%%%%%%%%%%%%%%%%


\section{The General Solution to a Linear System}

%------------------start slide----------------------%
{\small
\frame{
\begin{definition}
Suppose a linear system of equations can be written in the form 
\[
T(\vect{x}) = \vect{b}
\]
where $T$ is a linear transformation determined by $T(\vect{x}) = A\vect{x}$.

\pause

If $T(\vect{x}_p) = \vect{b}$, then $\vect{x}_p$ is called a \alert{particular solution} of the linear system
\end{definition}

\pause

\begin{example}
Suppose that a system of linear equations can be written as 
{\footnotesize
\[ T \left( \left[\begin{array}{c} x_1 \\ x_2 \\ x_3 \\ x_4 \end{array}\right] \right) =
\left[\begin{array}{c}
 5+2x_2-x_4 \\ x_2 \\ -4-3x_4 \\ x_4
\end{array}\right] \]}
\pause
Setting $x_2=2$ and $x_4=-1$ gives us the \alert{particular solution}
$ \vect{x}_p = \left[\begin{array}{c} x_1 \\ x_2 \\ x_3 \\ x_4 \end{array}\right] =
\left[\begin{array}{c}
 10 \\ 2 \\ -1 \\ -1
\end{array}\right].$
\end{example}
}}
%-----------------end slide-----------------------------%

%-----------------start slide-------------------%
\frame{\frametitle{Null Space and Associated Homogeneous System}

\begin{definition}
Let $T$ be a linear transformation. Define:
\[
ker(T) = \left\{ \vect{x} : T(\vect{x}) = \vect{0} \right\}
\]

\pause
The \alert{kernel}, $ker (T)$ consists of the set of all vectors $\vect{x}$ such that $T(\vect{x}) = \vect{0}$. This is also called the \alert{null space} of $T$ or the \alert{solution space} of $T(\vect{x}) = \vect{0}$. 
\end{definition}

\pause

\begin{definition}
Let $A\vect{x} = \vect{b}$ be a system of equations. Then, the corresponding system given by $A\vect{x} = \vect{0}$, found by replacing $\vect{b}$ with $\vect{0}$ is the \alert{associated homogeneous system}. 
\end{definition}
}
%-------------------end slide-----------------------%

%--------------------start slide--------------%
\frame{\frametitle{Associated Homogeneous System}

\begin{example}
Consider the system given by 
$\begin{array}{ccccccccc}
x_1 & + & x_2 & - & x_3 & + & 3x_4 & = & 2 \\
-x_1 & + & 4x_2 & + & 5x_3 & - & 2x_4 & = & 3 \\
x_1 & + & 6x_2 & + & 3x_3 & + & 4x_4 & = & 4
\end{array}$. 
\pause
We can write this as 
\[
\leftB \begin{array}{rrrr}
1 & 1 & -1 & 3 \\
-1 & 4 & 5 & -2 \\
1 & 6 & 3 & 4 
\end{array} \rightB
\leftB \begin{array}{c}
x_1 \\
x_2 \\
x_3 \\
x_4
\end{array} \rightB
=
\leftB \begin{array}{c}
2 \\
3 \\
4
\end{array}
\rightB
\]
\pause
whose \alert{associated homogeneous system} can be written as:
\[
\leftB \begin{array}{rrrr}
1 & 1 & -1 & 3 \\
-1 & 4 & 5 & -2 \\
1 & 6 & 3 & 4 
\end{array} \rightB
\leftB \begin{array}{c}
x_1 \\
x_2 \\
x_3 \\
x_4
\end{array} \rightB
=
\leftB \begin{array}{c}
0 \\
0 \\
0
\end{array}
\rightB
\]
\end{example}
}
%------------------end slide--------------------------%


%-------------- start slide -------------------------------%
\frame{\frametitle{Solutions to Systems as
Linear Combinations of Vectors}
\pause
\begin{alertblock}{Basic Solutions to Homogeneous Systems}
In our earlier treatment of homogeneous systems of linear equations,
we saw how to
write the solution to a homogeneous system of linear equations
as a \alert{linear combination} of \alert{basic solutions}.
\end{alertblock}
\pause
{\footnotesize
\begin{example}
The system
$\begin{array}{ccccccccc}
x_1 & + & x_2 & - & x_3 & + & 3x_4 & = & 0 \\
-x_1 & + & 4x_2 & + & 5x_3 & - & 2x_4 & = & 0 \\
x_1 & + & 6x_2 & + & 3x_3 & + & 4x_4 & = & 0
\end{array}$
has general solution
\pause
\[ \begin{array}{ccl}
x_1 & = & \vspace{0.05in}\frac{9}{5}s -\vspace{0.05in}\frac{14}{5}t \\
x_2 & = & -\vspace{0.05in}\frac{4}{5}s -\vspace{0.05in}\frac{1}{5}t \\
x_3 & = & s \\
x_4 & = & t
\end{array}
\pause
~\mbox{ or }~
\left[\begin{array}{c}
x_1 \\ x_2 \\ x_3 \\ x_4 \end{array} \right]
=
s\left[\begin{array}{r}
\vspace{0.05in}\frac{9}{5} \\
-\vspace{0.05in}\frac{4}{5} \\
1 \\
0
\end{array}\right]
+
t \left[\begin{array}{r}
-\vspace{0.05in}\frac{14}{5} \\
-\vspace{0.05in}\frac{1}{5} \\
0 \\
1
\end{array}\right].
\]
\end{example}
}}
%-----------------end slide-----------------------------%

%------------------start slide----------------------%
\frame{\frametitle{The General Solution to a Linear System}
\begin{theorem}\em
Let $T$ be a linear transformation given by $T(\vect{x}) = A\vect{x}$. 
Suppose that $\vect{x}_p$ is a particular solution 
of the system of linear equations given by $T(\vect{x})=\vect{b}$, i.e., $T(\vect{x}_p)=\vect{b}$.
Then if  $\vect{y}$ is any other solution of $T(\vect{x})=\vect{b}$, it can be written
in the form $\vect{y}=\vect{x}_p+\vect{x}_0$ for some (particular) solution $\vect{x}_0$
to the associated homogeneous system $A\vect{x}=0$.
\end{theorem}
\pause
\begin{alertblock}{}
The theorem implies that if $\vect{x}_p$ is a particular solution to
$A\vect{x}=\vect{b}$ and $\vect{x}_h$ is the general solution to the associated
homogeneous system $A\vect{x}=0$, then $\vect{x}_p+\vect{x}_h$ is the general
solution to $A\vect{x}=\vect{b}$.
\pause
Furthermore, the general solution to $A\vect{x}=\vect{b}$ can always be
written in the form $\vect{x}_p+\vect{x}_h$ where $\vect{x}_p$ is a particular
solution to $A\vect{x}=\vect{b}$ and $\vect{x}_h$ is the general solution to $A\vect{x}=0$.
\end{alertblock}
}
%-------------- end slide -------------------------------%
{\small
\frame{
\begin{example}
The system of linear equations $A\vect{x}=\vect{b}$, with
{\footnotesize
\vspace*{-.1in}

\[
A=\left[\begin{array}{rrrr}
2 & 1 & 3 & 3 \\
0 & 1 & -1 & 1 \\
-1 & 1 & -3 & 0
\end{array}\right] \mbox{ and }
\vect{b}=\left[\begin{array}{r}
4 \\ 2 \\ 1 
\end{array}\right]
\]}
\vspace*{-.1in}


\pause
Notice that
{\footnotesize $\vect{x}_p=\left[\begin{array}{c}
1 \\ 2 \\ 0 \\ 0
\end{array}\right]$ }
is a particular solution to $A\vect{x}=\vect{b}$ (obtained
by setting $s=t=0$), 
\pause
while
{\footnotesize
$\vect{x}_h=s\left[\begin{array}{r}
-2 \\ 1 \\ 1 \\ 0
\end{array}\right]
+ t \left[\begin{array}{r}
-1 \\ -1 \\ 0 \\ 1
\end{array}\right]$, $s,t\in\RR$}
is the the general solution to the
associated homogeneous system $A\vect{x}=0$.
\pause
Therefore the general solution is  $\vect{y}=\vect{x}_p + \vect{x}_h$:
\pause
{\footnotesize
\[ \vect{y}  =  \left[\begin{array}{c}
1-2s-t \\ 2+s-t \\ s \\ t 
\end{array}\right] 
 = \left[\begin{array}{r}
1 \\ 2 \\ 0 \\ 0
\end{array}\right]
+ s \left[\begin{array}{r}
-2 \\ 1 \\ 1 \\ 0
\end{array}\right]
+ t \left[\begin{array}{r}
-1 \\ -1 \\ 0 \\ 1
\end{array}\right],~~~s,t\in\RR.
\]}
\end{example}
}}


\end{document}
