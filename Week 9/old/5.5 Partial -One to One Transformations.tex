%%Options for presentations (in-class) and handouts (e.g. print). 
\documentclass[pdf
,handout
]{beamer}
\usepackage{pgfpages}
\pgfpagesuselayout{2 on 1}[letterpaper,border shrink=5mm]

\graphicspath{{../}}

%%%%%%%%%%%%%%%%%%%%%%
%% Change this for different slides so it appears in bar
\usepackage{authoraftertitle}
\date{Linear Transformations: One to One and Onto}

%%%%%%%%%%%%%%%%%%%%%%
%% Upload common style file
\usepackage{../LyryxLinearAlgebraSlidesStyle}

\begin{document}
	
	%%%%%%%%%%%%%%%%%%%%%%%
	%% Title Page and Copyright Common to All Slides
	
	%Title Page
	\input ../frontmatter/titlepage.tex
	
	%LOTS Page
	%\input frontmatter/lyryxopentexts.tex
	
	%Copyright Page
	\input ../frontmatter/copyright.tex
	
	%%%%%%%%%%%%%%%%%%%%%%%%%

\section{}
%%------------------start slide----------------------%
%\frame{\frametitle{Range of a Transformation}
%
%\begin{alertblock}{}
%Let $T: \mathbb{R}^n \mapsto \mathbb{R}^m$ be a linear transformation given by $T(\vect{x}) = A\vect{x}$. Consider all the vectors of the form $A\vect{x}$ for some $\vect{x} \in \mathbb{R}^n$.  This set of vectors is called the \alert{range} or \alert{image} of $T$. 
%
%\pause
%
%We denote this set as $T\mathbb{R}^n$, $T(\mathbb{R}^n)$ or Im$(T)$. Notice that these vectors $A(\vect{x})$ are in $\mathbb{R}^m$. 
%\end{alertblock}
%}
%%----------------end slide--------------------%
%
%%---------------start slide-------------------------%
%\frame{\frametitle{The Form $A\vect{x}$}
%
%\begin{theorem}\em
%Let $A$ be an $m \times n$ matrix where $A_1, ..., A_n$ denote the columns of $A$. Then for a vector $\vect{x} = \leftB
%\begin{array}{c}
%x_1 \\
%\vdots \\
%x_n
%\end{array}
%\rightB$ in $\mathbb{R}^n$,
%\[
%A\vect{x} = \sum_{k=1}^{n}x_{k}A_{k}
%\]
%\pause
%Therefore $A(\mathbb{R}^n)$ is the collection of all linear combinations of these products. 
%\end{theorem}
%}
%%------------------end slide-----------------------%

\section{One to One}
%--------------- start slide------------------------------%
\frame{\frametitle{Injections}
\begin{definition}
Let $T: \RR^n\to\RR^m$ be a linear transformation, and let
$\vect{x}_1$ and $\vect{x}_2$ be in $\mathbb{R}^n$.
\pause
We say that $T$ is an \alert{injection} or is
\alert{one-to-one} (sometimes written as 1-1)
if $\vect{x}_1 \neq \vect{x}_2$ implies that
\[ T\left( \vect{x}_1 \right) \neq T \left(\vect{x}_2\right).\]
\pause
Equivalently, if $T\left( \vect{x}_1 \right) =T\left( \vect{x}_2\right)$,
then $\vect{x}_1 = \vect{x}_2$.
Thus,  $T$ is one-to-one if two distinct vectors are 
never transformed into the same vector.
\end{definition}
\pause
\begin{theorem}\em
Let $A$ be an $m\times n$ matrix and let $\vect{x}$ be a vector of 
length $n$.
Then the transformation induced by $A$, $T_A$,
is one-to-one if and only if $A\vect{x}=0$ implies $\vect{x}=0$.
\end{theorem}
\pause
Since every linear transformation is induced by a
matrix $A$, in order to show that $T$ is one to one,
it suffices to show that $A\vect{x}=0$ has a unique solution.
}
%--------------------------end slide -----------------------------%

%------------------------start slide--------------------------------%
{\small
\frame{
\begin{problem}\em
Show that the transformation defined by 
\[
T \left[ \begin{array}{c} x \\ y \end{array} \right]
= \left[ \begin{array}{rr} 1 & 1\\ 0 & 1 \end{array} \right]
\left[ \begin{array}{c} x \\ y \end{array} \right] \]
is one-to-one.
\end{problem}
\pause
\begin{solution}\em
Since $T$ is a matrix transformation induced by
$A= \left[\begin{array}{rr}
1 & 1 \\ 0 & 1 \end{array}\right]$,
it follows from the previous theorem that all
we need to show is that
$A\vect{x}=0$ has the unique solution $\vect{x}=0$.
\pause
We do this in the standard way, by taking the augmented
matrix of the system $A\vect{x}=0$ and putting it in 
reduced row-echelon form.
\pause
\[ \left[\begin{array}{rr|r}
1 & 1 & 0 \\ 0 & 1 & 0 \end{array}\right]
\pause
\rightarrow
\left[\begin{array}{rr|r}
1 & 0 & 0 \\ 0 & 1 & 0 \end{array}\right].
\]
\pause

From this we see that the system has unique solution
$\vect{x} = \left[ \begin{array}{c} 0 \\ 0 \end{array} \right]$,
and therefore $T$ is a one-to-one.
\end{solution}
}
}
%--------------------------end slide -----------------------------%

%\section{Onto}
%%-------------------- start slide ------------------------%
%\frame{\frametitle{Surjections}
%\begin{definition}
%Let $T: \RR^n\to\RR^m$ be a linear transformation.
%\pause
%We say that $T$ is a \alert{surjection} or \alert{onto} if, for every $\vect{b} \in \RR^m$ there exists
%an $\vect{x}\in\RR^n$ so that $T(\vect{x}) =\vect{b}$.
%\end{definition}
%\pause
%\begin{example}
%Let $T:\RR^2\to\RR^2$ be the linear transformation 
%defined by
%\[ T\left[\begin{array}{c} a\\ b\end{array}\right] 
%=
%\left[\begin{array}{c} a+b\\ 0\end{array}\right]
%\mbox{ for all }\left[\begin{array}{c} a\\ b\end{array}\right]\in\RR^2.\]
%\pause
%Then $T$ is \alert{not onto}.  
%\pause 
%To see why, choose
%$\vect{b}=\left[\begin{array}{c} 0\\ 1\end{array}\right] \in\RR^2$.
%\pause
%Then there is no vector $\vect{x}\in\RR^2$ so that $T(\vect{x})=\vect{b}$; applying
%$T$ to any vector results in a vector whose second entry is \alert{$0$},
%and the second entry of $\vect{b}$ is $1$.
%\end{example}
%}
%%--------------------------end slide -----------------------------%
%
%%-------------------- start slide ------------------------%
%\frame{
%\begin{example}[continued]
%Consider the system $A\vect{x}=\vect{b}$, where $A$ is the matrix induced by $T$,
%\pause
%\vspace*{-.1in}
%
%\[ A=\left[\begin{array}{rr} 1 & 1 \\ 0 & 0\end{array}\right].\]
%
%\pause
%
%Then the augmented matrix is 
%\[ \left[\begin{array}{rr|r} 1 & 1 & 0 \\ 0 & 0 & 1\end{array}\right], \]
%\pause
%which is already in reduced row-echelon form.
%\pause
%The fact that this system is inconsistent implies that $T$ is 
%not onto.
%\end{example}
%\pause
%\begin{theorem}\em
%Let $A$ be an $m\times n$ matrix.
%\pause
%Then the transformation $T_A$, induced by $A$, is onto if and only
%if $A\vect{x}=\vect{b}$ is consistent for every vector $\vect{b}$ in $\RR^m$.
%\end{theorem}
%}
%%--------------------------end slide -----------------------------%
%
%%-------------------- start slide ------------------------%
%\frame{
%\begin{problem}\em
%Show that the transformation defined by 
%\[ T \left[ \begin{array}{c} x \\ y \end{array} \right]
%= \left[ \begin{array}{rr}
%1 & 2 \\ 3 & 5 
%\end{array} \right]
%\left[ \begin{array}{c} x \\ y \end{array} \right]
%\]
%is onto.
%\end{problem}
%\pause
%\begin{solution}\em
%Since $T:\RR^2\to\RR^2$, we must show that for every
%$\vect{b}=\left[ \begin{array}{c} a \\ b \end{array} \right]\in\RR^2$,
%the system $A\vect{x}=\vect{b}$ is consistent.
%\pause
%\medskip
%
%Putting the augmented matrix of $A\vect{x}=\vect{b}$ into row-echelon form,
%\pause
%\[
%\left[ \begin{array}{rr|c}
%1 & 2 & a \\ 3 & 5 & b
%\end{array} \right]
%\pause
%\rightarrow
%\left[ \begin{array}{rr|c}
%1 & 2 & a \\ 0 & -1 & b-3a
%\end{array} \right]
%\pause
%\rightarrow
%\left[ \begin{array}{rr|c}
%1 & 2 & a \\ 0 & 1 & 3a-b
%\end{array} \right].
%\]
%\pause
%We see that the system is consistent for all values of $a$ and $b$,
%and therefore $T$ is onto.
%\end{solution}
%}
%%--------------- end slide----------------%

\section{Examples}

%----------------start slide---------------%
\frame{\frametitle{Not one-to-one}
\begin{problem}\em
Let $T$ be the linear transformation induced by 
$A=\left[ \begin{array}{rrr}
1 & 2 & -1 \\ 3 & 5 & 0
\end{array} \right]$.
Show that $T_A$ is not one-to-one.
\end{problem}
\pause
\begin{solution}\em
Let $R$ be a row-echelon form of $A$.
\[ 
A=\left[ \begin{array}{rrr}
1 & 2 & -1 \\ 3 & 5 & 0
\end{array} \right]
\rightarrow
\left[ \begin{array}{rrr}
1 & 2 & -1 \\ 0 & 1 & -3
\end{array} \right]=R
\]
%For every $\vect{b}$ in $\RR^2$, the rank of the augmented matrix 
%$[A|\vect{b}]$ is equal to two, which is the rank of $A$.
%Therefore, the system $A\vect{x}=\vect{b}$ is consistent for every $\vect{b}$,
%so $T_A$ is onto.
\bigskip

Since $A$ has rank two, $A\vect{x}=0$ has infinitely many solutions,
so $\vect{x}=0$ is not the only solution. 
Therefore, $T_A$ is not one-to-one.
\end{solution}
}
%---------------------end slide-------------------------------%

%------------------------start slide--------------------------------%
{\small
\frame{\frametitle{One-to-one}
\begin{problem}\em
Let $T$ be the linear transformation induced by 
$A=\left[ \begin{array}{rr}
1 & -1 \\ 2 & 2 \\ -1 & 2
\end{array} \right]$.
Show that $T_A$ is one-to-one.
\end{problem}
\pause
\begin{solution}\em
Let $R$ be a row-echelon form of $A$.
\[
A=\left[ \begin{array}{rr}
1 & -1 \\ 2 & 2 \\ -1 & 2
\end{array} \right]
\rightarrow
\left[ \begin{array}{rr}
1 & -1 \\ 0 & 1 \\ 0 & 0
\end{array} \right] =R\]
\vspace*{-.1in}

%There exist vectors $\vect{b}\in\RR^3$ for which the rank of
%$[A|\vect{b}]$ will be equal to three, while the rank of $A$ is
%only two.  
%Therefore, the system $A\vect{x}=\vect{b}$ is not consistent for every $\vect{b}$,
%so $T_A$ is not onto.
%\bigskip

Since $A$ has rank two, every variable in $A\vect{x}=0$ 
is a leading variable,
so $\vect{x}=0$ is the unique solution.
Therefore, $T_A$ is one-to-one.
\end{solution}
}
}
%-----------------------end slide----------------------%

%------------------------start slide--------------------------------%
\frame{\frametitle{One-to-one and onto}
\begin{problem}\em
Let $T$ be the linear transformation induced by 
$A=\left[ \begin{array}{rr}
1 & -1 \\ 2 & -1 
\end{array} \right]$.
Show that $T_A$ is one-to-one and onto.
\end{problem}
\pause
\begin{solution}\em
Let $R$ be a row-echelon form of $A$.
\[
A=\left[ \begin{array}{rr}
1 & -1 \\ 2 & -1
\end{array} \right]
\rightarrow
\left[ \begin{array}{rr}
1 & -1 \\ 0 & 1
\end{array} \right] = R\]
In this case, $A$ is invertible, so $A\vect{x}=\vect{b}$ has a \alert{unique} solution
$\vect{x}$ for every $\vect{b}$ in $\RR^2$.
Therefore $T_A$ is both one-to-one and onto.
\end{solution}
}
%-----------------------end slide----------------------%

%------------------------start slide--------------------------------%
{\footnotesize
\frame{\frametitle{Neither one-to-one nor onto}
\begin{problem}\em
Let $T$ be the linear transformation induced by 
$A=\left[ \begin{array}{rrr}
1 & -1 & 1 \\ -1 & 2 & 1 \\ 1 & 0 & 3 
\end{array} \right]$.
Show that $T_A$ is neither one-to-one nor onto.
\end{problem}
\pause
\begin{solution}\em
Let $R$ be a row-echelon form of $A$.
\[ 
A=\left[ \begin{array}{rrr}
1 & -1 & 1 \\ -1 & 2 & 1 \\ 1 & 0 & 3
\end{array} \right]
\rightarrow
\left[ \begin{array}{rrr}
1 & -1 & 1 \\ 0 & 1 & 2 \\ 0 & 0 & 0
\end{array} \right]=R\]
\vspace*{-.1in}

Since $A$ has rank two, the augmented matrix $[A|\vect{b}]$ will have rank
three for some choice of $\vect{b}\in\RR^3$, resulting in $A\vect{x}=\vect{b}$ being
inconsistent.
Therefore, $T_A$ is not onto.
\medskip

The augmented matrix $[A|0]$ has rank two, so the system $A\vect{x}=0$ has
a non-leading variable, and hence does not have unique solution $\vect{x}=0$.
Therefore, $T_A$ is not one-to-one.
\end{solution}
}}
%-----------------------end slide----------------------%

\end{document}
