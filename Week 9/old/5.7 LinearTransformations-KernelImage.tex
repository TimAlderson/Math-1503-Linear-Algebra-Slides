%%%%%%%%%%%%%%%%%%%%%%
%%%%%%%%%%%%%%%%%%%%%%
%%Options for presentations (in-class) and handouts (e.g. print). 
\documentclass[pdf
,handout
]{beamer}
\usepackage{pgfpages}
\pgfpagesuselayout{2 on 1}[letterpaper,border shrink=5mm]

\graphicspath{{../}}
%%%%%%%%%%%%%%%%%%%%%%
%% Change this for different slides so it appears in bar
\usepackage{authoraftertitle}
\date{Linear Transformations: Kernel and Image}

%%%%%%%%%%%%%%%%%%%%%%
%% Upload common style file
\usepackage{../LyryxLinearAlgebraSlidesStyle}

\begin{document}
	
	%%%%%%%%%%%%%%%%%%%%%%%
	%% Title Page and Copyright Common to All Slides	
	%Title Page
	\input ../frontmatter/titlepage.tex	
	%LOTS Page
	%\input frontmatter/lyryxopentexts.tex	
	%Copyright Page
	\input ../frontmatter/copyright.tex	
	%%%%%%%%%%%%%%%%%%%%%%%%%

\section{Definition}

%-----------------start slide-------------%
\frame{\frametitle{Kernel and Image}

\begin{definition}[Kernel]
  Let $V$ be a subspace of $\mathbb{R}^n$ and $W$ a subspace of $\mathbb{R}^m$, and
  let $T: V \mapsto W$ be a linear transformation. 
\pause

Then the \alert{kernel} of $T$, $\ker(T)$, consists of all $\vect{v} \in V$ such that $T(\vect{v}) = \vect{0}$. 

\pause

\[
\ker(T) = \left\{ \vect{v} \in V : T(\vect{v}) = \vect{0} \right\}
\]
\end{definition}

\medskip
\pause

\begin{definition}[Image]
  Let $V$ be a subspace of $\mathbb{R}^n$ and $W$ a subspace of $\mathbb{R}^m$, and
  let $T: V \mapsto W$ be a linear transformation. 
\pause

Then the \alert{image} of $T$, $\im(T)$, consists of all $\vect{w} \in W$ such that $\vect{w} = T(\vect{v})$ for some $\vect{v} \in V$. 

\pause

\[
\im(T) = \left\{ T(\vect{v}) : \vect{v} \in V \right\}
\]

\end{definition}

}
%----------------end slide---------------%

%--------------start slide-----------------%
\frame{\frametitle{Problem to Try}

\begin{problem}
  Let $V$ be a subspace of $\mathbb{R}^n$ and $W$ a subspace of $\mathbb{R}^m$, and
let $T: V \mapsto W$ be a linear transformation.  

\medskip
\pause

Show that $\ker(T)$ is a subspace of $V$ and $\im(T)$ is a subspace of $W$. 
\end{problem}

}
%------------end slide-----------------%

\section{Finding the Kernel and Image}

%--------start slide----------------%
\frame{
\begin{example}
Let $T: \mathbb{R}^3 \mapsto \mathbb{R}^2$ be defined by 
\[
T \left( \leftB \begin{array}{r}
a \\
b \\
c
\end{array} \rightB \right) = \leftB \begin{array}{c}
a + b + c \\
c - a
\end{array}\rightB
\]
Then $T$ is a linear transformation. Find a basis for $\ker(T)$ and $\im(T)$. 
\end{example}

\medskip
\pause

\begin{solution}\em
You can (and should!) verify that $T$ is a linear transformation.
\end{solution}
}
%-------------end slide--------------%

%--------------start slide-----------%
\frame{

\begin{solution}[continued]\em
\textbf{Kernel of $T$:}
We look for all vectors $\vect{x} \in \mathbb{R}^3$ such that $T(\vect{x}) = \vect{0}$. % for $\vect{0} \in \mathbb{R}^2$. 
\vspace{-.1in}
{\small
\[
T \left( \leftB \begin{array}{r}
a \\ 
b \\
c
\end{array}\rightB \right) =  \leftB \begin{array}{c}
a + b + c \\
c - a
\end{array}\rightB =  \leftB \begin{array}{c}
0 \\
0
\end{array}\rightB
\]
}
\pause
%\medskip
%\vspace{-.1in}
This gives a system of equations:
%\vspace{-.1in}
\begin{eqnarray*}
a + b + c &=& 0 \\
c - a &=& 0
\end{eqnarray*}

\pause
%\medskip
%\vspace{-.1in}
The general solution is
%\vspace{-.1in}
{\small
\[
\left( \leftB \begin{array}{r} a \\ b \\ c \end{array}\rightB \right) =
\left\{ \leftB \begin{array}{r} t \\ -2t \\ t \end{array}\rightB : t \in \mathbb{R} \right\} =
\left\{ t \leftB \begin{array}{r} 1 \\ -2 \\ 1 \end{array}\rightB : t \in \mathbb{R} \right\}
\]
}
%\vspace{-.2in}
And therefore a basis for the kernel is {\small $
\left\{ \leftB \begin{array}{r}
1 \\
-2 \\
1
\end{array} \rightB \right\}.
$
}
\end{solution}
}
%------------end slide----------------%

%--------------start slide-----------%
\frame{

\begin{solution}[continued]\em
\textbf{Image of $T$:}
We can write the image as %{\small
\[ \begin{array}{cl}
\im(T) & = \left\{ \leftB \begin{array}{c}
a + b + c \\
c - a
\end{array}\rightB :  a,b,c  \in \mathbb{R} \right\} \\
& = \left\{ \leftB \begin{array}{c} a \\ -a  \end{array} \rightB
+ \leftB \begin{array}{c} b \\ 0  \end{array} \rightB
+ \leftB \begin{array}{c} c \\ c  \end{array} \rightB
:  a,b,c  \in \mathbb{R} \right\} \\
& = \left\{ a \leftB  \begin{array}{c} 1 \\ -1  \end{array} \rightB
+ b \leftB \begin{array}{c} 1 \\ 0  \end{array} \rightB
+ c \leftB \begin{array}{c} 1 \\ 1  \end{array} \rightB
:  a,b,c  \in \mathbb{R} \right\}
\end{array}
\]
%}

\pause
%\medskip

Thus $\im(T) = \mbox{span}\; \left\{ 
\leftB \begin{array}{r}
1 \\
-1 
\end{array}\rightB, 
\leftB \begin{array}{r}
1 \\
0 
\end{array}\rightB, 
\leftB \begin{array}{r}
1 \\
1 
\end{array}\rightB
\right\} $.

\pause
%\medskip

These vectors are not linearly independent, but the first two are so a basis for the image of $T$ is 
{\small
  \[
\left\{ \leftB \begin{array}{r}
1 \\
-1
\end{array} \rightB, \leftB \begin{array}{r} 
1 \\
0
\end{array}\rightB
\right\}.
\]
}

\end{solution}
}
%------------end slide----------------%

%-------------start slide------------%
\frame{\frametitle{Kernel and One to One}

\begin{alertblock}{}
The kernel of a linear transformation gives important information
about whether the transformation is one to one. Recall that a linear
transformation $T$ is one to one if and only if $T(\vect{x}) = \vect{0}$ implies $\vect{x} = \vect{0}$.
\end{alertblock}

\medskip
\pause

\begin{theorem}[Dimension Theorem]
  Let $T: V \mapsto W$ be a linear transformation where
  $V$ is a subspace of $\mathbb{R}^n$ and $W$ a subspace of $\mathbb{R}^m$. \\
  Then $T$ is one to one if and only if $\ker(T) = \left\{ \vect{0} \right\}$. 
\end{theorem}
}
%-----------------end slide------------%

\section{Dimension}

%----------------start slide-----------%
\frame{\frametitle{Dimension of the Kernel and Image}

\begin{theorem}[]\em
  Let $T: V \mapsto W$ be a linear transformation where
  $V$ is a subspace of $\mathbb{R}^n$ and $W$ is a subspace of $\mathbb{R}^m$. 
 Suppose further that the dimension of $V$ is $k$. Then
\[
k = \dim (\ker(T)) + \dim (\im(T))
\]
\end{theorem}

\medskip
\pause

\begin{corollary}[]\em
Let $T, V, W$ be defined as above, with $\dim(V) = k$. Then
\begin{eqnarray*}
\dim(\ker(T)) & \leq   k & \leq   n\\
\dim(\im(T))  & \leq   k & \leq   n \\
\end{eqnarray*}
\end{corollary}
}
%-------------------end slide----------%

%-----------------start slide---------%
\frame{
\begin{example}[Revisited]
Let $T: \mathbb{R}^3 \mapsto \mathbb{R}^2$ be defined by 
\[
T \left( \leftB \begin{array}{r}
a \\
b \\
c
\end{array} \rightB \right) = \leftB \begin{array}{c}
a + b + c \\
c - a
\end{array}\rightB
\]
Find the dimension of $\ker(T)$ and $\im(T)$.
\end{example}

\medskip
\pause

\begin{solution}\em
We already know that a basis for the kernel of $T$ is given by 
\[
\left\{ \leftB \begin{array}{r}
1 \\
-2 \\
1
\end{array} \rightB \right\}
\]

\medskip
\pause

Therefore $\dim(\ker(T)) = 1$. 
\end{solution}
}
%-----------------end slide-----------%

%---------------start slide-----------%
\frame{
\begin{solution}[continued]\em
We also found a basis for the image of $T$ as 
\[
\left\{ \leftB \begin{array}{r}
1 \\
-1
\end{array} \rightB, \leftB \begin{array}{r} 
1 \\
0
\end{array}\rightB
\right\}
\]
and this of course shows $\dim(\im(T)) = 2$. 

\medskip
\pause

But we could have found the dimension of $\im(T)$ without finding a basis. That's because 
since the dimension of $\mathbb{R}^3$ is $3$, and the dimension of $\ker(T)$ is 1, we get by the Dimension Theorem that:
\begin{eqnarray*}
\dim (\im(T)) &=& \dim (\mathbb{R}^3) - \dim (\ker(T)) \\
  &=& 3-1 =2 \\ 
\end{eqnarray*}
\end{solution}
}
%----------------end slide------------%
\end{document}
