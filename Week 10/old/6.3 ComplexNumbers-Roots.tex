%%%%%%%%%%%%%%%%%%%%%%
%%Options for presentations (in-class) and handouts (e.g. print). 
\documentclass[pdf
%,handout
]{beamer}
\usepackage{pgfpages}
%\pgfpagesuselayout{2 on 1}[letterpaper,border shrink=5mm]

\graphicspath{{../}}
%%%%%%%%%%%%%%%%%%%%%%
%% Change this for different slides so it appears in bar
\usepackage{authoraftertitle}
\date{6.3: Complex Numbers; Roots of Complex Numbers}

%%%%%%%%%%%%%%%%%%%%%%
%% Upload common style file
\usepackage{../LyryxLinearAlgebraSlidesStyle}

\begin{document}
	
	%%%%%%%%%%%%%%%%%%%%%%%
	%% Title Page and Copyright Common to All Slides	
	%Title Page
	\input ../frontmatter/titlepage.tex	
	%LOTS Page
	%\input frontmatter/lyryxopentexts.tex	
	%Copyright Page
	\input ../frontmatter/copyright.tex	
	%%%%%%%%%%%%%%%%%%%%%%%%%

\section{Roots of Complex Numbers}
%-------------- start slide -------------------------------%
\frame{\frametitle{Roots of Complex Numbers}
\begin{definition}
Let $z$ and $q$ be complex numbers, and let $n$ be a positive 
integer.  
\pause
Then $z$ is called \alert{an $n^{th}$ root of $q$} 
if $z^n=q$.
\end{definition}
\pause
\begin{alertblock}{De Moivre's Theorem and its implication}
If $\theta$ is any angle and $n$ is a positive integer,
$\left( e^{i \theta} \right)^n = e^{i n \theta}$.
This implies that
for any real number $r>0$ and any positive integer $n$, 
\[ (re^{i\theta})^n =r^ne^{i n\theta}. \]
This leads to the following result.
\end{alertblock}
\pause
\begin{corollary}\em
Let $q$ be a nonzero complex number and $n$ a positive integer.
Then $z^n=q$ has exactly $n$ complex solutions, i.e., $q$ 
has exactly $n$ complex $n^{th}$ roots.
%Let $z$ be a non zero complex number.
%\index{complex numbers ! roots}Then there are always exactly $k$ many  $k^{th}$
%roots of $z$ in $\mathbb{C}$.
\end{corollary}
}
%--------------end slide----------------------------%

%-------------- start slide -------------------------------%
\frame{
\begin{example}
For any positive real number $a$, $z^2=a$ has two complex
(in this case, real) solution, $z=\sqrt{a}$ and $z=-\sqrt{a}$.
This is equivalent to the statement that $a$ has two complex
(in this case, real) square roots.
\pause
\begin{itemize}
\item
One particular example: $25$ has two square roots, $5$ and $-5$, and
these are the two solutions to $z^2=25$.
\pause
\item
But we all knew that. A more interesting example is that $-1$ has no
real square roots, but suddenly it has two (complex) square roots, $i$
and $-i$.  These are the two (complex) solutions to $z^2=1$.
\end{itemize}
\end{example}
}
%--------------end slide----------------------------%

%-------------- start slide -------------------------------%
{\small
\frame{\frametitle{Cube Roots}
\begin{example}
To find the (three) cube roots of $i$, we solve the equation 
$z^3=i$.
\pause
To do so, we express both $z$ and $i$ in polar form: convert
$i$ to polar form, and write $z=re^{i\theta}$, giving us
\vspace*{-.25in}

\[ (re^{i\theta})^3 =\pause e^{\pi i/2}.\]
\vspace*{-.2in}

\pause
Thus
$r^3e^{3i\theta}=1e^{\pi i/2}$, 
\pause
implying that
$\textcolor{blue}{r^3=1}
\pause
\mbox{ and } \alert{3\theta = \frac{\pi}{2}}$.
\pause
\begin{itemize}
\item \textcolor{blue}{Since $r$ is a non-negative real number, $r^3=1$ implies 
that $r=1$.}
\pause
\item The statement $\alert{3\theta = \frac{\pi}{2}}$
is \alert{not completely correct}.
\pause
The problem that arises is that the argument for $i$, $\frac{\pi}{2}$
is not unique.
\pause
Instead, we could have written 
\vspace*{-.1in}

\[ i=e^{5\pi i/2}\mbox{ or } i=e^{9\pi i/2}\mbox{ or } i=e^{-3\pi i/2}.\]
\vspace*{-.2in}

\pause
\alert{In fact, there are infinitely many choices for the argument.}
\pause
The important thing to notice is that any two different arguments differ
by a multiple of $2\pi i$, 
\pause
and thus we may write
\vspace*{-.1in}

\[ 3\theta = \frac{\pi}{2} + 2\pi\ell,~\ell\in\ZZ.\]
\vspace*{-.2in}

\pause
($\ZZ$ denotes the set of integers:
$\{ \ldots, -3, -2, -1, 0,1, 2,3,\ldots\}$).
\end{itemize}
\end{example}
}}
%-----------------------end slide---------------------------%

%----------------------start slide -------------------------%
{\small
\frame{
\begin{example}[continued]
Dividing both sides of $3\theta = \frac{\pi}{2} + 2\pi\ell$
by $3$:
\[ \theta = \frac{\pi}{6} + \frac{2}{3}\pi\ell,\]
where $\ell$ is any integer.
\pause
The Corollary to De Moivre's Theorem tells us that there are only
\alert{three} different cube roots.
\pause
These are obtained by using $\ell=0$, $\ell=1$, and $\ell=2$, 
resulting in three values of $\theta$:
\pause
\[ \frac{\pi}{6}, \frac{5\pi}{6}, 
\mbox{ and } \frac{9\pi}{6}=\frac{3\pi}{2}.\]
\pause
Thus the cube roots of $i$ are
\[ e^{\pi i/6}, e^{5\pi i/6}, \mbox{ and } e^{3\pi i/2}.\]
\pause
We now convert these to Cartesian form.
\end{example}
}}
%-----------------------end slide---------------------------%

%----------------------start slide -------------------------%
{\small
\frame{
\begin{example}[continued]
\begin{eqnarray*}
e^{\pi i/6} & = & \pause \frac{\sqrt{3}}{2} +\frac{1}{2}i, \\
\pause
e^{\pi i/6} & = & \pause -\frac{\sqrt{3}}{2} +\frac{1}{2}i, \\
\pause
e^{3\pi i/2} & = & \pause -i .
\end{eqnarray*}
\pause
\textcolor{blue}{You can check your work by computing the cube of each of
these.}
\end{example}

\pause

\begin{alertblock}{}
This process is summarized in the following procedure.
\end{alertblock}
}}
%-----------------------end slide---------------------------%
%------------start slide------------------------%
\frame{
\begin{alertblock}{Finding Roots of a Complex Number}
Let $w$ be a complex number. We wish to find the $n^{th}$ roots of $w$, that is all $z$ such that $z^n = w$. 

There are $n$ distinct $n^{th}$ roots and they can be found as follows:. 
\pause
\begin{enumerate}
\item[1.] Express both $z$ and $w$ in polar form $z=re^{i\theta}, w=se^{i\phi}$. Then $z^n = w$ becomes:
\[
(re^{i\theta})^n = r^n e^{i n \theta} = se^{i\phi}
\]
We need to solve for $r$ and $\theta$. 
\pause
\item[2.] Solve the following two equations:
\begin{eqnarray*}
r^n &=& s 
\end{eqnarray*}
\begin{eqnarray}
e^{i n \theta} &=& e^{i \phi}
\label{rootseqns}
\end{eqnarray}
\end{enumerate}
\end{alertblock}
}
%------------------end slide------------------%

%-----------------start slide-----------------%
\frame{
\begin{alertblock}{Continued}
\begin{enumerate}
\item[3.] The solutions to $r^n = s$ are given by $r = \sqrt[n]{s}$. 
\pause
\item[4.] The solutions to $e^{i n \theta} = e^{i \phi}$ are given by:
\[
n\theta = \phi + 2\pi \ell,  \; \mbox{for} \; \ell = 0,1,2, \cdots, n-1
\]
or
\[
\theta = \frac{\phi}{n} + \frac{2}{n} \pi \ell, \; \mbox{for} \; \ell = 0,1,2, \cdots, n-1 
\]
\pause
\item[5.]
Using the solutions $r, \theta$ to the equations given in (\ref{rootseqns})
construct the $n^{th}$ roots of the form $z = re^{i\theta}$.  
\end{enumerate}
\end{alertblock}
}
%-------------------end slide------------------------%

%-------------- start slide -------------------------------%
{\small
\frame{
\begin{problem}\em
Find all complex numbers $z$ such that
$z^4=2(\sqrt{3}i-1)$,
and express each in the form $a+bi$.
\end{problem}
\pause
\begin{solution}\em
\begin{enumerate}
\item[1.] Convert $2(\sqrt{3}i-1)=-2+2\sqrt{3}i$ to polar form:
\pause
\[ |z^4|=\sqrt{(-2)^2+(2\sqrt{3})^2} =\sqrt{16}=4. \]

\pause
If $\phi$ is an argument for $-2+2\sqrt{3}i$, then
\[\cos\phi = \frac{-2}{4}=-\frac{1}{2}
\mbox{ and }
\sin\phi = \frac{2\sqrt{3}}{4}=\frac{\sqrt{3}}{2},
\mbox{ so } \phi = \frac{2\pi}{3}.\]  
\pause 
Thus
$z^4 = 4 e^{2\pi i/3}$.
\pause
Let $z=re^{i\theta}$. 
\pause
\item[2.] 
The equation becomes $r^4e^{i4\theta} =  4 e^{2\pi i/3}$, so we need to solve
\begin{eqnarray*}
r^4&=&4 \\
e^{i4\theta} &=& e^{2\pi i/3}
\end{eqnarray*}
\end{enumerate}
\end{solution}
}
%----------------------end slide----------------------%

%--------------------start slide----------------%
\frame{
\begin{solution}[continued]\em
\begin{enumerate}
\item[3.]
Since $r^4=4$, $r^2=\pm 2$.  But $r$ is \alert{real}, and so
$r^2=2$, implying $r=\pm\sqrt{2}$.
However $r\geq 0$, and therefore $r=\sqrt{2}$.
\pause
\item[4.]
The solutions to $e^{i4\theta} = e^{2\pi i/3}$ are given by 
\[ 4\theta = \frac{2}{3}\pi +2\pi \ell, \ell=0, 1, 2, 3.\]
\pause
Therefore, 
\[ \theta =  \frac{2\pi}{12} + \frac{2\pi \ell}{4} 
 =  \frac{\pi}{6} + \frac{\pi \ell}{2}  
 =  \frac{\pi(3\ell+1)}{6},
\mbox{ for } \ell =0,1,2,3.\]

\end{enumerate}
\end{solution}
}}
%--------------------end slide--------------------%

%-------------start slide----------------------%
\frame{
\begin{solution}[continued]\em
\begin{enumerate}
\item[5.]
Thus $r=\sqrt{2}$ and
$\theta= \left(\frac{3\ell +1}{6}\right)\pi$, $\ell =0,1,2,3$.
\pause
Converting to Cartesian form:
\pause
\[ \begin{array}{llll} 
\ell =0: & z=\sqrt{2}e^{\pi i/6} & =
\sqrt{2}(\frac{(\sqrt{3}}{2} + \frac{1}{2}i) &
=\frac{\sqrt{6}}{2} + \frac{\sqrt{2}}{2}i \\ 
\pause
\ell =1: & z=\sqrt{2}e^{2\pi i/3} & =
\sqrt{2}(-\frac{1}{2} + \frac{\sqrt{3}}{2}i)  &
= -\frac{\sqrt{2}}{2} + \frac{\sqrt{6}}{2}i\\
\pause
\ell =2: & z=\sqrt{2}e^{7\pi i/6} & =
\sqrt{2}(-\frac{\sqrt{3}}{2} - \frac{1}{2}i)  &
=-\frac{\sqrt{6}}{2} - \frac{\sqrt{2}}{2}i \\ 
\pause
\ell =3: & z=\sqrt{2}e^{5\pi i/3} & =
\sqrt{2}(\frac{1}{2} - \frac{\sqrt{3}}{2}i)  &
= \frac{\sqrt{2}}{2} - \frac{\sqrt{6}}{2}i\\
\end{array}\]
\pause

Therefore, the fourth roots of $2(\sqrt{3}i-1)$ are:
\[ \frac{\sqrt{6}}{2} + \frac{\sqrt{2}}{2}i,
\pause
\textcolor{blue}{-\frac{\sqrt{2}}{2} + \frac{\sqrt{6}}{2}i},
\pause
-\frac{\sqrt{6}}{2} - \frac{\sqrt{2}}{2}i,
\pause
\textcolor{blue}{\frac{\sqrt{2}}{2} - \frac{\sqrt{6}}{2}i}.\]
\end{enumerate}
\end{solution}
}
%-------------- end slide -------------------------------%


%--------------- start slide ------------------------%
{\small
\frame{\frametitle{Roots of Unity}
\begin{definition}
A complex number $z$ is a \alert{root of unity} if there
exists a positive integer $n$ so that $z^n=1$.
\end{definition}
\pause
\begin{problem}\em
Find the sixth roots of unity, i.e., all solutions to $z^6=1$.
\end{problem}
\pause
\begin{solution}\em
Write $z=re^{i\theta}$ and convert $1$ to polar form to get
\vspace*{-.1in}

\[ (re^{i\theta})^6 = e^{i0},\mbox{ and so }
r^6e^{6\theta i} = e^{i0}.\]
\vspace*{-.2in}

\pause
Equating the absolute values and arguments,
\vspace*{-.1in}

\[ r^6=1 \mbox{ and } 6\theta = 0 + 2\pi \ell,~\ell=0,1,2,3,4,5.\]
\vspace*{-.2in}

\pause
Since $r$ is real, $r=1$. 
\pause
The six arguments for the solutions are
\vspace*{-.1in}

\[ \theta=\frac{2\pi\ell}{6}=\frac{\pi\ell}{3},~\ell = 0,1,2,3,4,5.\]

\end{solution}
}}
%-----------------------end slide---------------------------%

%--------------- start slide ------------------------%
{\small
\frame{
\begin{solution}[continued]\em
The six arguments for the solutions are
\[ \theta=\frac{2\pi\ell}{6}=\frac{\pi\ell}{3},~\ell = 0,1,2,3,4,5.\]
\pause
Converting these to Cartesian form:
\pause
\[ \begin{array}{l|l|l}
\ell & \theta & z \\ \hline \hline
% & & \\
0 & 0 & e^{0i}=1 \\
\pause
\textcolor{blue}{1} & \textcolor{blue}{\frac{\pi}{3}} &
\textcolor{blue}{e^{\pi i/3}=\frac{1}{2} + \frac{\sqrt{3}}{2}i} \\
\pause
2 & \frac{2\pi}{3} & e^{2\pi i/3}=-\frac{1}{2} + \frac{\sqrt{3}}{2}i \\
\pause
\textcolor{blue}{3} & \textcolor{blue}{\pi} &
\textcolor{blue}{e^{\pi i}=-1} \\
\pause
4 & \frac{4\pi}{3} & e^{4\pi i/3}=-\frac{1}{2} - \frac{\sqrt{3}}{2}i \\
\pause
\textcolor{blue}{5} & \textcolor{blue}{\frac{5\pi}{3}} & 
\textcolor{blue}{e^{5\pi i/3}=\frac{1}{2} - \frac{\sqrt{3}}{2}i} 
\end{array}\]
\pause
If you graph these six point in the complex plane, you'll see
that they result in six equally spaced points on the unit circle,
one of them being $(1,0)$.
\end{solution}
}}
%-------------- end slide -------------------------------%

%-------------- start slide -------------------------------%
\frame{
\begin{alertblock}{Roots of Unity}
For any integer $n\geq 1$, the (complex) solutions to $z^n=1$ are
\[ z=e^{2\pi\ell i / n} \mbox{ for } \ell=0,1,2,\ldots,n-1.\]
\pause
Furthermore, the $n^{th}$ roots of unity correspond to $n$ 
equally spaced points on the unit circle, one of them being
$(1,0)$.
\end{alertblock}
}
%-------------- end slide -------------------------------%


\end{document}
