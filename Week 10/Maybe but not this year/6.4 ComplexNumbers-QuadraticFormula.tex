%%%%%%%%%%%%%%%%%%%%%%
%%Options for presentations (in-class) and handouts (e.g. print). 
\documentclass[pdf
,handout
]{beamer}
\usepackage{pgfpages}
\pgfpagesuselayout{2 on 1}[letterpaper,border shrink=5mm]

\graphicspath{{../}}
%%%%%%%%%%%%%%%%%%%%%%
%% Change this for different slides so it appears in bar
\usepackage{authoraftertitle}
\date{Complex Numbers: The Quadratic Formula}

%%%%%%%%%%%%%%%%%%%%%%
%% Upload common style file
\usepackage{../LyryxLinearAlgebraSlidesStyle}

\begin{document}
	
	%%%%%%%%%%%%%%%%%%%%%%%
	%% Title Page and Copyright Common to All Slides	
	%Title Page
	\input ../frontmatter/titlepage.tex	
	%LOTS Page
	%\input frontmatter/lyryxopentexts.tex	
	%Copyright Page
	\input ../frontmatter/copyright.tex	
	%%%%%%%%%%%%%%%%%%%%%%%%%

\section{The Quadratic Formula}
%-------------- start slide -------------------------------%
\frame{\frametitle{Real Quadratics}
\begin{definition}
A \alert{real} quadratic is an expression of the form
$ax^2+bx+c$ where $a, b, c\in\RR$ and $a\neq 0$.
\end{definition}

\pause
To find the roots of a real quadratic, we can either
factor by inspection, or use the \alert{quadratic formula}:
\[ 
x= \frac{-b \pm\sqrt{b^2-4ac}}{2a} \]

\pause
The expression $b^2-4ac$ in the quadratic formula is called
the \alert{discriminant}, \pause and
\begin{itemize}
\item
if $b^2-4ac\geq 0$, then the roots of the quadratic are \alert{real};
\pause
\item
if $b^2-4ac<0$, then the quadratic has \alert{no real roots}.
\end{itemize}
}
%-------------- end slide -------------------------------%

%-------------- start slide -------------------------------%
\frame{
\begin{definition}
A real quadratic $ax^2+bx+c$ is called
\alert{irreducible} if the discriminant is less than zero,
i.e., $b^2-4ac<0$.
\end{definition}
\pause
\begin{alertblock}{}
Notice that if $b^2-4ac<0$, then
\[
\sqrt{b^2-4ac}=\sqrt{(-1)(4ac-b^2)} = (\pm) i\sqrt{4ac-b^2}.\]
\end{alertblock}
\pause
It follows that the roots of an irreducible quadratic are
\[ \frac{-b\pm i\sqrt{4ac-b^2}}{2a}
=\left\{ \begin{array}{l}\vspace*{.02in}
-\frac{b}{2a} + \frac{\sqrt{4ac-b^2}}{2a}i \\
-\frac{b}{2a} - \frac{\sqrt{4ac-b^2}}{2a}i
\end{array}\right. ,
\]
\pause
and we see that
the two roots are complex conjugates of each other.
\pause

We denote the two roots by
\[ u = -\frac{b}{2a} + \frac{\sqrt{4ac-b^2}}{2a}i 
\mbox{ and }
\overline{u}= -\frac{b}{2a} - \frac{\sqrt{4ac-b^2}}{2a}i. \]
}
%-------------- end slide -------------------------------%

%-------------- start slide -------------------------------%
\frame{\frametitle{Real Quadratics with Complex Roots}
\begin{example}
The quadratic $x^2-14x+58$ has roots
\begin{eqnarray*}
x & = & \frac{14 \pm\sqrt{196-4\times 58}}{2}\\
\pause
& = & \frac{14 \pm\sqrt{196-232}}{2} \\
\pause
& = & \frac{14 \pm\sqrt{-36}}{2} \\
\pause
& = & \frac{14 \pm 6i}{2} \\
\pause
& = & 7\pm 3i,
\end{eqnarray*}
\pause
so the roots are $7+3i$ and $7-3i$.
\end{example}
}
%-------------- end slide -------------------------------%

%-------------- start slide -------------------------------%
\frame{
\begin{alertblock}{}
Conversely, given $u=a+bi$ with $b\neq 0$, there is an 
irreducible quadratic having roots $u$ and $\overline{u}$.
\end{alertblock}
\pause
\begin{problem}\em
Find an irreducible quadratic with $u=5-2i$ as a root.
What is the other root?
\end{problem}
\pause
\begin{solution}\em
\vspace*{-.2in}

\begin{eqnarray*}
(x-u)(x-\overline{u}) & = & \pause (x-(5-2i))(x-(5+2i)) \\
\pause
& = & x^2 -(5-2i)x -(5+2i)x +(5-2i)(5+2i) \\
\pause
& = & x^2-10x+29.
\end{eqnarray*}
\pause
Therefore, $x^2-10x+29$ is an irreducible quadratic with
roots $5-2i$ and $5+2i$.
\pause

\alert{Notice that $-10 = -(u+\overline{u})$ and 
$29=u\overline{u}=|u|^2$.}
\end{solution}
}
%-------------- end slide -------------------------------%

%-------------- start slide -------------------------------%
\frame{
\begin{block}{Exercise}
Find an irreducible quadratic with root $u=-3+4i$, and 
find the other root.
\end{block}
\begin{block}{Answer}
$x^2+6x+25$ has roots $u=-3+4i$ and $\overline{u}=-3-4i$.
\end{block}
}
%-------------- end slide -------------------------------%

%-------------- start slide -------------------------------%
\frame{\frametitle{Quadratics with Complex Coefficients}
\begin{problem}\em
Find the roots of the quadratic $x^2-(3-2i)x+(5-i)=0$.
\end{problem}
\pause
\begin{solution}\em
Using the quadratic formula
\[ x=\frac{3-2i \pm\sqrt{ (-(3-2i))^2-4(5-i)}}{2}. \]
\pause
Now,
\[  (-(3-2i))^2-4(5-i) = 5-12i-20+4i=-15-8i,\]
\pause
so
\[ x=\frac{3-2i \pm\sqrt{-15-8i}}{2}. \] 
\pause
To find $\pm\sqrt{-15-8i}$, solve $z^2=-15-8i$ for $z$.
\end{solution}
}
%-------------- end slide -------------------------------%

%-------------- start slide -------------------------------%
\frame{
\begin{solution}[continued]\em
Let $z=a+bi$ and $z^2=-15-8i$.  
\pause
Then
\[ (a^2-b^2)+2abi=-15-8i,\]
\pause
so $a^2-b^2=-15$ and $2ab=-8$.
\medskip

\pause
Solving for $a$ and $b$ gives us
$z= 1-4i, -1+4i$, i.e., $z=\pm(1-4i)$.
\pause

Therefore, 
\[ x=\frac{3-2i \pm(1-4i)}{2}, \] 
\pause
and
\[\begin{array}{l}
\frac{3-2i +(1-4i)}{2}
=\frac{4-6i}{2} = 2-3i,\\ \\
\frac{3-2i -(1-4i)}{2}
=\frac{2+2i}{2}= 1+i.
\end{array}\]
\pause
Thus the roots of $x^2-(3-2i)x+(5-i)$ are 
$2-3i$ and $1+i$.
\end{solution}
}
%-------------- end slide -------------------------------%

%-------------- start slide -------------------------------%
\frame{
\begin{block}{Exercise}
Find the roots of $x^2-3ix+(-3+i)$.
\end{block}
\begin{block}{Answer}
$1+i$ and $-1+2i$.
\end{block}
}
%-------------- end slide -------------------------------%

%-------------- start slide -------------------------------%
\frame{
\begin{problem}\em
Verify that $u_1=(4-i)$ is a root of 
\[ x^2-(2-3i)x-(10+6i) \]
and find the other root, $u_2$.
\end{problem}
\pause
\begin{solution}\em
First,
\vspace*{-.3in}

\begin{eqnarray*}
u_1^2 -(2-3i)u_1 -(10+6i) 
& = & (4-i)^2-(2-3i)(4-i)-(10+6i)\\
& = & (15-8i)-(5-14i)-(10+6i) \\
& = &  0,
\end{eqnarray*}
\vspace*{-.3in}

so $u_1=(4-i)$ is a root.
\end{solution}
}
%-------------- end slide -------------------------------%

%-------------- start slide -------------------------------%
\frame{
\begin{solution}[continued]\em
Recall that if $u_1$ and $u_2$ are the roots of the quadratic,
then
\[ u_1 + u_2 = (2-3i) \mbox{ and } u_1u_2 = -(10+6i). \]
Solve for $u_2$ using either one of these equations.
\pause

Since $u_1=4-i$ and $u_1+u_2=2-3i$,
\[ u_2=2-3i-u_1=2-3i-(4-i)=-2-2i.\]
\pause
Therefore, the other root is $u_2=-2-2i$.
\pause

You can easily verify your answer by computing $u_1u_2$:
\[ u_1u_2=(4-i)(-2-2i)= -10 -6i=-(10+6i).\]
\end{solution}
}
%-------------- end slide -------------------------------%

\end{document}
